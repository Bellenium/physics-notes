\chapter{Addition of Angular Momentum}
\section{The General Problem}
How do we add an arbitray $J_{1}$ and $J_{2}$? If we did what are their $\hat{J}^{2}$ and $\hat{J}_{z}$ eigenvalues like? One way to do this would be to construct square matrices of the dimension $2j +1$. However for this we would first need to know the allowed values of $j$. This we conjecture to be
\begin{equation}
j_{1} \otimes j_{2} = (j_{1} + j_{2}) \oplus (j_{1} + j_{2} - 1) \oplus ... \oplus (j_{1} - j_{2})
\end{equation}
For our example, the total number of kets is:
$$\ket{jm, j_{1}j_{2}}$$
where
$$j_{1} + j_{2} \geq j \leq j_{1} - j_{2}$$
$$j \geq m \leq -j$$
with the normalization conditions:
$$\alpha + \beta = 0$$
$$\alpha^{2} + \beta^{2} = 1$$
\subsection{Clebsch-Gordan Coeeficients}
The completeness of the product kets allows us to write,
\begin{equation}
\ket{jm,j_{1}j_{2}} = \sum_{m_{1}} \sum_{m_{2}} \braket{j_{1}m_{1},j_{2}m_{2}}{jm, j_{1}j_{2}} \ket{j_{1}m_{1},j_{2}m_{2}}
\end{equation}
Here the coffecients of the expansion are termed "Clebsch-Gordan" coefficients. Here are some of their properties:
\begin{itemize}
\item $\braket{j_{1}m_{1},j_{2}m_{2}}{jm}\neq 0$ if and only if $j_{1} - j_{2}\leq j \leq j_{1} + j_{2}$ or $m_{1} + m_{2} = m$
\item $\in \mathbb{R}$
\item $\braket{j_{1}m_{1},j_{2}m_{2}}{jm} > 0$
\item $\braket{j_{1}m_{1},j_{2}m_{2}}{jm} = {(-1)}^{j_{1} + j_{2} - j}\braket{j_{1}(-m_{1}),j_{2}(-m_{2})}{j(-m)}$
\end{itemize}

\subsection{Modified Spectroscopic Notation}
In absence of spin using s,p,d.. is sufficient to describe angular momentum. However, in the presence of spin we change out notation to:
\begin{itemize}
\item Use capital letters S,P,D or L typically to indicate the value of angular momentum
\item Append a subscript $J$ to the right of $L$ i.e. $L_{J}$ to indicate the $j$ value
\item Append a superscript $2S+1$ to the left of $L$ i.e $\prescript{2S+1}{}L$. to indicate the degeneracy due to spin projections
\end{itemize}
%\section{Irreducible Tensor Operators}
%\subsection{Tensor Operators}
%We know that a vector can be linearly expanded in terms of it's basis,
%\begin{equation}
%\ket{V} = \sum^{3}_{i} v_{i}\ket{i}
%\end{equation}
%A second rank tensor similarly can be expressed as,
%%	\ket{T^{(2)}} = \sum^{3}_{i}\sum^{3}_{j} t_{ij}\ket{i}\otimes\ket{j}
%\end{equation}
%\begin{itemize}
%\item 
%\end{itemize}