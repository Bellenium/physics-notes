\documentclass[]{article}
\usepackage{lipsum}
%\usepackage[utf8]{inputenc}
%\usepackage{graphicx}
\usepackage{amsmath}
\usepackage{amssymb}
\usepackage{physics}
\usepackage{tcolorbox}
\usepackage{color}   %May be necessary if you want to color links
\usepackage{hyperref}
\usepackage{mathtools}
\usepackage{graphicx} % Allows including images
\usepackage{booktabs} % Allows the use of \toprule, \midrule and \bottomrule in tables
\usepackage{xmpmulti}
\usepackage{tikz}
\usepackage{color}
\usepackage{caption}
%\DeclareCaptionFont{red}{\color{red}}
\captionsetup{font={small,sf},labelfont={bf},margin=1em}
\usepackage{subcaption}
\usepackage{cite}

%opening
\title{Notes on Variational Calculus}
\author{Pugazharasu A D}

\begin{document}

\maketitle

\begin{abstract}

\end{abstract}

\section{Introduction}
\section{Statement of the Problem}
\section{The Euler-Lagrangian Equation}
\section{The "Second Form" of the Euler Equation}
\section{The "$\delta$" Notation}
\section{Special Cases}
\subsection{$F$ Does Not Contain $y$ Explicitly}
\subsection{$F$ Does Not Contain $x$ Explicitly}
\section{Some extensions}
\subsection{Several Dependent Variables}
\subsection{Several Independent Variables}
\subsection{Higher-Order Derivatives}
\subsection{Variable End-Points}
\section{Constrained Variation}
\section{Physical Variational Principles}
\subsection{Fermat's Principle in Optics}
\subsection{Hamilton's Principle in Mechanics}
\section{General Eigenvalue Problem}
\section{Estimation of Eigenvalues and Eigenfunctions}
\section{Adjustment of Parameters}
\section*{References}
\end{document}
