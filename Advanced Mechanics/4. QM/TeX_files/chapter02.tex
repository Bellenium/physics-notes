\chapter{Formalism}
\section{State Vector}
In Quantum Mechanics, we start with an object called the state vector $\ket{\psi}$. All the information about the system is contained in it. The position basis representation of the state vector is called the wavefunction $\psi (\vec{x}, t)$. \\
If we wish to know about a particular physical measurable such as an object's position of momentum, we can extract this information from the State vector by means of acting on with an Operator that corresponds to the measurable quantity.\\
\section{Observables}
To get down to even more specifics if I consider an observable $\hat{O}$, then in general I have the form:
\begin{equation}
	\hat{O} \ket{\psi} = o \ket{\psi}
\end{equation}
Where, $o$ is an Eigenvalue . The only types of operators that are constrained in such a fashion are "Hermitian Operators", they are identified with the condition:
\begin{equation}
	content...
\end{equation}
\section{Time Evolution}
\subsection{Schrodinger Picture}
Where\\
If we consider the Schrodinger picture i.e. the State vector evovles with time whereas the Observables are in a loose sense eternal. The time evolution of the state vector is given by the Schrodinger equation:
\begin{equation}
	i \hbar \frac{\partial \ket{\psi}}{\partial t} = \hat{H} \ket{\psi}
\end{equation}
Or,
\begin{equation}
	i \hbar \frac{\partial \psi}{\partial t} = \hat{H} \psi
\end{equation}
in terms of the Wavefunction. Where, $\hat{H}$ is the Hamiltonian operator, which can be expressed as:
\begin{equation}
	\hat{H} = -\frac{\hbar^{2} \nabla^2}{2m} + V(\vec{x})
\end{equation}
for a free particle. 
\section{Measurement}
Measurement is defined as a form of time-evolution that is non-unitary and non-deterministic. 
According to Born's rule
\begin{equation}
	\int_{a}^{b} \abs{\psi(\vec{x}, t)}^{2} dx = \text{Probability of finding the particle at a time t between positions a and b}
\end{equation}
Thus, . Physically speaking this lends a kind of indeterminancy to the wavefunction. We can only speak of probabilities. Therefore, we can only , this brings to the measurement hypothesis, that is the State vector evolves to the state corresponding to the measurement being made. And unlike the Schrodinger equation, this evolution is non-deterministic. This tension is often called the "measurement problem", i.e. why is the measurement of an observable a special process distinct from others? Several theories and models claim to have resolved this, but we shall save that discussion for another time. We will fully focus on understanding the theory of Quantum Mechanics in a pragmatic lens before we question its foundations (although the converse isn't necessarily a bad thing, it isn't the purpose of this manuscript).
\section{Normalization}
Normalization is a process through which we ensure that,
\begin{equation}\label{norm}
	\int_{- \infty}^{\infty} \abs{\psi(\vec{x}, t)}^{2} dx = 1
\end{equation}
This is a natural consequence of Born's rule, we simply want all the probabilities to add up to 1. Thus, to rule out any other absurd scenarios, we make a ruling that non-Normalizable and non-square integrable Wavefunctions are unphysical.\\
We can also prove that once normalized, the wavefunction always remains normalized, we start by differentiating \ref{norm} with respect to time\\
$$\frac{d}{dt} \int_{- \infty}^{\infty} \abs{\psi(\vec{x}, t)}^{2} dx = \frac{\partial}{\partial t}\int_{- \infty}^{\infty} \abs{\psi(\vec{x}, t)}^{2} dx$$
Dealing with the term inside the integral,
$$ \frac{\partial}{\partial t}\abs{\psi(\vec{x}, t)}^{2} = \frac{\partial}{\partial t} (\psi^{*} \psi ) = \psi^{*}\frac{\partial \psi}{\partial t} + \psi\frac{\partial \psi^{*}}{\partial t}$$
Now the Schrodinger equation for a free particle reads as,
$$\frac{\partial \psi}{\partial t} = \frac{i \hbar}{2m}\frac{\partial^{2} \psi}{\partial x^{2}} - \frac{i}{\hbar}V \psi$$
Conjugating this we can see that,
$$\frac{\partial \psi^{*}}{\partial t} = -\frac{i \hbar}{2m}\frac{\partial^{2} \psi^{*}}{\partial x^{2}} + \frac{i}{\hbar}V \psi^{*}$$
Thus, () becomes,
$$\frac{\partial}{\partial t}\abs{\psi(\vec{x}, t)}^{2} = \frac{i \hbar}{2m}\left( \psi^{*} \frac{\partial^{2} \psi}{\partial x^{2}} - \psi \frac{\partial^{2} \psi^{*}}{\partial x^{2}} \right) = \frac{\partial }{\partial x}\left[\frac{i \hbar}{2m}\left( \psi^{*} \frac{\partial \psi}{\partial x} - \psi \frac{\partial \psi^{*}}{\partial x} \right)\right]$$
Now we evaluate the integral,
$$\frac{d}{dt} \int_{- \infty}^{\infty} \abs{\psi(\vec{x}, t)}^{2} dx = \frac{i \hbar}{2m}\left( \psi^{*} \frac{\partial \psi}{\partial x} - \psi \frac{\partial \psi^{*}}{\partial x} \right)^{\infty}_{- \infty} $$
But $\psi$ must go to zero as goes to infinity, otherwise the wave function would not be normalizable. Thus it follows that.
\begin{equation}
	\frac{d}{dt} \int_{- \infty}^{\infty} \abs{\psi(\vec{x}, t)}^{2} dx = 0
\end{equation}
And hence, the integral is constant i.e. independent of time. Therefore if is normalized at a time $t = 0$, it remains normalized for all future. 
\section{Summary of Postulates}
\section{Generalized Uncertainty Principle}
Suppose we have a ket $\ket{\psi}$ and two operators $\hat{A}$ and $\hat{B}$, we define two new vectors

$$,$$

$$,$$

We use the Cauchy-Shwarz inequality,

$$ 2|X||Y| \geq |\langle X|Y \rangle + \langle Y|X \rangle |$$

Substituting in the left-hand side,
$2\sqrt{\langle X|X\rangle\langle Y|Y\rangle} \geq |\langle X| Y  \rangle+ \langle Y | X \rangle|$
Plugging in Eqs. (4) and (5),
$2\sqrt{\langle \psi |A^{2} |\psi \rangle \langle \psi | -B^{2}| \psi \rangle} \geq |\langle X | Y \rangle + \langle Y | X \rangle |$
Taking the $-1$ outside,
$2i\sqrt{\langle \psi |A^{2} |\psi \rangle \langle \psi | B^{2}| \psi \rangle} \geq |\langle X | Y \rangle + \langle Y | X \rangle|$
We now substitute in the right hand of the equation
$2i\sqrt{\langle \psi |A^{2} |\psi \rangle \langle \psi | B^{2}| \psi \rangle} \geq | \langle {\psi} |\hat{A}\hat{B}| {\psi} \rangle - \langle {\psi} |\hat{B}\hat{A}| {\psi} \rangle|$
The negative sign is due to the $i$, this also seems to represent the commutator, so we substitute
$2i\sqrt{\langle \psi |A^{2} |\psi \rangle \langle \psi | B^{2}| \psi \rangle }\geq |\langle\psi |[\hat{A},\hat{B}]|\psi\rangle$
Again, the right hand side looks like the expectation value of a quantity, so
$2i\sqrt{\langle A^{2} \rangle \langle B^{2} \rangle} \geq |\langle [\hat{A},\hat{B}] \rangle |$
$\sqrt{\langle A^{2} \rangle \langle B^{2} \rangle} \geq \frac{1}{2i} |\langle [\hat{A},\hat{B}] \rangle |$
We use Eq. (2),

$\sqrt{\sigma_{A}^{2}\sigma_{B}^{2}} \geq \frac{1}{2i} |\langle [\hat{A},\hat{B}] \rangle|$
Removing the square root we get the expression:
$\sigma_{A}\sigma_{B} \geq \frac{1}{2i} |\langle[\hat{A}, \hat{B}]\rangle|$

This is called the generalized uncertainty principle. This basically states that two variables that do not commute cannot be measured with precision simultaneously.

Talking about position and momentum

We know that observable properties can be represented using operators, here we'll 

$\hat{x} = x$
$\hat{P} = -i\hbar \frac{\partial}{\partial x}$
So we now try to find the commutator now
$[\hat{x}, \hat{p}] = \hat{x}\hat{p} - \hat{p}\hat{x}$
$[\hat{x}, \hat{p}] = -ix\hbar \frac{\partial}{\partial x} + i\hbar \frac{\partial}{\partial x}$
Now let's apply this to state vector to obtain the expectation value
$[\hat{x}, \hat{p}] |\psi\rangle = -ix\hbar \frac{\partial}{\partial x} |\psi\rangle + i\hbar \frac{\partial x|\psi\rangle}{\partial x}$
$$[\hat{x}, \hat{p}] |\psi\rangle = -ix\hbar \frac{\partial}{\partial x} |\psi\rangle + ix\hbar \pdv{(|\psi\rangle)}{x} + i\hbar$$
$[\hat{x}, \hat{p}] |\psi\rangle = i\hbar$
Substituting this into Eq.(),
$\sigma_{x}\sigma_{p} \geq \frac{1}{2i} i\hbar$
$\sigma_{x}\sigma_{p} \geq \frac{\hbar}{2}$ 
$\sigma_{x}\sigma_{p} \geq \frac{h}{4 \pi}$
\section{Generalized Statistical Interpretation}
If you measure an observable $\hat{O}$ on a particle in the state $\psi()$, you will certainly get one of the eigenvalues of the observable. If the spetra is discrete, the probability of getting the particular eigenvalue $q_{n}$ associated with the orthonormalized eigenfunction $f_{n}(x)$ is
\begin{equation}
	P(q_{n}) = \abs{c_{n}}^{2} = \abs{\braket{f_{x}}{\psi}}^{2}
\end{equation}
\subsection{Position Measurements}

\subsection{Momentum Measurements}
