\chapter{Epilogue: What lies ahead} \label{epi}
With all of what we have  discussed, whilst being the most experimentally accurate theory, Quantum Mechanics still remains incomplete due to three key issues that arise from internal consistency and consistency with other theories such relativity: 
\begin{itemize}
\item \textbf{Locality:} Why do non-local effects arise in Quantum Mechanics? Are they artifacts of our ignorance or are they real?
\item \textbf{Measurement:} Why is measurement distinct from time evolution? Why is it stochastic?
\item \textbf{Ontology:} Is the wavefunction a calulative device or does it actually exist?
\end{itemize}
This is a realm that might quickly slip into philosophy\footnote{For a brief overview refer to \cite{p27} and \cite{p28}} so we only mention points here and point to reading material since a subject of this depth deserves several volumes to dissect. Various different interpretations and formulations exist in the community that solve one or more of these problems. Few have suggested that Quantum Mechanics is truly deterministic at heart \footnote{See \cite{p26} and \cite{p24}}, few more suggest that the measurement problem has a great deal to do with the effect of gravity or some other novel mechanism\footnote{See \cite{p2}}, Rovelli suggests that Quantum Mechanics is about how one system is related to another \footnote{Read \cite{p25}} and a new proposal even goes on to state that classical mechanics is non-deterministic \footnote{Read \cite{p29}}! 
\\
As exemplified by the diversity of proposals, there is absolutely zero consensus in the community as to which approach is the most fruitful \footnote{See \cite{p31}}. Maybe in the future, people will laugh at our ignorance and foolishness. But as form of anaesthasia \footnote{Or as David Mermim would call it a "pillow"}, I would quote,
\begin{tcolorbox}
\begin{center}
"The aspiration to truth is more precious than its assured possession" 
\end{center}
- Gotthold Lessing
\end{tcolorbox}
