\documentclass[]{article}
\usepackage{lipsum}
%\usepackage[utf8]{inputenc}
%\usepackage{graphicx}
\usepackage{amsmath}
\usepackage{amssymb}
\usepackage{physics}
\usepackage{tcolorbox}
\usepackage{color}   %May be necessary if you want to color links
\usepackage{hyperref}
\usepackage{mathtools}
\usepackage{graphicx} % Allows including images
\usepackage{booktabs} % Allows the use of \toprule, \midrule and \bottomrule in tables

%opening
\title{Tensors}
\author{Pugazharasu A D}

\begin{document}

\maketitle

\begin{abstract}

\end{abstract}
\section{Vector Transformation Rules}
The rules:
\begin{itemize}
	\item For basis vectors forward transformations brings us from old to new coordinate systems and backward brings us from new to old.
	\item However, with vector components it's the opposite.\\
\end{itemize}

Suppose we have a vector $\vec{v}$ in a basis $\vec{e}_j$. We now transform it to a basis $\tilde{\vec{e}}_i$ where it becomes $\tilde{v}$. We call the forward transformation as $F_{ij}$ and the backward as $B_{ij}$ which we define as:
$$\tilde{\vec{e}}_j = \sum_{i = 1}^{n} F_{ij} \vec{e}_i$$
$$\vec{e}_j = \sum_{i = 1}^{n} B_{ij} \tilde{\vec{e}}_i$$
We can try to derive the statements made previously,
$$\vec{v} = \sum_{j = 1}^{n} v_{j}\vec{e}_j = \sum_{i = 1}^{n} \tilde{v}_{i}\tilde{\vec{e}}_i$$
$$\vec{v} = \sum_{j = 1}^{n} v_{j}\vec{e}_j = \sum_{j = 1}^{n} {v}_{j}(\sum_{i=1}^{n} B_{ij} \tilde{\vec{e}}_i) =  \sum_{i = 1}^{n} \sum_{j = 1}^{n} (B_{ij} {v}_{j}) \tilde{\vec{e}}_i$$
Thus,
\begin{equation}
	\tilde{v}_i = \sum_{j = 1}^{n} B_{ij} {v}_{j}
\end{equation}
Similarly,
$$\vec{v} = \sum_{j = 1}^{n} v_{j}\vec{e}_j = \sum_{i = 1}^{n} \tilde{v}_{i}\tilde{\vec{e}}_i$$
$$\vec{v} = \sum_{j = 1}^{n} \tilde{v}_{j}\tilde{\vec{e}}_j = \sum_{j = 1}^{n} \tilde{v}_{j}(\sum_{i = 1}^{n} F_{ij} \vec{e}_i) =  \sum_{i = 1}^{n} \sum_{j = 1}^{n} (F_{ij} \tilde{v}_{j}) {\vec{e}}_i$$
Thus,
\begin{equation}
	{v}_i = \sum_{j = 1}^{n} F_{ij} \tilde{v}_{j}
\end{equation}
Now because vector components beehave contrary to the basis vectors, they are said to be \textit{\textbf{"Contravariant"}}
\section{Index Notation}
\subsection{Einstein Notation i.e. Summing convention}
Let us consider the sum\footnote{Mind you there are no exponents there.},
$$x_{i} = \sum_{j}^{n}\Lambda_{ij}\mathcal{X}^{j}$$
Is the same as,
$$x_{i} = \Lambda_{ij}\mathcal{X}^{j}$$ 
Here, we define $i$ to be the free index and $j$ to be the summing index or the dummy index that is repeated to signify so.
\subsection{Index Convention}
When we sum from $1$ to $3$ we use the symbols $i,j$ and $k$ i.e. the English alphabet to signify that we are only considering dimnesions that are spatial/that are not a timme dimension. However, when we use the symbols $\nu$ and $\mu$ i.e. Greek alphabets we are summing from 0 to 3, we also include the temporal dimension according to the tradition of special relativity in which we name components as $\{x^{0}, x^{1}, x^{2}, x^{3}\} = \{t, x, y, z\}$ in the Cartesian framework.

\section{Covectors}
\begin{itemize}
	\item Covectors can be thought of as row vector or as functions that act on Vectors such that any covector $\alpha : \mathbb{V} \rightarrow \mathbb{R}$
	\item Covectors are linear maps i.e. $ \beta ( \alpha ) \vec{v} = \beta \alpha \vec{v}$ and $(\beta + \alpha ) \vec{v} = \alpha \vec{v} + \beta \vec{v}$
	\item Covectors are elements of a Dual vector space $\mathbb{V}^*$ which has different rules for addition and scaling i.e. scalar multiplication
	\item You visualize covectors to be some sort of gridline on your vector space such that applying a covector to a vector is equivalent to projecting the vector along the gridline
	\item Covectors are invariant but their components are not
	\item The covectors that form the basis for the set of all covectors is called the \textit{\textbf{"Dual Basis"}}, because they are a basis for the Dual Space $\mathbb{V}^*$ i.e. any covector can be expressed as the linear combination of the dual basis
	\item However we are free to choose a dual basis
	\item For covector components, forward transformation brings us from old to new and backwards vice versa
	\item We can flip between row and column vectors for an orthonormal basis
	\item Vector components are measured by counting how many are used in the construction of a vector, but covector components are measured by counting the number of covector lines that the basis vector pierces
	\item The covector basis transforms contravariantly compared to the basis and it's components transform covariantly according to the basis 
\end{itemize}


\subsection{Contravariant Components}
We denote contravariant components using the symbols
$$A^i$$
and their basis like
$$\overrightarrow{e}_i$$
\subsection{Covariant Components}
We denote convariant components using the symbols
$$A_i$$
and their basis like
$${\overrightarrow{e}}^i$$
\subsection{Relationship Between the Two Types of Components}

$$|\overrightarrow{e}^1| = \frac{1}{|\overrightarrow{e}_1|\cos(\theta_1)}$$
and,
$$|\overrightarrow{e}_1| = \frac{1}{|\overrightarrow{e}^1|\cos(\theta_1)}$$
Or with 3 components we have:
Since both types of components represent the same vector (as in same magnitude) only inn  different bases, we can write
$$\overrightarrow{A} = A^{i}\overrightarrow{e}_i = A_{i}\overrightarrow{e}^i$$
\subsection{Using Cramer's Method to find Components}
\section{Linear Maps}
Linear maps to put it naively, Linear Maps transform input vectors but not the basis. Geometrically speaking, Linear Maps:
\begin{itemize}
	\item Keep gridlines parallel
	\item Keep gridlines evenly spaced
	\item Keep the origin stationary
\end{itemize}
To put it more abstractly, Linear Maps:
\begin{itemize}
	\item Maps vectors to vectors, $\mathbb{L}: \mathbb{V} \rightarrow \mathbb{V}$
	\item Adds inputs or outputs, $\mathbb{L}(\vec{V} + \vec{W}) = \mathbb{L}(\vec{V}) + \mathbb{L}(\vec{W})$
	\item Scale the inputs or outputs, $\mathbb{L}(\alpha \vec{V}) = \alpha \mathbb{L}(\vec{V})$
	\item i.e. They are Linear/Linearity
\end{itemize}
When I transform the basis using a forward transformation, the transformed Linear map $\tilde{\mathbb{L}^{l}_{i}}$ can be written as:
\begin{equation}
	\tilde{\mathbb{L}^{l}_{i}} = \mathbb{B}^{l}_{k} \mathbb{L}^{k}_{j} \mathbb{F}^{j}_{i}
\end{equation}

\section{Metric Tensor}
\begin{itemize}
	\item Pythagoras' theorem is a lie for non-orthonormal bases
	\item The metric Tensor is Tensor that helps us compute lengths and angles
	\item For two dimensions it can be written as:
\end{itemize}
$$\textit{g}_{ij} = \begin{bmatrix}
e_{1}e_{1} & e_{1}e_{2} \\
e_{2}e_{1} & e_{2}e_{2} 
\end{bmatrix}$$
\begin{itemize}
	\item Or more abstractly
\end{itemize}
$$\textit{g}_{ij} = e_{i}e_{j}$$
\begin{itemize}
	\item The dot product between two vectors can be written as
\end{itemize}
$$||\vec{v}|| ||\vec{w}||\cos{\theta} = v^{i}w^{j}\textit{g}_{ij}$$
\begin{itemize}
	\item we can see how this allows us to compute angles as well
	\item To transform the components of the Metric Tensor we have to apply the transformation twice i.e.$\tilde{g}_{\rho \sigma} = \mathbb{F}^{\mu}_{\rho} \mathbb{F}^{\nu}_{\sigma}\tilde{g}_{\mu \nu}$ or $g_{\rho \sigma} = \mathbb{B}^{\mu}_{\rho} \mathbb{B}^{\nu}_{\sigma}\tilde{g}_{\mu \nu}$
\end{itemize}

\section{Bilinear Forms}

\section{Tensor Products}

\section{Definition of a Tensor}

\section{Raising and Lowering Indices}
\section{Covariant Quantities}
\section{Contravariant Quantities}
\section{Clearer Definitions}
\section{Covariant Differentiation}

\end{document}
