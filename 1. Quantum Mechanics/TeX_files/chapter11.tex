\chapter{Relativistic Quantum Mechanics}
\section{The Klein-Gordon Equation}
In order to develop quantum mechanics relativistically, they started from the relativistic
energy–momentum relation,
\begin{equation}
	E^2 - \big|\vec{p}\big|^2c^2 -m^2c^4 = 0
\end{equation}
Substituting $E$ and $p$ for their respective operators,

$$E \Rightarrow i\hbar\frac{\partial }{\partial t},\hspace{50pt}\vec{p} \Rightarrow -i\hbar\vec{\nabla},$$ 

and letting the equation act on a wavefunction $\phi(\vec{x} , t)$  the equation becomes,
\begin{equation}
	\Big( -\hbar^2\frac{\partial^2 }{\partial t^2} + \hbar^2c^2\nabla^2 - m^2c^4 \Big)\phi(\vec{x} , t) = 0
\end{equation}

This is the Klein-Gordon Equation. The wave function $\phi(\vec{x} , t)$ is also an object called a field because its arguments extend over all space and time. Another way to say this is that it exists
throughout spacetime and its fluctuations are described by the Klein–Gordon
equation. In natural units,
$$\Big( \frac{\partial^2}{\partial t^2} + \nabla^2 - m^2 \Big)\phi(\vec{x},t) = 0$$

This equation was the first attempt at a relativistic quantum mehcanical equation of a wave. it tells us how the field of a particle fluctuates.

The Klein-Gordon equation can also be written in the Lorentz-invariant form,
\begin{equation}
	(\partial^\mu\partial_\mu+m^2)\psi=0
\end{equation}

Where,
\begin{equation*}
	\partial^\mu\partial_mu\equiv\frac{\partial^2}{\partial t^2}- \frac{\partial^2}{\partial x^2}-\frac{\partial^2}{\partial y^2}-\frac{\partial^2}{\partial z^2}
\end{equation*}
The Klein-Gordon equation has plane wave solutions,
\begin{equation}
	\psi(x,t)=Ne^{i(p\cdot x-Et)}
\end{equation}

Substituting in Equation (3),
\begin{equation*}
	E^2\psi=p^2\psi+m^2\psi
\end{equation*}
\section{The Dirac Equation}
\section{Covariant Formalism}
\subsection{The Adjoint Spinor and the Covariant Current}
\section{Solutions to the Dirac Equation}
\subsection{Particles at Rest}
\subsection{General Free-Particle Solutions}
\section{Antiparticles}
\subsection{The Dirac Sea Interpretation}
\subsection{The Feynman-Stuckelberg Interpretation}
\subsection{Antiparticle Spinors}