\chapter{Groups}
This chpater is a summary of \cite{Nathan}, Chapters 2, 5, 6, 7, 8 and 9 from \cite{Artin}, Chapter 2 from \cite{Stillwell} and chapters 1 and 2 from 
\section{What is a group?}
A group is a system/algebraic structure that consists of a collection of actions/a set and an operator/binary operator to combine actions satisfying certain axioms (see below), in such a way that the result of two actions combined using a the operator is yet another action in the set.
\subsection{Group "Rules"}
\begin{enumerate}
	\item There is a predefined list of actions that never changes.
	\item Every action is reversible.
	\item Every action is deterministic.
	\item Any sequence of consecutive actions is also an action
\end{enumerate}
\subsection{Group Axioms}
A group is an algebraic structure $(\mathcal{G},\circ)$ which satisfies the following four conditions:
\begin{enumerate}
	\item \textbf{Closure:} $\forall \ a,b \in \mathcal{G}: a \circ b \in \mathcal{G}$ 
	\item \textbf{Associativity:} $\forall \ a, b, c \in \mathcal{G}: a \circ (b \circ c)  = (a \circ b) \circ c $   	  
	\item \textbf{Identity:} $\exists \ e \in \mathcal{G}: \forall \ a \in \mathcal{G}: e \circ a = a = a \circ e $
	\item \textbf{Inverse:} $ \forall \ a \in \mathcal{G}: \exists \  b \in \mathcal{G}: a \circ b = e = b \circ a$  	  %∀a∈G:∃b∈G:	a∘b=e=b∘a
\end{enumerate}
\section{What do groups look like?}
\subsection{Mapmaking}
\subsection{Mapping a group}
\subsection{Cayley diagrams}
\subsection{A touch more abstract}

\section{Why study groups?}
\subsection{Groups of symmetries}
\subsection{Groups of actions}
\subsection{Examples}

\section{Group Algebra}
\subsection{Multiplication tables}
\subsection{The classic definition}

\pagebreak

\section{Five families}
\subsection{Cyclic groups}
\subsection{Abelian groups}
\subsection{Dihedral groups}
\subsection{Symmetric groups}
\subsection{Alternating groups}

\section{Subgroups}
\subsection{Multiplication tables and Cayley diagrams}
\subsection{Seeing subgroups}
\subsection{Revealing subgroups}
\subsection{Cosets}
\subsection{Lagrange's theorem}

\section{Products and quotients}
\subsection{The direct product}
\subsection{Semidirect products}
\subsection{Normal subgroups and quotients}
\subsection{Normalizers}
\subsection{Conjugacy}

\section{The power of homomorphisms}
\subsection{Embeddings}
\subsection{Quotient maps}
\subsection{The Fundamental Homomorphism Theorem}
\subsection{Modular arithmetic}
\subsection{Direct products and relatively prime numbers}
\subsection{The Fundamental Theorem of Abelian Groups}
\subsection{Semidirect products revisited}

\pagebreak

\section{Sylow theory}
\subsection{Group actions}
\subsection{Approaching Sylow: Cauchy's Theorem}
\subsection{p-groups}
\subsection{Sylow Theorems}


