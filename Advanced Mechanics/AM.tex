\documentclass[a4paper,12pt]{book}
\usepackage[utf8]{inputenc}
\usepackage{bm}
\usepackage{graphicx}
\usepackage{lipsum}
%\usepackage[utf8]{inputenc}
%\usepackage{graphicx}
\usepackage{amsmath}
\usepackage{amssymb}
\usepackage{physics}
\usepackage{tcolorbox}
\usepackage{color}   %May be necessary if you want to color links
\usepackage[hidelinks]{hyperref}
\usepackage{mathtools}
\usepackage{graphicx} % Allows including images
\usepackage{booktabs} % Allows the use of \toprule, \midrule and \bottomrule in tables
\usepackage{calligra}
\usepackage{mathrsfs}
\usepackage{tabularx}
%\usepackage{rsfso}

\DeclareMathAlphabet{\mathcalligra}{T1}{calligra}{m}{n}
\DeclareFontShape{T1}{calligra}{m}{n}{<->s*[2.2]callig15}{}
\newcommand{\scriptr}{\mathcalligra{r}\,}
\newcommand{\boldscriptr}{\pmb{\mathcalligra{r}}\,}

\graphicspath{{Images/}}
\begin{document}

\author{Rishi Kumar, Pugazharasu A D}
\title{Notes on Quantum Mechanics}


\frontmatter
\maketitle
\tableofcontents
\chapter{Preface}

\mainmatter
%\part{History}
%\chapter{A Historical Overview}
Rishi's article + JP sir's slides
\subsection{Blackbody Radiation}
\subsection{The de Broglie Hypothesis}
In , the French physicist de Brogile proposed that this wave like structure applies to electrons too and follows the equation:
\begin{equation}
p = \frac{h}{\lambda} = \frac{2 \pi \hbar}{ \lambda}
\end{equation}
%\part{Tools}
%% !TeX spellcheck = <none>
\chapter{Mathematical Preliminaries}
This chapter is a discussion of all the mathematical tools and tricks one would require to master Quantum mechanics. We assume that the reader has a lucid understanding of matrices and vector calculus. If not the reader may refer to:
\begin{itemize}
	\item 
\end{itemize}
to refresh themselves or learn those concepts before 
\section{Matrix Inversion}
\section{Complex Numbers}
A complex number is an order pair ${} \in \mathbb{C}$ where $a,b \in \mathbb{R}$ where we can denote it as $z = a + ib$ where $i = \sqrt{-1}$
\subsection{Addition}
$z_{1} = a_{1} + ib_{1}, \ z_{2} = a_{2} + ib_{2}$
$$z_{1} + z_{2} =  (a_{1} + a_{2}) + i(b_{1} + b_{2})$$
\subsection{Multiplication}
$z_{1} = a_{1} + ib_{1}, \ z_{2} = a_{2} + ib_{2}$
$$z_{1}z_{2} =  (a_{1} + ib_{1})(a_{2} + ib_{2}) = (a_{1}a_{2} - b_{1}b_{2}) + i(a_{1}b_{2} + a_{2}b_{1})$$
\subsection{Properties}
Where, $\mathcal{W}, \mathcal{Z}, \lambda \in \mathbb{C}$
\subsubsection{Commutativity}
$$\mathcal{W} + \mathcal{Z} = \mathcal{Z} + \mathcal{W}$$
$$\mathcal{W}\mathcal{Z} = \mathcal{Z}\mathcal{W}$$
\subsubsection{Associativity}
$$(\mathcal{Z}_1 + \mathcal{Z}_2) + \mathcal{Z}_3 = \mathcal{Z}_1 + (\mathcal{Z}_2 + \mathcal{Z}_3)$$
$$(\mathcal{Z}_1\mathcal{Z}_2)\mathcal{Z}_3 = \mathcal{Z}_1(\mathcal{Z}_2\mathcal{Z}_3)$$
\subsubsection{Identities}
$$\mathcal{Z} + 0 = \mathcal{Z}$$
$$\mathcal{Z}1 = \mathcal{Z}$$
\subsubsection{Additive Inverse}
$$\forall \ \mathcal{Z} \ \exists \ \mathcal{Z}^{-1} \ | \ \mathcal{Z} + \mathcal{Z}^{-1} = 0$$
\subsubsection{Multiplicative Inverse}
$$\forall \  \mathcal{Z} \neq 0 \ \exists \ \mathcal{W} \ | \ \mathcal{Z}\mathcal{W} = 1$$
\subsubsection{Distributive Property}
$$\lambda(\mathcal{W} + \mathcal{Z}) = \lambda\mathcal{W} + \lambda\mathcal{Z}$$
\subsection{Notation}
\textit{\textbf{n-tuple}} refers to an ordered set of $n$ numbers over a field $\mathcal{F}$.\footnote{For our case $\mathcal{F}$ simply refers to $\mathbb{C}$}
\subsection{Wessel Plane}
\begin{figure}
	\centering
	\includegraphics[scale=0.05]{wessel-plane.png}
	\caption{Wessel Plane Plot: (Complex conjugate picture.svg from Wikimedia Commons)}
\end{figure}
\section{Linear Vector Spaces} 
A linear vector space or simply a vector space $\mathbb{V}$ is a set along with the regular multiplication and addition operations over a field $\mathcal{F}$, such that the following axioms hold: \footnote{Here, $\alpha , \beta \in \mathcal{F}$ and $\mathcal{U}, \mathcal{V} $ and $\mathcal{W} \in \mathbb{V}$} \\
\subsection{Commutativity}
$$\mathcal{U} + \mathcal{V} = \mathcal{V} + \mathcal{U}$$
\subsection{Associativity}
$$(\mathcal{U} + \mathcal{V}) + \mathcal{W} = \mathcal{V} + (\mathcal{U} + \mathcal{W})$$
$$(\alpha \beta) \mathcal{V} = \alpha (\beta \mathcal{V})$$
\subsection{Additive Identity}
$$\exists \  0 \in \mathbb{V} \ | \ \mathcal{V} + 0 = 0 + \mathcal{V} = \mathcal{V}$$
\subsection{Additive Inverse}
$$\forall \ \mathcal{V} \ \exists \ \mathcal{V}^{-1} \ | \ \mathcal{V} + \mathcal{V} = 0$$
\subsection{Multiplicative identity}
$$\exists \ 1 \in \mathbb{V} \ | \ 1 \mathcal{V} = \mathcal{V}$$
\subsection{Distributive properties}
$$\alpha (\mathcal{U} + \mathcal{V}) = \alpha \mathcal{U} + \alpha \mathcal{V}$$
$$(\alpha + \beta) \mathcal{U} = \alpha \mathcal{U} + \beta \mathcal{U}$$
\section{Inner Product Spaces}
An inner product is simply an operation that takes a Dual $\ket{\psi}$ and it's corresponding vector $\bra{\psi}$ and maps them to $\mathbb{R}$:
$$\braket{expression1}{expression2}$$
\section{Dual Spaces}
\section{Dirac Notation}
Operators are represented with respect to a particular basis (in this case $\{e_{m}, e_{n}\}$) by their matrix elements
\begin{equation}
\langle e_{m}| \hat{O} | {e_n} \rangle = \hat{O}_{mn}
\end{equation}
\section{Subspaces}
Given a vector space $\mathbb{V}$, a subset of its elements that form a vector space among themselves is called a subspace. We will denote a particular subspace $i$ of dimensionality $n_{i}$ by $\mathbb{V}^{n_{i}}_{i}$.\\
   Given two subspaces, and , we define their sum $\mathbb{V}^{n_{i}}_{i} \oplus \mathbb{V}^{m_{i}}_{i} = \mathbb{V}^{l_{i}}_{i}$ \footnote{Here $\oplus$ is the direct sum defined as: } as the set containing:
\begin{enumerate}
\item All the elements of $\mathbb{V}^{n_{i}}_{i}$
\item All the elements of $\mathbb{V}^{m_{j}}_{j}$
\item And all possible linear combinations of the above
\end{enumerate} 
However for the elements of (3), closure is lost. The dimensionality of such a subspace is $n + m$.
\section{Hilbert Spaces}
A Hilbert space $H$ is simply a normed vector space (a Banach space), whose norm is defined as:
\begin{equation} \label{norm}
\norm{V} := \sqrt{\braket{V}{V}}
\end{equation}
This is an axiomatic definition of a Hilbert space, but we are more concerned with the corollaries of it. All the Cauchy sequences \footnote{Defintion} of functions in a Hilbert space always converge to a function that is also a memeber of the space i.e. it is said to be \textbf{complete} which implies that the integral of the absolute square of a function must converege \footnote{we simply state this but a proof can be found in}
\begin{equation}
\int_{a}^{b} \abs{f(x)}^{2} dx < \infty
\end{equation}
Moreover this means that, any function in Hilbert space can eb expressed as a linear combination of other functions i.e. it is closed/complete
\begin{equation}
f(x) = \sum_{n = 1}^{\infty} c_{n} f_{n}(x)
\end{equation}
Where, $c_{n} \in \mathbb{C}$
\section{Linear Operators}
\section{Eigenvalue Problem}
\section{Eigenfunctions of a Hermitian Operator}
\section{Transformations}
\section{Active Tranformation}
In a loose sense this can be thought of as,

\section{Passive Tranformation}
From our discussion before it is also clear that the same transformation can be implemented as,
\begin{equation}
\hat{O} \rightarrow U^{\dagger}\hat{O}U
\end{equation}
This is a very different viewpoint, we can understand this by visualizing it to be a 
\subsection{Equivalence of Transformation types}
It's pretty simple to see that both types of transformation constitute the same physical picture. Thus, we can take both viewpoints to mean the same physical transformation in each case, and later on we will see how this leads us two different pictures of Quantum Mechanics and how they are related.
\section{Functions of Operators}
\section{Generalization to Infinite Dimensions}
\section{Probability}
\subsection{Discrete Variables}
Suppose we have a frequency distribution 
\begin{equation}
N = \sum_{j=0}^{\infty} N(j)
\end{equation}
The probability of $N_{j}$ is defined as,
\begin{equation}
P(j) = \frac{N(j)}{N}
\end{equation}
\subsection{Continuous Variables}
We now move to a continuous probability distribution, we'll create continuous analogs of all the quantities we just introduced. Let's start with probability, the probability of that $x$ lies between $a$ and $b$
\begin{equation}
	P_{ab} = \int_{a}^{b} \rho(x) dx
\end{equation}
where $\rho(x)$ is the called the probability density i.e. the probability of getting $x$, or more concretely,
$$\rho(x)dx = \text{Probability that an individual is chosend at random lies between } x \text{ and } x + dx$$
Now supposing the rules we held for discrete variables hold, the continuous analogs look like this:
\begin{equation}
	1 = \int_{- \infty}^{\infty} \rho(x) dx
\end{equation}
\begin{equation}
	\expval{x} = \int_{- \infty}^{\infty} x \rho(x) dx
\end{equation}
\begin{equation}
	\expval{f(x)} = \int_{- \infty}^{\infty} f(x) \rho(x) dx
\end{equation}
\begin{equation}
	\sigma^{2} := \expval{(\Delta x)^{2}} = \expval{x^{2}} - {\expval{x}}^{2}
\end{equation}
\section{Expectation Values}
In this section we'll explore how we express the expectation values of a few opeartors. Let's start with the position opeartor in the position representation (i.e. position basis):
\begin{equation} \label{posex}
	\expval{x} = \int_{- \infty}^{\infty} x \abs{\psi(\vec{x}, t)}^{2} dx
\end{equation}
We can differentiate \ref{posex} with respect to time to find the expectation value for "velocity":
$$\frac{d \expval{x}}{dt} = $$
Throwing away 
\begin{equation}
	\expval{v} = \frac{d \expval{x}}{dt} = -\frac{i \hbar}{m} \int \psi^{*} \frac{\partial \psi}{\partial x} dx
\end{equation}
Therefore we can write the expectation value of momentum as,
\begin{equation}
	\expval{p} = m \frac{d \expval{x}}{dt} =  -i \hbar \int \left(\psi^{*} \frac{\partial \psi}{\partial x} \right) dx
\end{equation}
In general, every observable is a function of position and momentum, thus for an observable $\hat{O}(x,p)$, the expectation value is given by,
\begin{equation}
	\expval{\hat{O}(x,p)} = \int \psi^{*} \hat{O}(x,-i \hbar \nabla) \psi dx
\end{equation}
For example, the expectation value of kinetic energy is,
\begin{equation}
\expval{T} = -\frac{\hbar^{2}}{2m} \int \psi^{*} \frac{\partial^{2} \psi}{\partial x^{2}} dx
\end{equation}
Or to sum it up in Dirac notation,
\begin{equation}
	\expval{\hat{O}} = \expval{\hat{O}}{\psi}
\end{equation}
\section{Fourier Analysis}
\subsection{Dirichelet's Theorem}

\subsection{Fourier Transform}

\section{Delta Function}
\subsection{The Divergence of $\frac{\hat{r}}{r^{2}}$}
We can see why the divergence is,
\begin{equation}
\nabla . \frac{\hat{r}}{r^{2}} = 0
\end{equation}
But if we calculate this using the Divergence theorem, we find that ,
\begin{equation}
	\oint v .da = \int \left( \frac{\hat{r}}{r^{2}} \right) . \left( r^{2} \sin(\theta) d \theta d \phi \hat{r} \right) = \left( \int_{0}^{\pi} \sin(\theta) d \theta \right) \left( \int_{0}^{2\pi} d \phi \right) = 4 \pi
\end{equation}
This is paradoxical. The issue is that it blows up at $r=0$ but is is neglible everywhere else. How do we fix this? The Dirac Delta functional!
\subsection{The One-Dimensional Dirac Delta Functional}
The Dirac Delta is a functional \footnote{An object that is a map between functions} which we define as,
\begin{equation} \label{deltadef}
\delta(x-a)= 
\begin{cases}
0, & \text{if } x \neq a\\
\infty,              & \text{if } x = a
\end{cases}
\end{equation}
\begin{equation}
\int_{- \infty}^{+ \infty} \delta(x-a) dx = 1
\label{del2}
\end{equation}
$\forall \  a \in \mathbb{R}$
We can visualize it as a sharp peak at $a$,
\begin{figure}
	\centering
	\includegraphics[scale=0.5]{delta-distribution.png}
	\caption{A Plot of $\delta(x)$}
\end{figure}
We can interpret \ref{del2} as saying "the area of the delta distribution is always 1".
\begin{equation}
f(x)\delta(x - a ) = f(a)
\end{equation}
We can combine these to get,
\begin{equation}
\int_{- \infty}^{+ \infty} \delta(x-a) f(x) dx = f(a)
\end{equation}
\subsubsection{A few interesting properties}
\begin{equation}
\delta(kx) = \frac{1}{|k|}\delta(x)
\end{equation}
\begin{equation}
\frac{d}{dx}(\delta(x)) = -\delta(x)
\end{equation}
where k is a constant
\begin{equation}
\frac{d \theta}{dx} = \delta(x)
\end{equation}
Where $\theta$ is the step function defined as,
\begin{equation}
\theta(x)= 
\begin{cases}
1, & \text{if } x > 0\\
o,              & \text{if } x \leq 0
\end{cases}
\end{equation}

\subsection{The Three-Dimensional Dirac Delta Function}
We generalize (\ref{deltadef}) to three dimensions,
\begin{equation}
\delta^{3}(\vec{r} - \vec{a}) = \delta(x-a_{x})\delta(y-a_{y})\delta(z-a_{z})
\end{equation}
\begin{equation}
\int_{- \infty}^{+ \infty} \delta^{3}(\vec{r} - \vec{a}) dV = 1
\end{equation}
We can also define the three-dimensional delta function as
\begin{equation}
\delta^{3}(\boldscriptr) = \frac{1}{4 \pi} \left[\nabla \cdot \left( \frac{\hat{\boldscriptr}}{{\scriptr	}^{2}}\right)\right]
\end{equation}
Since,
$$\nabla \left(\frac{1}{\scriptr}\right) = -\frac{\hat{\boldscriptr}}{\scriptr^{2}}$$
We can rewrite as,
\begin{equation}
\delta^{3}(\boldscriptr) = -\frac{1}{4 \pi} \left[\nabla^{2}  \left( \frac{1}{\scriptr}\right)\right]
\end{equation}
\section{Gaussian Integrals}
\section{The $i \epsilon$ Prescription}
We will now derive and interpret the formula:
\begin{equation}
\frac{1}{x \mp i \epsilon} = \mathscr{P} \frac{1}{x} \pm \pi \delta (x)
\end{equation}
where $\epsilon \rightarrow 0$ is a positive infinitesimally small quantity. Now we'll consider the integral
\begin{equation}
	content...
\end{equation}
$$$$
\begin{equation}
a
\end{equation}
\begin{equation}
asdfkjh
\end{equation}
\section{Permutation Functions}
\subsection{Kronecker delta}
It simply has the ‘function’ of ‘renaming’ an index:
$$\delta^{\mu}_{\nu} x^{\nu} = x^{\mu}$$
it is in a sense simply the identity matrix. Or it is sometimes defined as:
\begin{equation}
\delta_{ij} = \begin{cases}
1 \ \text{if } i = j \\
0 \ \text{if } i \neq j\\
\end{cases}
\end{equation}
\subsection{Levi-Civita Pseudotensor}
\label{Levi}
The Levi-Civita Pseudotensor i.e. Tensor density is a completely anti-symmetric i.e. $\epsilon_{ijk} = -\epsilon_{jik} = -\epsilon_{ikj} = -\epsilon_{kji}$, we define it as:
\begin{equation}
\epsilon_{ijk} = \begin{cases}
1 \ \text{if } ijk \text{ is an even permuation of } 123\\
-1 \ \text{if } ijk \text{ is an odd permuation of } 123\\
0  \text{ if two indices are equal}\\
\end{cases}
\end{equation}
\subsubsection{Identities}
\begin{equation}
\epsilon_{\alpha \beta \nu}\epsilon_{\alpha \beta \sigma} = \delta_{\mu \rho} \delta_{\nu \sigma} - \delta_{\mu \sigma}\delta_{\nu \rho}
\end{equation}
From this it follows that,
\begin{equation}
\epsilon_{\alpha \beta \nu}\epsilon_{\alpha \beta \sigma} = 2\delta_{\nu \sigma}
\end{equation}
and
\begin{equation}
\epsilon_{\alpha \beta \gamma}\epsilon_{\alpha \beta \gamma} = 6
\end{equation}
%\part{Rules}
%\chapter{Formalism}
In Quantum Mechanics, we start with an object called the state vector $\ket{\psi}$. All the information about the system is contained in it. The position basis representation of the state vector is called the wavefunction $\psi (\vec{x}, t)$. \\
If we wish to know about a particular physical measurable such as an object's position of momentum, we can extract this information from the State vector by means of acting on with an Operator that corresponds to the measurable quantity.\\

To get down to even more specifics if I consider an observable $\hat{O}$, then in general I have the form:
\begin{equation}
\hat{O} \ket{\psi} = o \ket{\psi}
\end{equation}
Where, $o$ is an Eigenvalue . The only types of operators that are constrained in such a fashion are "Hermitian Operators", they are identified with the condition:
\begin{equation}
	content...
\end{equation}
Where\\
If we consider the Schrodinger picture i.e. the State vector evovles with time whereas the Observables are in a loose sense eternal. The time evolution of the state vector is given by the Schrodinger equation:
\begin{equation}
	i \hbar \frac{\partial \ket{\psi}}{\partial t} = \hat{H} \ket{\psi}
\end{equation}
Or,
\begin{equation}
i \hbar \frac{\partial \psi}{\partial t} = \hat{H} \psi
\end{equation}
in terms of the Wavefunction. Where, $\hat{H}$ is the Hamiltonian operator, which can be expressed as:
\begin{equation}
\hat{H} = -\frac{\hbar^{2} \nabla^2}{2m} + V(\vec{x})
\end{equation}
for a free particle. According to Born's rule
\begin{equation}
	\int_{a}^{b} \abs{\psi(\vec{x}, t)}^{2} dx = \text{Probability of finding the particle at a time t between positions a and b}
\end{equation}
Thus, . Physically speaking this lends a kind of indeterminancy to the wavefunction. We can only speak of probabilities. Therefore, we can only , this brings to the measurement hypothesis, that is the State vector evolves to the state corresponding to the measurement being made. And unlike the Schrodinger equation, this evolution is non-deterministic. This tension is often called the "measurement problem", i.e. why is the measurement of an observable a special process distinct from others? Several theories and models claim to have resolved this, but we shall save that discussion for another time. We will fully focus on understanding the theory of Quantum Mechanics in a pragmatic lens before we question its foundations (although the converse isn't necessarily a bad thing, it isn't the purpose of this manuscript).
\section{Normalization}
Normalization is a process through which we ensure that,
\begin{equation}\label{norm}
\int_{- \infty}^{\infty} \abs{\psi(\vec{x}, t)}^{2} dx = 1
\end{equation}
This is a natural consequence of Born's rule, we simply want all the probabilities to add up to 1. Thus, to rule out any other absurd scenarios, we make a ruling that non-Normalizable and non-square integrable Wavefunctions are unphysical.\\
We can also prove that once normalized, the wavefunction always remains normalized, we start by differentiating \ref{norm} with respect to time\\
$$\frac{d}{dt} \int_{- \infty}^{\infty} \abs{\psi(\vec{x}, t)}^{2} dx = \frac{\partial}{\partial t}\int_{- \infty}^{\infty} \abs{\psi(\vec{x}, t)}^{2} dx$$
Dealing with the term inside the integral,
$$ \abs{\psi(\vec{x}, t)}^{2} = $$
\section{Summary of Postulates}
\section{Generalized Uncertainty Principle}

\section{The Uncertainty Principle}

%\part{Toy Models}
%\chapter{Toy Models}
\section{Time-Dependent Schrodinger Equation}
\section{Time-Independent Schrodinger Equation}
\section{Stationary States}
\section{The Infinite Square Well}
\section{Harmonic Oscillator}
\section{Free Particle}
\section{Delta-Function Potential}
\section{Finite Square Well}
\section{Wave-Packets}
%\part{Applications}
%\chapter{Symmetries and their Consequences}
Shankar chapters 11,12 13, 14
,15
\chapter{Addition of Angular Momentum}
\section{The General Problem}
How do we add an arbitray $J_{1}$ and $J_{2}$? If we did what are their $\hat{J}^{2}$ and $\hat{J}_{z}$ eigenvalues like? One way to do this would be to construct square matrices of the dimension $2j +1$. However for this we would first need to know the allowed values of $j$. This we conjecture to be
\begin{equation}
j_{1} \otimes j_{2} = (j_{1} + j_{2}) \oplus (j_{1} + j_{2} - 1) \oplus ... \oplus (j_{1} - j_{2})
\end{equation}
For our example, the total number of kets is:
$$\ket{jm, j_{1}j_{2}}$$
where
$$j_{1} + j_{2} \geq j \leq j_{1} - j_{2}$$
$$j \geq m \leq -j$$
with the normalization conditions:
$$\alpha + \beta = 0$$
$$\alpha^{2} + \beta^{2} = 1$$
\subsection{Clebsch-Gordan Coeeficients}
The completeness of the product kets allows us to write,
\begin{equation}
\ket{jm,j_{1}j_{2}} = \sum_{m_{1}} \sum_{m_{2}} \braket{j_{1}m_{1},j_{2}m_{2}}{jm, j_{1}j_{2}} \ket{j_{1}m_{1},j_{2}m_{2}}
\end{equation}
Here the coffecients of the expansion are termed "Clebsch-Gordan" coefficients. Here are some of their properties:
\begin{itemize}
\item $\braket{j_{1}m_{1},j_{2}m_{2}}{jm}\neq 0$ if and only if $j_{1} - j_{2}\leq j \leq j_{1} + j_{2}$ or $m_{1} + m_{2} = m$
\item $\in \mathbb{R}$
\item $\braket{j_{1}m_{1},j_{2}m_{2}}{jm} > 0$
\item $\braket{j_{1}m_{1},j_{2}m_{2}}{jm} = {(-1)}^{j_{1} + j_{2} - j}\braket{j_{1}(-m_{1}),j_{2}(-m_{2})}{j(-m)}$
\end{itemize}

\subsection{Modified Spectroscopic Notation}
In absence of spin using s,p,d.. is sufficient to describe angular momentum. However, in the presence of spin we change out notation to:
\begin{itemize}
\item Use capital letters S,P,D or L typically to indicate the value of angular momentum
\item Append a subscript $J$ to the right of $L$ i.e. $L_{J}$ to indicate the $j$ value
\item Append a superscript $2S+1$ to the left of $L$ i.e $\prescript{2S+1}{}L$. to indicate the degeneracy due to spin projections
\end{itemize}
%\section{Irreducible Tensor Operators}
%\subsection{Tensor Operators}
%We know that a vector can be linearly expanded in terms of it's basis,
%\begin{equation}
%\ket{V} = \sum^{3}_{i} v_{i}\ket{i}
%\end{equation}
%A second rank tensor similarly can be expressed as,
%%	\ket{T^{(2)}} = \sum^{3}_{i}\sum^{3}_{j} t_{ij}\ket{i}\otimes\ket{j}
%\end{equation}
%\begin{itemize}
%\item 
%\end{itemize}
\chapter{Systems with N degrees of freedom}
\section{Schrodinger Equation in 3 Dimensions}
\section{The Hydrogen Atom}
\section{N Particles in One Dimension}
\section{More Particles in More Dimensions}
\section{Identical Particles}
\chapter{Approximations}
\section{The Variational Principle}
\subsection{Theorem}
Suppose you want to find the ground state energy $E_g$ for a system described by the Hamiltonian $H$, but you are unable to solve the time-dependent Schrodinger equation. Pick any normalised function $\psi$ whatsoever, then,
\begin{equation}
	E_g\le\langle\psi|H|\psi\rangle\equiv\langle H\rangle
\end{equation}
The expectation value of $H$ in the tate $\psi$ is certain to overestimate the ground-state energy. If $\psi$ is an excited state, then it is obvious that the energy is greater than the ground state energy, but the theorem states that the same holds for any $\psi$ whatsoever.

\subsection{Proof}
Since the eigenfunctions of $H$ form a complete set, we can express $\psi$ as a linear combination of them,
\begin{equation*}
	\psi=\sum_{n}^{}c_n\psi_n, \; \text{with} \, H\psi_n=E_n\psi_n
\end{equation*}

Since $\psi$ is normalized,
\begin{equation*}
	1=\braket{\psi}{\psi}=\langle \sum_{m}^{}c_m\psi_m|\sum_{n}^{}c_n\psi_n\rangle=\sum_{m}^{}\sum_{n}^{}c^*_mc_n\langle\psi_m|\psi_n\rangle=\sum_{n}^{}|c_n|^2
\end{equation*}
This is under the assumption that the eigenfuctions have been orthornormalized. Meanwhile,
\begin{equation*}
	\langle H\rangle=\langle \sum_{m}^{}c_m\psi_m|H\sum_{n}^{}c_n\psi_n\rangle=\sum_{m}^{}\sum_{n}^{}c^*_mE_nc_n\langle\psi_m|\psi_n\rangle=\sum_{n}^{}E_n|c_n|^2
\end{equation*}
By defintion, ground state energy is the smallest eigenvalue, so $E_g\le E_n$, and hence,
\begin{equation*}
	\langle H\rangle\ge E_g\sum_{n}^{}|c_n|^2=E_g
\end{equation*}

\subsection{The Ground State of Helium}
The helium atom consists of two electrons in orbit around a nucleus containing two protons, and neutrons, but these are irrelevant as we are only considering charge here. The Hamiltonian for this system, ignoring fine structure and smaller corrections is,
\begin{equation}
	H=-\frac{\hbar^2}{2m}(\nabla^2_1+\nabla^2_2)-\frac{e^2}{4\pi\epsilon_0}\left(\frac{2}{r_1}+\frac{2}{r_2}-\frac{1}{|\textbf{r}_1-\textbf{r}_2|}\right)
\end{equation}

The ground state energy has been experimentally found out to be,
\begin{equation*}
	E_g=-78.975eV
\end{equation*}
We are trying to reproduce this value theoretically. This problem as no exact solution due to the electron-electron repulsion,
\begin{equation}
	V_{ee}=\frac{e^2}{4\pi\epsilon_0}\frac{1}{|\textbf{r}_1-\textbf{r}_2|}
\end{equation}
Ignoring this term, $H$ splits into two independent hydrogen Hamiltonians, and the exact solution is just the product of hydrogenic wave functions,
\begin{equation}
	\psi_0(\textbf{r}_1,\textbf{r}_2)\equiv\psi_{100}(\textbf{r}_1)\psi_{100}(\textbf{r}_2)=\frac{8}{\pi a^3}e^{-2(r_1+r_)/a}
\end{equation}
and the energy is $8E_!=-109 eV$. We can see that this isnt accurate.\\
To get a better approximation for $E_g$, we will apply the variational principle using $\psi_0$ as the trial wave function.
\begin{equation}
	H\psi_0=(8E_1+V_{ee})\psi_0
\end{equation}
Thus,
\begin{equation}
	\langle H\rangle=8e_1+\langle V_{ee}\rangle
\end{equation}
Where,
\begin{equation}
	\langle V_{ee}\rangle=\left(\frac{e^2}{4\pi\epsilon_0}\right)\left(\frac{8}{\pi a^3} \right)^2\int_{}^{}\frac{e^{-2(r_1+r_)/a}}{|\textbf{r}_1-\textbf{r}_2|}d^3\textbf{r}_1d^3\bm{r}_2
\end{equation}
\newpage
The choice of coordinates is as given below,
\begin{center}
	\includegraphics[scale=0.5]{coordinate}
\end{center}
Integral $\bm{r}_2$ is done first. From law of cosines,
\begin{equation}
	|\textbf{r}_1-\textbf{r}_2|=\sqrt{r_1^2+r_2^2-2r_1r_2cos\theta_2}
\end{equation}
And hence,
\begin{equation}
	I_2\equiv\frac{e^{-4r_2/a}}{|\textbf{r}_1-\textbf{r}_2|}=\int\frac{e^{-4r_2/a}}{\sqrt{r_1^2+r_2^2-2r_1r_2cos\theta_2}}r_2^2sin\theta_2dr_2\theta_2d\phi_2
\end{equation}
The $\phi_2$ integral equates to $2\pi$, the $\theta_2$ integral is,

\begin{equation*}
	\int_{0}^{\pi}\frac{sin\theta_2}{\sqrt{r_1^2+r_2^2-2r_1r_2cos\theta_2}}d\theta_2=\frac{\sqrt{r_1^2+r_2^2-2r_1r_2cos\theta_2}}{r_1r_2}\Big\rvert^\pi_0
\end{equation*}

\begin{equation}
	=\frac{1}{r_1r_2}[(r_1+r_2)-(r_1-r_2)]= 
	\begin{cases}
		2/r_1, & \text{if}\ r_2<r_1 \\
		2/r_2, & \text{if}\ r_2>r_1
	\end{cases}
\end{equation}

Thus,
\begin{equation*}
	I_2=4\pi\left(\frac{1}{r_1}\int_{0}^{r_1}e_{-4r_2/a}r_2^2dr_2+\int_{r_1}^{\infty}e^{-4r_2/a}r_2dr_2\right)
\end{equation*}
\begin{equation}
	=\frac{\pi a^3}{8r_1}\left[1-\left(1+\frac{2r_1}{a}e^{-4r_1/a}\right)\right]
\end{equation}

It then follows that $\langle V_{ee}\rangle$ is equal to,
\begin{equation*}
	\left(\frac{e^2}{4\pi\epsilon_0}\right)\left(\frac{8}{\pi a^3} \right)\int \frac{\pi a^3}{8r_1}\left[1-\left(1+\frac{2r_1}{a}e^{-4r_1/a}\right)\right]e^{-4r_1/a}r_1sin\theta_1 dr_1 d\theta_1 d\phi_1
\end{equation*}

The angular integrals are easy ($4\pi$) and the $r_1$ integral becomes,
\begin{equation*}
	\int_{0}^{\infty}\left[re^{-4r/a}-\left(r+\frac{2r^2}{a}\right)e^{-8r/a}\right]dr=\frac{5a^2}{128}
\end{equation*}

Finally,
\begin{equation}
	\langle V_{ee}\rangle=\frac{5}{4a}\left(\frac{e^2}{4\pi\epsilon_0}\right)=-\frac{5}{2}E_1=34 eV
\end{equation}
And therefore,
\begin{equation}
	\langle H \rangle=-109 eV+34eV=-75eV
\end{equation}

This is fairly accurate, but we can improve the result by thinking of a better trial wavefunction. Taking that on average, each electron represents a cloud of negative charge that partially shields the nucleus, so that the other electon sees an effective nucler charge that is somewhat less than 2.\\
This suggests that we use a trial function of the form,
\begin{equation}
	\psi_1(\bm{r}_1\bm{r}_2)\equiv\frac{Z^3}{\pi a^3}e^{-Z(r_1+r_2)/a}
\end{equation}
We vary $Z$ as a variational parameter, picking the value that minimizes $\langle H\rangle$.\\
The wavefunction is an eigenstate of the unperturbed hamiltonian, but with $Z$, instead of 2 in the couloumb terms. We rewrite $H$,
\begin{multline}
	H=\frac{\hbar^2}{2m}(\nabla^2_1+\nabla^2_2)-\frac{e^2}{4\pi\epsilon_0}\left(\frac{Z}{r_1}+\frac{Z}{r_2}\right)\\
	+\frac{e^2}{4\pi\epsilon_0}\left(\frac{(Z-2)}{r_1}+\frac{(Z-2)}{r_2}+\frac{1}{|\bm{r}_1-\bm{r_2}|}\right)
\end{multline}

The expectation value of $H$ is evidently,
\begin{equation}
	\langle H\rangle=2Z^2E_1+2(Z-2)\left(\frac{e^2}{4\pi\epsilon_0}\right)\langle\frac{1}{r}\rangle+\left\langle V_{ee}\right\rangle
\end{equation}
Here the $\langle<1/r\rangle$ is the expectation value of $1/r$ n the hydrogenic ground state $\psi_{100}$, 
\begin{equation}
	\left\langle \frac{1}{r}\right\rangle=\frac{Z}{a}
\end{equation}

Now the expectation value of $V_{ee}$ is,
\begin{equation}
	\langle V_{ee}\rangle =\frac{5Z}{8a}\left(\frac{e^2}{4\pi\epsilon_0}\right)=-\frac{5Z}{4}E_1
\end{equation}

Putting everything together,
\begin{equation}
	\langle H \rangle=[2Z^2-4Z(Z-2)-(5/4)Z]E_1=[-2Z^2+(27/4)Z]E_1
\end{equation}
According to the variational principle, this quantity exceeds $E_g$ for any value of $Z$. The lowest upper bound occurs when $\langle H \rangle$ is minimised,
\begin{equation}
	\frac{d}{dZ}\langle H \rangle=[-4Z+(27/4)]E_1=0
\end{equation}
from which it follows that,
\begin{equation}
	Z=\frac{27}{16}=1.69
\end{equation}
This says that the other electron partially screens the nucleus, reducing its effective charge from 2 to 1.69. Substituting this in Z, we find,
\begin{equation}
	\langle H \rangle=\frac{1}{2}\left(\frac{3}{2}^6\right)E_1=-77.5eV
\end{equation}

Hence we see that the ground state of helium has been calculated with great precision. We can improve this by selecting better trial wavefunctions, but we do not require that since we have already reached within 2\% of the correct answer.
\subsection{The Hydrogen Molecule Ion}
Another classic application of the variational principle is to the hydrogen molecule ion, H , consisting of a
single electron in the Coulomb field of two protons . Assuming for the moment that the
protons are fixed in position, a specified distance R apart, although one of the most interesting byproducts of
the calculation is going to be the actual value of R. The Hamiltonian is,
\begin{equation}
	H=\frac{\hbar^2}{2m}\nabla^2-\frac{e^2}{4\pi\epsilon_0}\left(\frac{1}{r}+\frac{1}{r'}\right)
\end{equation}

Where $r$ and $r'$ are the distances to the electron from the respective protons. Next we try to guess a reasonable trial wavefunction, by taking a hydrogen atom in its ground state,
\begin{equation}
	\psi_0(r)=\frac{1}{\sqrt{\pi a^3}}e^{-r/a}
\end{equation}

bringing the second proton in from “infinity,” and nailing it down a distance R away. If R is substantially
greater than the Bohr radius, the electron’s wave function probably isn’t changed very much. But we would
like to treat the two protons on an equal footing, so that the electron has the same probability of being
associated with either one. This suggests that we consider a trial function of the form,
\begin{equation}
	\psi=A[\psi_0(r)+\psi_0(r')]
\end{equation}
Normalising this wave function,
\begin{equation}
	1=\int|\psi|^2d^3r=|A|^2\left[\psi_0(r)^2d^3\bm{r}+\int\psi_0(r')^2+2\int\psi_0(r)\psi_0(r')d^3r\right]
\end{equation}
Evaluating the integrals give the normalisation factor to be,
\begin{equation}
	|A|^2=\frac{1}{2(1+I)}
\end{equation}
Where $I$ is,
\begin{equation}
	I=e^{-R/a}\left[1+\frac{R}{a}+\frac{1}{3}\left(\frac{R}{a}\right)^2\right]
\end{equation}

Now computing the expectation value of the Hamiltonian in the trial state $\psi$,
\begin{equation}
	\langle H\rangle=\left[1+2\frac{(D+X)}{1+I}\right]E_1
\end{equation}
According to the variational principle,
\begin{equation}
	V_{pp}=\frac{e^2}{4\pi\epsilon_0}\frac{1}{R}=-\frac{2a}{R}E_1
\end{equation}
Thus the total energy of the system, in the units of $-E_1$, and expressed as a function of $x\equiv R/\alpha$, is less than,
\begin{equation}
	F(x)=-1+\frac{2}{x}\left\{\frac{(1-(2/3)x^2)e^{-x}+(1+x)e^{-2x}}{1+(1+x+(1/3)x^2)e^{-x}}\right\}
\end{equation}

Plotting this function,
\begin{center}
	\includegraphics[scale=0.4]{variational}
\end{center}

Evidently bonding does occur, for there exists a region in which the graph goes below -1 , indicating that the energy is less than that of a neutral atom plus a free proton.

\newpage


\section{The WKB Approximation}
\subsection{Introduction}
The WKB (Wentzel, Kramers, Brillouin)1 method is a technique for obtaining approximate solutions to the
time-independent Schrödinger equation in one dimension (the same basic idea can be applied to many other
differential equations, and to the radial part of the Schrödinger equation in three dimensions). It is particularly
useful in calculating bound state energies and tunneling rates through potential barriers.
Imagine a particle of energy E moving through a region where the potential $V(x)$ is a constant. if $E>V$, the wavefunction is of the form,
\begin{equation*}
	\psi(x)=Ae^{\pm ikx}, \; \text{with}\; k\equiv\sqrt{2m(E-V)/\hbar}
\end{equation*}

The plus sign indicates that the particle is traveling to the right, and the minus sign means it is going to the
left (the general solution, of course, is a linear combination of the two). The wave function is oscillatory, with
fixed wavelength and unchanging amplitude . Now suppose that the potential is not constant, but
varies rather slowly in comparison to $\lambda$ , so that over a region containing many full wavelengths the potential is
essentially constant. Then it is reasonable to suppose that $\psi$ remains practically sinusoidal, except that the wavelength and the amplitude change slowly with x. This is the inspiration behind the WKB approximation.
In effect, it identifies two different levels of x-dependence: rapid oscillations, modulated by gradual variation in amplitude and wavelength.\\
If $E<V$,
\begin{equation*}
	\psi(x)=Ae^{\pm kx}, \; \text{with}\; k\equiv\sqrt{2m(V-E)/\hbar}
\end{equation*}

If $V(x)$ is not constant, but varies slowly in comparison with $1/k$, the solution remains practically exponential. The approximation fails at the classical turning point, where $E\approx V$, where $\lambda$ goes to infinity.

\subsection{The Classical Region}
The Schrodinger equation,
\begin{equation*}
	-\frac{\hbar^2}{2m}\frac{d^2\psi}{dx^2}+V(x)\psi=E\psi
\end{equation*}

can be rewritten as,
\begin{equation}
	\frac{d^2\psi}{dx^2}=-\frac{p^2}{\hbar^2}\psi
\end{equation}
where,
\begin{equation}
	p(x)\equiv\sqrt{2m[E-V(x)]}
\end{equation}
is the classical formula for momentum of a particle with total energy $E$ and potential energy $V(x)$. We assume that $E>V$, and this is the classical region and the particle is confined to this range of x. Writing the complex function $\psi$ in terms of amplitude and phase,
\begin{equation}
	\psi(x)=A(x)e^{i\phi(x)}
\end{equation}

Using a prime to denote the derivative with respect to $x$, we find,
\begin{equation*}
	\frac{d\psi}{dx}=(A'+iA\phi')e^{i\phi}
\end{equation*}

And,
\begin{equation}
	\frac{d^2\psi}{dx^2}=[A''+2iA'\phi'+iA\phi''-A(\phi')^2]
\end{equation}
Putting this in equation (32)
\begin{equation}
	\frac{d^2\psi}{dx^2}=[A''+2iA'\phi'+iA\phi''-A(\phi')^2]= -\frac{p^2}{\hbar^2}A
\end{equation}

We can split this into two parts, the real and imaginary,
\begin{equation}
	A''-A(\phi')^2=\frac{-p^2}{\hbar^2}A, \; \text{or} \; A''=A\left[(\phi')^2-\frac{p^2}{\hbar^2}\right]
\end{equation}
And,
\begin{equation}
	2A'\phi'+ A\phi''=0, \; \text{or} \; (A^2\phi')'=0
\end{equation}

From this we get that,
\begin{equation}
	\phi(x)=\pm\frac{1}{\hbar}\int p(x)dx
\end{equation}
And then,
\begin{equation}
	\psi(x)\approxeq\frac{C}{\sqrt{p(x)}}e^{\pm \frac{1}{\hbar}\int p(x)dx}
\end{equation}
And that the general approximate solution will be a linear combination of two of these terms, one with each sign. See that,
\begin{equation}
	|\psi(x)|^2\approxeq\frac{|C|^2}{p(x)}
\end{equation}
Which says that the probability of finding a particle at point x is inversely proportional to the classical momentum at that point. Sometimes, the WKB approximation is derived from this semiclassical observation.

\subsection{Tunneling}
Now assuming that $E<V$, $p(x)$ becomes imaginary,
\begin{equation}
	\psi(x)\approxeq\frac{C}{\sqrt{|p(x)|}}e^{\pm \frac{1}{\hbar}\int |p(x)|dx}
\end{equation}
Take a problem of scattering from a rectangular barrier with a bumpy top, to the left of the barrier,
\begin{equation}
	\psi(x)=Ae^{ikx}+Be^{-ikx}
\end{equation}
To the right of the barrier,
\begin{equation}
	\psi{x}=Fe^{ikx}
\end{equation}
Where F is the transmitted amplitude and the tunneling probability is,
\begin{equation}
	T=\frac{|F|^2}{|A|^2}
\end{equation}
Our WKB approximation gives us,
\begin{equation}
	\psi(x)\approxeq \frac{C}{\sqrt{|p(x)|}}e^{\pm \frac{1}{\hbar}\int_{0}^{x} |p(x')|dx'} +\frac{D}{\sqrt{|p(x)|}}e^{\pm -\frac{1}{\hbar}\int_{0}^{x} |p(x')|dx'}
\end{equation}

If the barrier is sufficiently wide, the exponentially increasing term diminishes, and the relative amplitudes look like,
\begin{equation*}
	\frac{|F|}{|A|}\approx e^{\pm -\frac{1}{\hbar}\int_{0}^{\alpha} |p(x')|dx'}
\end{equation*}

So that,
\begin{gather}
	T\approxeq e^{-2\gamma}\\
	\gamma\equiv \frac{1}{\hbar}\int_{0}^{\alpha} |p(x')|dx'
\end{gather}


\subsection{The connection formulas}
We have obtained the wavefunctions for before and after the walls of the potential well, now to determine the region within the potential well, or at a turning point ($E=V$), shifting the axes so that the right hand turning point occurs at $x=0$, in the WKB approximation we have,
\begin{equation}
	\psi(x)\approxeq
	\begin{cases}
		\frac{1}{\sqrt{|p(x)|}}\left[Be^{ \frac{1}{\hbar}\int_{x}^{0} |p(x')|dx'} +Ce^{\pm -\frac{1}{\hbar}\int_{x}^{0} |p(x')|dx'}\right], \; \text{if}\; x<0 \\
		
		\frac{1}{\sqrt{|p(x)|}}De^{-\frac{1}{\hbar}\int_{0}^{x}|p(x')|dx'} \; \text{if} \; x>0
	\end{cases}
\end{equation}
We essentially patch the two WKB solutions, by using the connection formulas,
\begin{equation}
	B=-ie^{i\pi/4}D, \; \text{and}\; C=ie^{-i\pi/4}D
\end{equation}

Now providing the final wavefunction,
\begin{equation}
	\psi(x)\approxeq
	\begin{cases}
		\frac{-2D}{\sqrt{|p(x)|}}sin\left[ \frac{1}{\hbar}\int_{x}^{x^2} |p(x')|dx'+\frac{\pi}{4}\right],\; \text{if}\; x<x_2\\
		
		\frac{D}{\sqrt{|p(x)|}}e^{-\frac{1}{\hbar}\int_{x_2}^{x}|p(x')|dx'} \; \text{if} \; x>x_2
	\end{cases}
\end{equation}
Where $x_2$ is an arbitrary point.

\section{The Adiabatic Approximation}
\subsection{Adiabatic processes}
A gradual change in the external conditions characterises an adiabatic process. There are two characteristic times involved, $T_i$, the internal time, which governs the motion of the system itself and $T_e$, the external time regarding the change in external parameters.\\

Suppose that the Hamiltonian changes gradually from some initial form $H^i$ to some final form $H^f$, the adiabatic theorem states that if the particle was initially in the nth eigenstate of $H^i$, it will be carried under the Schrodinger equation into the nth eigenstate of $H^f$.\\

Take the example of a particle prepared in the ground state of an infinite square well,
\begin{equation}
	\psi^i(x)=\sqrt{\frac{2}{a}}\sin\left(\frac{\pi}{a}x\right)
\end{equation}

If we now gradually move the right wall out to 2a. the adiabatic theorem says that the particle will end up in the ground state of the expanded well,
\begin{equation}
	\psi^f(x)=\sqrt{\frac{1}{a}}\sin\left(\frac{\pi}{2a}x\right)
\end{equation}

Note that this change in Hamiltonian isnt small, like perturbation theory, the change can be as huge as needed, but the change should happen slowly is what the adiabatic approximation requires.

\subsection{Berry's Phase}
Now we discuss how the Adiabatic approximation is used in nonholonomic processes, where a system does not return to its original state when transported around a closed loop.\\
If the Hamiton is independent of time, then a particle which starts out in the $n$th eigenstate $\psi_n(x)$,
\begin{equation}
	H\psi_n(x)=E_n\psi_n(x)
\end{equation}
remains in the $n^{th}$ eignstate, simply picking up a phase factor,
\begin{equation}
	\Psi_n(x,t)=\psi_n(x)e^{-iE_nt/\hbar}
\end{equation}
But the adiabatic theorem states that when H changes gradually, the particle remains in the $n^th$ eignstate, even as the eigenfunction itself evolves,
\begin{equation}
	\Psi_n(x,t)=\psi_n(x,t)e^{-\frac{1}{\hbar}\int_{0}^{t}E_n(t')dt'}e^{i\gamma_n(t)}
\end{equation}
Where the term,
\begin{equation}
	\theta_n(t)\equiv-\frac{1}{\hbar}\int_{0}^{t}E_n(t')dt'
\end{equation}
Is the dynamic phase and the extra phase, $\gamma_n(t)$ is the geometric phase, which has its significance in the adiabatic theorem.\\
The net geometric phase change is usually given by,
\begin{equation}
	\gamma_n(t)=i\oint \langle\psi_n|\nabla_R\psi_n\rangle \cdot dR
\end{equation}

This is a line integral around a losed loop in parameter space, and its not zero. This is called Berry's phase, and it depends only on the path taken, not on how fast the path is traversed, meanwhile the dynamic phase depends critically on elapsed time
\chapter{Perturbation Theory}
\section{Non-Degenerate Perturbation Theory}
\subsection{General Formulation}
Suppose we have solved the time-independent Schrodinger wave equation for a given potential (in this case, an infinite potential square well)
\begin{equation}
H^0\psi_{n}^0=E^{0}_{n}\psi_{n}^0
\end{equation}
and obtaining a complete set of orthonormal eigenfunctions $\psi_{n}^0$,
\begin{equation}
\bra{\psi_{n}^0}{\psi_{m}^0}\rangle=\delta_{nm}
\end{equation}
and the corresponding eigenvalues $E^0_n$. If we perturb the potential slightly in the potential well and try to solve for the new eigenvalues and eigenfunctions, 
\begin{equation}
H\psi_n=E_n\psi_n
\end{equation}
Here, we use perturbation theory to get approximate solutions to the perturbed problem by building on the exact solutions of the unperturbed case.\\
To begin with, we write the perturbed/new Hamiltonian as the sum of two terms,
\begin{equation}
H=H^0+\lambda H'
\end{equation}
Where H' is the perturbation. We take $\lambda$ to be a small number, and the $H$ will be the true, exact Hamiltonian. Writing $\psi_n$ and $E_n$ as a power series in $\lambda$, we get,
\begin{gather}
\psi_{n}=\psi_{n}^0 + \lambda \psi_{n}^1 + \lambda^2 \psi_{n}^2+... \\
E_n= E^0_n + \lambda E^1_n + \lambda^2 E^2_n+...
\end{gather}
Here $E^1_n$ is the first-order correction to the $n^th$ eigenvalue, and $\psi_{n}^1$ is the first-order correction to the $n^th$ eigenfunction. $E^2_n$ and $\psi_{n}^2$ are the second-order corrections to the eigenvalues and eigenfunctions, and so on. Plugging in Equations (4),(5) and (6) in Equation (3) gives us,
\begin{multline}
(H^0+\lambda H')[\psi_{n}^0 + \lambda \psi_{n}^1 + \lambda^2 \psi_{n}^2+... ]=\\( E^0_n + \lambda E^1_n + \lambda^2 E^2_n+...)[\psi_{n}^0 + \lambda \psi_{n}^1 + \lambda^2 \psi_{n}^2+...]
\end{multline}
We can rewrite Equation (7) by collecting like powers of $\lambda$ in the form,
\begin{multline*}
H^0\psi^0 + \lambda(H^0\psi^1_n+H'\psi_n^0) + \lambda^2(H^0\psi^2_n+H'\psi^1_n)+...\\E^0_n\psi^0 + \lambda(E^0_n\psi^1_n+E^1_n\psi_n^0) + \lambda^2(E^0_n\psi^2_n+E_n^1\psi^1_n+E^2_n\psi^0_n)+...
\end{multline*} 
We can get the first order ($\lambda^1$) equation from Equation (7),
\begin{equation}
H^0\psi_{n}^1+H'\psi^0_n=E^0_n\psi^1_n+E^0_n+\psi_n^0
\end{equation}
And the second order ($\lambda^2$),
\begin{equation}
H^0\psi_{n}^2+H'\psi^1_n=E^0_n\psi^2_n+E^1_n\psi^1_n+E^2_n\psi^0_n
\end{equation}
And this can be done for higher powers of $\lambda$ as well.

\subsection{First order perturbation theory}
If we take the inner product of Equation (8), with $\psi^0_n$,
\begin{equation}
\langle\psi^0_n\vert H^0\psi^1_n\rangle + \langle\psi^0_n\vert H'\psi^0_n\rangle=E^0_n\langle\psi^0_n\vert\psi^1_n\rangle +E^1_0\langle\psi^0_n\vert\psi^0_n\rangle
\end{equation}
Because of the useful property of $H^0$ to be Hermitian, hence Equation (10) becomes,
\begin{equation}
\langle\psi^0_n\vert H^0\psi^1_n\rangle=\langle H^0\psi^0_n\vert \psi^1_n\rangle=E^0_n\langle\psi^0_n\vert\psi^1_n\rangle=\langle E^0_n\psi^0_n\vert\psi^1_n\rangle
\end{equation}
And hence the terms in Equation (10) cancel out and the property $\langle\psi^0_n\vert\psi^0_n\rangle=1$ give the equation,
\begin{equation}
E^1_n=\langle\psi_n^0\vert H'\vert\psi^0_n\rangle
\end{equation}
This is a fundamental result in first-order perturbation theory, and it states that first-order correction to energy is the expectation value of the pertubation in the unperturbed state.

Now to get the first-order correction to the wave function, we rewrite Equation (8),
\begin{equation}
(H^0-E^0_n)\psi^1_n=-(H'E^1_n)\psi^0_n
\end{equation}
The right side is a known fucntion, so this amounts to an inhomogenious differential equation for $\psi^!_n$. The unpertubed wave functions constitute a complete set, so $\psi^1_n$ can be writtten as a linear combination of them,
\begin{equation}
\psi^1_n=\sum_{m\neq n}c^{(n)}_m\psi^0_m
\end{equation}
We know that $\psi^0_m$ satisfies the unpertubed Schrodinger wave equation, so we have,
\begin{equation}
\sum_{m\neq n}^{}(E^0_m-E^0_n)c^{(n)}_m\psi^0_m=-(H'-E^1_n)\psi^0_n
\end{equation}

Taking the inner product with $\psi^0_l$,
\begin{equation}
\sum_{m\neg n}(E^0_m-E^0_n)c^{(n)}_m\braket{\psi^0_l}{\psi^0_m}=-\langle\psi^0_l\vert H' \vert \psi^0_n\rangle+E^1_n\braket{\psi^0_l}{\psi^0_n}
\end{equation}
If $l=n$, is zero, we then get,

\begin{equation}
E^0_m-E^0_n)c^{(n)}_l=-\langle\psi^0_l\vert H' \vert \psi^0_n\rangle
\end{equation}
Or that,
\begin{equation}
c^{(n)}_n=\frac{\langle\psi^0_m\vert H' \vert \psi^0_n\rangle}{E^0_n-E^0_m}
\end{equation}
So,
\begin{equation}
\psi^1_n=\sum_{m\neg n}\frac{\langle\psi^0_m\vert H' \vert \psi^0_n\rangle}{E^0_n-E^0_m}\psi^0_m
\end{equation}
Note that the perturbed energies are surprisingly accurate, while the wave functions are of poor accuracy.


\subsection{Second order perturbation theory}
We take the inner poduct of the second-order equation with $\psi^0_n$,
\begin{equation}
\braket{\psi^0_n}{H^0\psi^2_n}+\braket{\psi^0_n}{H'\psi^1_n}=E^0_n\braket{\psi^0_n}{\psi^2_n}+E^1_n\braket{\psi^0_n}{\psi^1_n}+ E^2_n\braket{\psi^0_n}{\psi^0_n}
\end{equation}
We exploit the Hermiticity of $H^0$,
\begin{equation}
\braket{\psi^0_n}{H^0\psi^2_n}=\braket{H^0\psi^2_n}{\psi^2_n}=E^0_n\braket{\psi^0_n}{psi^2_n}
\end{equation}
So the first term on the left cancels the first term on the right. Hence we get the formula for $E^2_n$ to be,
\begin{equation}
E^2_n=\langle\psi^0_n\vert H' \vert \psi^1_n\rangle - E^1_n\braket{\psi^0_n}{\psi^1_n}
\end{equation}
But,

\begin{equation}
\braket{\psi^0_n}{\psi^1_n}=\sum_{m\neq n}^{}c^{(n)}_m\braket{\psi^0_n}{\psi^0_m}=0
\end{equation}

so,
\begin{equation}
E^2_n=\langle\psi^0_n\vert H' \vert \psi^1_n\rangle=\sum_{m\neq n}^{}c^{(n)}_m\braket{\psi^0_n}{\psi^0_m}= \sum_{m\neq n}^{}c^{(n)}_m\frac{\langle\psi^0_m\vert H' \vert \psi^0_n\rangle \langle\psi^0_m\vert H' \vert \psi^0_n\rangle}{E^0_n-E^0_m}
\end{equation}
Therefore,
\begin{equation}
E^2_n= \sum_{m\neq n}^{}c^{(n)}_m\frac{\vert\langle\psi^0_m\vert H' \vert \psi^0_n\rangle\vert^2}{E^0_n-E^0_m}
\end{equation}
This is the fundamental result of second order perturbation theory.

\section{Degenerate Perturbation Theory}
\subsection{Motivation}
If two or more distinct states, take $\psi^0_a$ and $\psi^0_b$ share the same energy, ordinary perturbation theory fails since Equation (25) blows up. So hence we need to obtain a different way to handle the problem.

\subsection{Twofold Degeneracy}
Suppose,
\begin{align*}
H^0\psi^0_a=E^0\psi^0_a\\
H^0\psi^0_b=E^0\psi^0_b\\
\braket{\psi^0_a}{\psi^0_b=0}
\end{align*}
And note that any of linear combinations of these states,
\begin{equation}
\psi^0=\alpha\psi^0_a+\beta\psi^0_b
\end{equation}
is still an eigenstate of $H^0$, with the same eigenvalue $E^0$,
\begin{equation}
H^0\psi^0=E^0\psi^0
\end{equation}	 
When $H$ is perturbed, it breaks the degeneracy. When we increase $\lambda$, the common unperturbed energy $E^0$ splits into two. When we take away the perturbation, the upper state redyces to one linear combination of $\psi^0_a$ and $\psi^0_b$, and the lower state reduces to some other linear combination. We need to figure out the good linear combinations.\\
Now writing the good unperturbed states in general form, keeping $\alpha$ and $\beta$ adjustable and solving the Schrodinger equation,
\begin{equation}
H\psi=E\psi
\end{equation}
With $H=H^0+\lambda H'$ and,
\begin{align}
E=E^0+\lambda E^1 + \lambda^2 E^2 +...\\
\psi=\psi^0+\lambda\psi^1+\lambda^2\psi^2+...
\end{align} 
Plugging these into Equation (28) and collecting like powers of $\lambda$, as before, we find,
\begin{equation}
H^0\psi^0+\lambda(H\psi^0+H^0\psi^1)+...=E^0\psi^0+\lambda(E^1\psi^0+E^0\psi^1)+...
\end{equation}
But $H^0\psi^0=E^0\psi^0$, so the first term cancel; at order $\lambda^1$ we have,
\begin{equation}
H\psi^0+H^0\psi^1=E^1\psi^0+E^0\psi^1
\end{equation}
Taking inner product with $\psi^0_a$,
\begin{equation}
\langle\psi^0_a\vert H^0 \vert \psi^1\rangle+ \langle\psi^0_a\vert H' \vert \psi^0\rangle=E^0\braket{\psi^0_a}{\psi^1} + E^1\braket{\psi^0_a}{\psi^0}
\end{equation}
Because $H^0$ is Hermitian, the first term on the left cancels the term on the right. Putting this in Equation (26), we get,
\begin{equation}
\alpha \langle\psi^0_a\vert H' \vert \psi^0_a\rangle=\beta \langle\psi^0_a\vert H' \vert \psi^0_b\rangle=\alpha E^1
\end{equation}
Or in a more compact form,
\begin{equation}
\alpha W_{aa}+\beta W_{ab}=\alpha E^1
\end{equation}
Where,
\begin{equation*}
W_{ab}= langle\psi^0_a\vert H' \vert \psi^0_b\rangle
\end{equation*}
Similarly, the inner product with $\psi^0_b$ gives us,
\begin{equation}
\alpha W_{ba}+\beta W_{bb}=\beta E^1
\end{equation}
Now using Equation (35) and (36),
\begin{equation}
\alpha[W_{ab}W_{ba}-(E^1-W_{aa}))(E^1-W_{bb})]=0
\end{equation}
When $\alpha\neq 0$,
\begin{equation}
(E^1)^2-E^1(W_{aa}+W{bb})+(W_{aa}W_{bb}-W_{ab}W_{ba})=0
\end{equation}
Using the quadratic formula and knowing that $W_{ba}=W^*_{ab}$,
\begin{equation}
E^1_\pm=\frac{1}{2}\left[W_{aa}+W_bb\pm\sqrt{(W_{aa}-W_{bb})^2+4\vert W_{ab}\vert^2}\right]
\end{equation}
This is the fundamental result of degenerate pertubration theory, the two roots correspond to the two perturbed energies.\\
Note that when $\alpha=0$, we get the nondegenerate perturbation theory (since $\beta=1$).

\subsection{Higher-Order Degeneracy}
We start by rewriting Equations (35) and(36) in matrix form,
\begin{gather}
\begin{pmatrix}
W_{aa} & W_{ab}\\
W_{ba} & W_{bb}
\end{pmatrix}
\begin{pmatrix}
\alpha\\
\beta
\end{pmatrix}
= E^1
\begin{pmatrix}
\alpha\\
\beta
\end{pmatrix}
\end{gather}
The $E^1$s are the eigenvalues of the $W$-matrix, Equation (38) being the characteristic equation for this matrix and the good linear combinations of the unperturbed states are the eigenvectors of $W$. For $n$-fold degeneracy, we look for the eigenvalues of the $n\cross n$ matrix,
\begin{equation}
W_{ij}=\langle \psi^0_i\vert H' \vert \psi^0_j\rangle
\end{equation}
\subsection{Lamb Shift}
An interesting feature of the fine structure formula is that it depends only on $j$ and not $l$, moreover in general two different values of $l$ share the same energy. For example, the $2S_{1/2} ()$ and $2P_{1/2} ()$ states should remain perfectly degenerate. However in 1947 Lamb and Retherford performed an experiment that displayed something to the contrary. The $S$ state is slightly higher in energy than the $p$ state. The explanation of this "Lamb" shift was later explained by Bethe, Feynman, Schwinger and Tomonaga (the founders of QED) as a corollary of the electromagnetic field itself being quantised.  Sharply in contrast to the hyperfine structure of Hydrogen, the Lamb shift is a completely novel i.e. non-classical (as the hyperfine structure is explained through Coulomb's law and the Biot-Savart Law) phenomena. It arises from a radiative correction in Quantum Electrodynamics to which classical theories are mute. In Feynman lingo, this arises from loop corrections as potrayed below.
\begin{figure}[h]
	\centering
	\includegraphics[width=0.6\linewidth, height=0.3\linewidth]{feyn-loop.png}
	\caption{Different kinds of radiative corrections}
\end{figure}
Naively,
\begin{enumerate}
	\item the first diagram describes pair-production in the neighbhorhood of a nucleus, leading to a partial screening effect of the proton's charge;
	\item the second diagram reflects the fact that the electromagnetic field has a non-zero ground state
	\item the third diagram leads to a tiny modification of the electron's magnetic dipole moment (an addition of $a + \alpha/2\pi = 1.00116$)
\end{enumerate}
We shall not discuss the results in depth but rather consider two exemplary cases:\\
For $ l = 0$,
\begin{equation}
\Delta E_{Lamb} = \alpha^{5}mc^{2}\frac{1}{4n^{3}}\left[k(n,0)\right]
\end{equation}
Where $k(n,0)$ is a numerical factor defined as:
$$k(n,0) = \begin{cases}
12.7, & \text{if } n = 1\\
13.2,              & \text{if } n \rightarrow \infty
\end{cases} $$
For $ l = 0$ and $j = l \pm \frac{1}{2}$,
\begin{equation}
\Delta E_{Lamb} = \alpha^{5}mc^{2}\frac{1}{4n^{3}}\left[ k(n,0) \pm \frac{1}{\pi (l + \frac{1}{2}) (l + \frac{1}{2})} \right]
\end{equation}
Here, $k(n,l)$ is a very small number $(< 0.05)$ which varies a little with it's arguments.\\
The Lamb shift is tiny except for the case $l=0$, for which it amounts to about $10 \% $ of the fine structure. However, since it depends on $l$, it lifts the degeneracy of the pairs of states with common $n$ and $j$ and in particular it splits $2 S_{1/2}$ and $2 P_{1/2}$.
\section{The Zeeman Effect}
When an atom is placed in a uniform magnetic field $B_{Ext.}$, the energy levels are shifted, this is known as the Zeeman effect. For the case of a single electron, the shift is:
\begin{equation}\label{zeeman_def}
H^{'}_{Z} = -(\mu_{l} + \mu_{s}).B_{Ext.}
\end{equation}
Where,
\begin{equation}
\mu_{s} = -\frac{e}{m_{e}}S
\end{equation}
is the magnetic dipole moment associated with electron spin, and
\begin{equation}
\mu_{l} = -\frac{e}{2m_{e}}L
\end{equation}
is the dipole moment associated with orbital motion. The gyromagnetic ratio in this case is simply classical i.e. $q/2m$, it is only for spin that we have an extra factor of 2. We now rewrite (\ref{zeeman_def}) as:
\begin{equation}
H^{'}_{Z} = \frac{e}{2m_{e}}(L + 2S).B_{Ext.}
\end{equation}
The nature of the Zeeman splitting depnds on the strength of the external field vs. the internal one that gives rise to spin-orbit/spin-spin coupling. This table provides a short review of the different cases:
\begin{center}
	\begin{tabularx}{0.9\textwidth} { 
			| >{\centering\arraybackslash}X 
			| >{\centering\arraybackslash}X 
			| >{\centering\arraybackslash}X | }
		\hline
		\textbf{Case} & \textbf{Name} & \textbf{Comments} \\
		\hline
		$B_{Ext.} >> B_{Int.}$  & Strong-Field Zeeman Effect  & Zeeman effect dominates; fine structure becomes the perturbation  \\
		\hline
		$B_{Ext.} << B_{Int.}$  & Weak-Field Zeeman Effect  & Fine structure dominates; $H^{'}_{z}$ can be treated as a small perturbation   \\
		\hline
		$B_{Ext.} = B_{Int.}$  & Intermediate Zeeman Effect  & Both the fields are equal in strength thus we would need elements of degenerate peturbation theory and will need to diagonlize the necessary portion of the Hamiltonian "by hand" \\
		\hline
	\end{tabularx}
\end{center}
In the next few sections we'll explore all of them in depth.
\subsection{Weak-Field Zeeman Effect}
Here the fine structure dominates, thus the conserved quantum numbers are $n$, $l$, $j$ and $m_{j}$, but not $m_{l}$ and $m_{s}$ due to the spin-orbit coupling L and S are not separately conserved. Generally speaking, in this problem we have a perturbation pile on top of a perturbation. Thus, the conserved quantum number are those appropriate to the dominant . In first order perturbation theory, the Zeeman correction to energy is,
\begin{equation}
E^{1}_{Z} = \expval{H^{'}_{Z}}{nlj m_{j}} = \frac{e}{2m}B_{Ext.} \expval{L + 2S}
\end{equation} 
Now to figure out $\expval{L + 2S}$, we know that $L + 2S = J + S$, this doesn't immediately tell us the expectation value of $S$ but we can figure it out as by understanding that $J = L + S$ is conserved and that the time average of $S$ is simply it's projection along $J$:
\begin{equation}
S_{Ave} = \frac{(S.J)}{J}J
\end{equation}
But, $L = J - S$, so  $L^{2} = J^{2} + S^{2} - 2 J.S$, hence:
\begin{equation}
S.J = \frac{1}{2}(J^{2} + S^{2} - 2 J.S) = \frac{\hbar^{2}}{2}[j(j+1)+ s(s+1)-l(l+1)]
\end{equation}
from which it follows that,
\begin{equation}
\expval{L + 2S} = \expval{\left(1 + \frac{S.J}{J^{2}}J\right)} = \left[1 + \frac{j(j+1)-l(l+1) + 3/4}{2j(j+1)}\right]\expval{J}
\end{equation}
The term in the square brackets is called the Lande g-factor, denoted by $g_{j}$. Now, if we choose $B_{z}$ to lie along $B_{Ext.}$, then:
\begin{equation}
E^{1}_{Z} = \mu_{B} g_{j} B_{Ext.} m_{j}
\end{equation}
where,
$$\mu_{B} = \frac{e \hbar}{2m} = 5.788 \times 10^{-5} \ eVT^{-1}$$
is the so called Bohr magneton. The total energy is the sum of the fine-structure part and the Zeeman contribution, in the ground state i.e. $n = 1, l = 0, j = 1/2$ and therefore, $g_{J} = 2$, it splits into two levels:
\begin{equation}
-13.6 \ eV(1 + \alpha^{2}/4) \pm \mu_{B} B_{Ext.}
\end{equation}
with different signs for different $m_{j}$'s this is plotted below.
\begin{figure}[h]
	\centering
	\includegraphics[width=0.6\linewidth, height=0.5\linewidth]{strong-split.png}
	\caption{Weak-field Zeeman splitting of the ground state of hydrogen; the upper line has a slope of $1$ and the lower line a slope of $-1$}
\end{figure}
\begin{figure}[h]
	\centering
	\includegraphics[width=0.6\linewidth, height=0.5\linewidth]{weak-split.png}
	\caption{In the presence of spin-orbit coupling, $L$ and $S$ are not separately conserved, they precess about the fixed total angular momentum $J$}
\end{figure}
\subsection{Strong-Field Zeeman Effect}
In this case, the Zeeman effect is often referred to as the "Paschen-Back" effect. The conserved quantum numbers are now but and because in the presence of an external torque, the total angular momentum is not conserved but the it's individual components are. The Zeeman Hamiltonian is,
\begin{equation}
H^{'}_{Z} =  \frac{e}{2m} B_{Ext.} (L_{z} + 2 S_{z})
\end{equation}
and the unperturbed energies are:
\begin{equation}
E_{nm_{l}m_{s}} = -\frac{13.6 \ eV}{n^{2}} + \mu_{B}B_{Ext.}(m_{l} + 2 m_{s})
\end{equation}
This would be our result if we ignore the fine structure completely. However, we need to take that into account as well. In first-order perturbation theory, the fine structure correction to these levels is:
\begin{equation}
E^{1}_{fs} = \expval{H^{'}_{r} + H^{'}_{so}}{n \ l \ m_{l} \ m_{s}}
\end{equation}
The relativistic contribution is the same as before for the spin-orbit term, we need
\begin{equation}
\expval{S.L} = \expval{S_{x}}\expval{L_{x}} + \expval{S_{y}}\expval{L_{y}} + \expval{S_{z}}\expval{L_{z}} = \hbar^{2}m_{l}m_{s}
\end{equation}
Here $\expval{S_{x}} = \expval{S_{y}} = \expval{L_{x}} = \expval{L_{y}} = 0$ for the eigenstates of $S_{z}$ and $L_{z}$. Putting it all together:
\begin{equation}
E^{1}_{fs} = \frac{13.6 \ eV}{n^{3}}\alpha^{2} \left(\frac{3}{4n} - \left[ \frac{l(l+1)-m_{l}m_{s}}{l(l+1/2)(l+1)}\right]\right)
\end{equation}
Th term in the square brackets is indeterminate for $l=0$, it's correct value in this case is 1. The total energy here is the sum of the Zeeman part and the fine structure contribution.
\subsection{Intermediate Zeeman Effect}
In this case, we must treat both the effects as perturbations to the Bohr Hamiltonian,
\begin{equation}
H^{'} = H^{'}_{Z} + H^{'}_{fs}
\end{equation}
In section we'll discuss the case $n = 2$, and use it as the basis for degerate perturbation theory. The states here are characterized by $l$, $j$ and $m_{j}$. We could use $l$,$m_{l}$,$m_{s}$ states but this makes the matrix elements of $H^{'}_{Z}$ easier to deal with but that of $H^{'}_{fs}$ difficult. Using the Clebsch-Gordan coefficients to express $\ket{j m_{j}}$ as a linear combination of $\ket{l m_{l}} \ket{s m_{s}}$ we have:
$$l = 0 = \begin{cases}
\psi_{1} & \ket{\frac{1}{2} \frac{1}{2}} = \ket{0 \ 0}\ket{\frac{1}{2}\frac{1}{2}}\\
\psi_{2} & \ket{\frac{1}{2} \frac{-1}{2}} = \ket{0 \ 0}\ket{\frac{1}{2}\frac{-1}{2}}
\end{cases} $$
$$l = 1 = \begin{cases}
\psi_{3} & \ket{\frac{3}{2} \frac{3}{2}} = \ket{1 \ 1}\ket{\frac{1}{2}\frac{1}{2}}\\
\psi_{4} & \ket{\frac{3}{2} \frac{-3}{2}} = \ket{1 \ -1}\ket{\frac{1}{2}\frac{-1}{2}}\\
\psi_{5} & \ket{\frac{3}{2} \frac{1}{2}} = \sqrt{2/3}\ket{1 \ 0}\ket{\frac{1}{2}\frac{1}{2}} + \sqrt{1/3}\ket{1 \ 1}\ket{\frac{1}{2}\frac{-1}{2}}\\
\psi_{6} & \ket{\frac{1}{2} \frac{1}{2}} = -\sqrt{1/3}\ket{1 \ 0}\ket{\frac{1}{2}\frac{1}{2}} + \sqrt{2/3}\ket{1 \ 1}\ket{\frac{1}{2}\frac{-1}{2}}\\
\psi_{7} & \ket{\frac{3}{2} \frac{-1}{2}} = \sqrt{1/3}\ket{1 \ -1}\ket{\frac{1}{2}\frac{1}{2}} + \sqrt{2/3}\ket{1 \ 0}\ket{\frac{1}{2}\frac{-1}{2}}\\
\psi_{8} & \ket{\frac{1}{2} \frac{-1}{2}} = -\sqrt{2/3}\ket{1 \ -1}\ket{\frac{1}{2}\frac{1}{2}} + \sqrt{1/3}\ket{1 \ 0}\ket{\frac{1}{2}\frac{-1}{2}}\\
\end{cases} $$
In this basis the matrix the non-zero elements of $H^{'}_{fs}$ are all on the diagonal and are given by the Bohr model. $H^{'}_{z}$ has four off diagonal elements. The complete matrix, W as we will see is more complicated but its eigenvalues are the same since they are independent of the chosen basis.
\begin{equation}
\begin{pmatrix}
5 \gamma - \beta & 0 & 0 & 0 & 0 & 0 & 0 & 0 \\
0 & 5 \gamma + \beta & 0 & 0 & 0 & 0 & 0 & 0\\
0 & 0 & \gamma - 2 \beta & 0 & 0 & 0 & 0 & 0\\
0 & 0 & 0 & \gamma + 2 \beta & 0 & 0 & 0 & 0\\
0 & 0 & 0 & 0 & \gamma - \frac{2}{3} \beta & \frac{\sqrt{2}}{3} \beta & 0 & 0\\
0 & 0 & 0 & 0 & \frac{\sqrt{2}}{3} \beta & 5 \gamma - \frac{1}{3} \beta & 0 & 0\\
0 & 0 & 0 & 0 & 0 & 0 & \gamma + \frac{2}{3} \beta & \frac{\sqrt{2}}{3} \beta\\
0 & 0 & 0 & 0 & 0 & 0 & \frac{\sqrt{2}}{3} \beta & 5 \gamma + \frac{1}{3} \beta
\end{pmatrix}
\end{equation}
Where,
$$\gamma = {(\alpha / 8)}^{2}13.6 \ eV$$
and,
$$\beta = \mu_{B}B_{Ext.}$$
The first four eigenvalues are already displayed along the diagonal. We only need to find the eigenvalues of the two  $2 \times 2$ blocks. The characteristic equation for the first one is given as:
\begin{equation}
\lambda^{2} - \lambda(6\gamma - \beta) + \left(5 \gamma^{2} - \frac{11}{3} \gamma \beta\right) = 0
\end{equation} 
and the quadratic formula gives the eigenvalues:
\begin{equation}
\lambda_{\pm} = 3 \gamma - (\beta /2) \pm \sqrt{4 \gamma^{2} + (2/3) \gamma \beta + (\beta^{2}/4)} 
\end{equation}
The eigenvalues of the second block are the same but with the sign of $\beta$ reversed. The eight energy levels are listed in the table and are plotted against in the figure (). In the zero field limit they reduce to the fine structure values. For the other cases, the splitting is seen clearly.
\begin{figure}[h]
	\centering
	\includegraphics[width=0.6\linewidth, height=0.3\linewidth]{zee-table.png}
	\caption{Energy levels for the $n=2$ states of hydrogen, with fine structure and Zeeman splitting}
\end{figure}
\begin{figure}[h]
	\centering
	\includegraphics[width=0.6\linewidth, height=0.6\linewidth]{intermediate-split.png}
	\caption{Zeeman splitting of the $n = 2$ states of hydrogen, in the weak, intermediate and strong field regimes}
\end{figure}
\section{Hyperfine Splitting in Hydrogen}
The proton also has a magnetic dipole moment, however this is much smaller than that of the electron due to the mass of the proton. It is given by,
\begin{equation}
\mu_{p} = \frac{g_{p} e}{2 m_{p}}S_{p}
\end{equation}
And the magnetic dipole moment of the electron,
\begin{equation}
\mu_{e} = -\frac{e}{m_{e}}S_{e}
\end{equation}
Classically speaking, the dipole $\mu$ sets up a magnetic field:
\begin{equation}
B = \frac{\mu_{0}}{4 \pi r^{3}}[3(\mu . \hat{r})\hat{r} - \mu] + \frac{2 \mu_{0}}{3} \mu \delta^{3}(r)
\end{equation}
So the Hamiltonian of the electron, in the magnetic field due to the proton's magnetics dipole moment, is
\begin{equation}
H^{'}_{hf} = \frac{\mu_{0} g_{p} e^{2}}{8 \pi m_{p} m_{e}}\frac{[3(S_{p}. \hat{r})(S_{e}. \hat{r}) - S_{p}.S_{e}]}{r^{3}} + \frac{\mu_{0} g_{p} e^{2}}{3 m_{p} m_{e}}S_{p}.S_{e} \delta^{3}(r)
\end{equation}
According to perturbation theory, the first-order correcction to the energy is the expectation value of the perturbing Hamiltonian:
\begin{equation}
E^{1}_{hf} = \frac{\mu_{0} g_{p} e^{2}}{8 \pi m_{p} m_{e}} \expval{\frac{3(S_{p}.\hat{r})(S_{e}.\hat{r}) - S_{p}.S_{e}}{r^{3}}} + \frac{\mu_{0} g_{p} e^{2}}{3 m_{p} m_{e}}\expval{S_{p}.S_{e}}\abs{\psi(0)}^{2}
\end{equation}
In the groud state or any other state at which $l = 0$, the wavefunction is spherically symmetrical, and the first expectation value vanishes. Meanwhile, from the Schrodinger equation in three dimensions, we find that $\abs{\psi(0)}^{2} = 1/(\pi a^{3})$, thus,
\begin{equation}
E^{1}_{hf} = \frac{\mu_{0} g_{p} e^{2}}{3 \pi m_{p}m_{e} a^{3}}\expval{S_{p}.S_{e}}
\end{equation}
in the groud state. This is called Spin-Spin coupling because it involves the dot product of two spins in contrast with spin-orbit coupling which involves $S.L$. In the presence of spin-spin coupling, the individual spin angular momenta are no longer conserved. However the eigenvectors of the total spin is conserved:
\begin{equation}
S = S_{e} + S_{p}
\end{equation}
We square this out to get,
\begin{equation}
S_{p}. S_{e} = \frac{1}{2}(S^{2} - S^{2}_{e} - S^{2}_{p})
\end{equation}
But the electron and proton both have spin $1/2$, so $S^{2}_{e} = S^{2}_{p} = (3/4) \hbar^{2}$. In the triplet i.e. parallel spin state, the total spin is $1$, and hence $S^{2} = 2 \hbar^{2}$. In the singlet state the total spin is $0$, and  $S^{2} = 0$. Thus,
\begin{equation}
E^{1}_{hf} = \frac{4 g_{p} \hbar^{4}}{3 m_{p} m^{2}_{e}c^{2}\alpha^{4}} \begin{cases}
+1/4, & \text{ (triplet)};\\
-3/4, & \text{ (singlet)}
\end{cases}
\end{equation}
The Spin-Spin coupling breaks the spin degeneracy of the groud state, lifting the triplet and depressing the singlet, leading to an energy gap. The energy gap is given by:
\begin{equation}
\Delta E = \frac{4 g_{p} \hbar^{4}}{3 m_{p} m^{2}_{e}c^{2}\alpha^{4}} = 5.88 \times 10^{-6} \ eV
\end{equation}
\begin{figure}[h]
	\centering
	\includegraphics[width=0.6\linewidth, height=0.3\linewidth]{hyp-split.png}
	\caption{Hyperfine splitting in the ground state of Hydrogen}
\end{figure}
The frequency of the photon emitted when the triplet transitions to a singlet state is:
\begin{equation}
\nu = \frac{\Delta E}{h} = 1420 \text{ MHz}
\end{equation}
The corresponding wavelength is $21 $ cm which falls in the microwave region. It permeates the universe and is a very important part of Astrophysics.
\section{Introduction to quantum dynamics}
So far, we looked at systems that were time-independent of sorts (quantum statics), and the potentials themselves were time independent, in other words, $V(r,t)=V(r)$. Hence the time-dependent Schrodinger equation took the form,
\begin{equation}
	H\psi=i\hbar\frac{\partial \psi}{\partial t}
\end{equation}
And solving by separation of variables,
\begin{equation}
	\psi(r,t)=\psi(r)e^{iEt/\hbar}
\end{equation}
where $\psi(r)$ satisfies the time independent Schrodinger equation,
\begin{equation}
	H\psi=E\psi
\end{equation}
Due to the nature of the term that carries time dependence, the exponential factor $e^{iEt/\hbar}$, this term cancels out when we construct the physically relevant quantity $|\psi|^2$, which leads to all the expectation values and probabilities to be constant in time, and this is the same case in more complex states where we have linear combinations of these stationary states.\\
For us, if we want to investigate transitions between one energy level to another, we introduce a time-dependent potential, hence the name of quantum dynamics arises.

\section{Two level systems}
\subsection{Introduction}
Let us suppose that there are two states of the unpertubed sustem, $\psi_a$ and $\psi_b$. They are eigenstates of the unpertubed Hamiltonian, $H_0$,
\begin{align}
	H_0\psi_a=E_a  \psi_a \; \text{and} \; H_0\psi_b=E_b\psi_b
\end{align}
and they are orthonormal,
\begin{equation}
	\langle\psi_a|\psi_b\rangle=\delta_{ab}
\end{equation}
And any state can be expressed as a linear combination of them, or in particular,
\begin{equation}
	\Psi(0)=c_a\psi_a+c_b\psi_b
\end{equation}
In the absence of perturbation, each component evolves with its characteristic exponential factor.
\begin{equation}
	\Psi(t)=c_a\psi_ae^{-iE_at/\hbar}+c_b\psi_be^{-iE_bt/\hbar}
\end{equation}
Calculating $|c_a|^2$ is th probability that the particle is in he state $\psi_a$, and the measurement of the energy will yield $E_a$. Normalizing $\Psi$,
\begin{equation}
	|c_a|^2+|c_b|^2=1
\end{equation}

\subsection{The perturbed system}
Turning on a time-dependent perturbation $H'(t)$, the coefficients $c_a$ and $c_b$ become functions of $t$ and the equation then becomes,
\begin{equation}
	\Psi(t)=c_a(t)\psi_ae^{-iE_at/\hbar}+c_b(t)\psi_be^{-iE_bt/\hbar}
\end{equation} 
Now we solve for $c_a(t)$ and $c_b(t)$ by using the time-dependent Schrodinger equation,
\begin{equation}
	H\Psi=i\hbar\frac{\partial\Psi}{\partial t}, \: \text{where} \: H=H_0+H'(t)
\end{equation}

Then we see that,
\begin{equation}
	\begin{split}
		c_a[H_0\psi_a]_e^{-iE_at/\hbar}+c_b[H_0\psi_b]e^{-iE_bt/\hbar}+c_a[H'\psi_a]_e^{-iE_at/\hbar} \\
		+c_b[H'\psi_b]e^{-iE_bt/\hbar}=i\hbar[ \dot{c_a}\psi_a e^{-iE_at/\hbar}		\dot{c_b}\psi_b e^{-iE_bt/\hbar}\\
		+c_a\psi_a\left(\frac{iE_a}{\hbar}\right)  e^{-iE_at/\hbar}+ c_b\psi_b\left(\frac{iE_b}{\hbar}\right)  e^{-iE_bt/\hbar}]
	\end{split}
\end{equation}
This then simplifies to,
\begin{equation}
	c_a[H'\psi_a]_e^{-iE_at/\hbar}+c_b[H'\psi_b]e^{-iE_bt/\hbar}=i\hbar[ \dot{c_a}\psi_a e^{-iE_at/\hbar}		\dot{c_b}\psi_b e^{-iE_bt/\hbar}
\end{equation}
We isolate $\dot{c_a}$ by taking the inner product with $\psi_a$ and exploiting the orthogonality of $\psi_a$ and $\psi_b$.
\begin{equation}
	c_a\langle \psi_a| H'|\psi_a\rangle_e^{-iE_at/\hbar}+c_b\langle \psi_a|H'|\psi_b]e^{-iE_bt/\hbar}=i\hbar\dot{c_a}e^{-iE_at/\hbar}
\end{equation}
Then we define,
\begin{equation}
	H'_{ij}\equiv\langle\psi_i|H'|\psi_j\rangle
\end{equation}
The hermiticity of $H'$ says that $H'_{ji}=(H'_ij)^*$. Now multiplying with $-(i/\hbar)e^{iE_at/\hbar}$, we conclude that,
\begin{equation}
	\dot{c_a}=-\frac{i}{\hbar}[c_aH'_{aa}+c_bH'_{ab}e^{-i(En-E_a)t/\hbar}]
\end{equation}
Similarly the inner product with $\psi_b$ isolate $\dot{c_b}$ and gives the result,

\begin{equation}
	\dot{c_b}=\frac{i}{\hbar}[c_bH'_{bb}+c_aH'_{ba }e^{-i(En-E_a)t/\hbar}]
\end{equation}
Equations (15) and (16) are equivalent to the time-dependent Schrodinger equation for a two level system. And the diagonal matrix elements of $H'$ vanish giving,
\begin{equation}
	H'_{aa}=H'_{bb}=0
\end{equation}
And the equations simplify to,
\begin{align}
	\dot{c_a}=-\frac{i}{\hbar}H'_{ab}e^{-i\omega_0t}c_b \\
	\dot{c_b}=-\frac{i}{\hbar}H'_{ab}e^{i\omega_0t}c_a
\end{align}
Where,
\begin{equation*}
	\omega_0=\frac{E_b-E_a}{\hbar}
\end{equation*}

\subsection{Time-Dependent Perturbation Theory}
Defining a size for the perturbation $H'$, considering it to be small, we can obtain solutions for equations (18) and (19) by the process of successive approximations.\\
Suppoe the particle starts out in the lower state,
\begin{equation}
	c_a(0)=1. \; c_b(0)=0
\end{equation}
If there exists no perturbation, these states remain iike this forever.\\

\textbf{Zeroth Order:} \\
\begin{equation}
	c^{(0)}_a(t)=1 \; c^{(0)}_b(t)=0
\end{equation}
To calculate the first-order approximation, we insert these values on the right side of equations (18) and (19)\\
\\
\textbf{First Order:} \\
\begin{equation*}
	\frac{dc_a}{dt}=0 \rightarrow c^{(1)}_a(t)=1;
\end{equation*}	
\begin{equation}
	\frac{dc_b}{dt}=-\frac{i}{\hbar}H'_{ba}e^{i\omega_0t}\rightarrow c^{(1)}_b=\frac{i}{\hbar}\int_{0}^{t}H'_{ba}(t')e^{i\omega_0t'}dt'
\end{equation}
\\
\textbf{Second Order:}\\
\begin{equation*}
	\frac{dc_a}{dt}=-\frac{i}{\hbar}H'_{ba}e^{i\omega_0t}\left(\frac{i}{\hbar}\right)\int_{0}^{t}H'_{ba}(t')e^{i\omega_0t'}dt'\rightarrow
\end{equation*}
\begin{equation}
	c^{(2)}_a(t)=1-\frac{1}{\hbar}\int_{0}^{t}H'_{ab}(t')e^{i\omega_0t'}\left[\int_{0}^{t'}H'_{ba}(t'')e^{i\omega_0t''}dt''\right]dt'
\end{equation}
We can continue this process for obtaining n-order approximations.

\subsection{Sinusoidal Perturbations}
If the perturbation has sinusoida time dependence,
\begin{equation}
	H'(r,t)=V(r)cos(\omega t)
\end{equation}
so that,
\begin{equation}
	H'_{ab}=V_{ab}cos(\omega t)
\end{equation}
Where,
\begin{equation}
	V_{ab)}\equiv \langle\psi_a|V|\psi_b\rangle
\end{equation}
For the first order perturbations, using equation (22),
\begin{equation}
	c_b(t)\approx -\frac{i}{\hbar}\int_{0}^{t}cos(\omega t')dt'=-\frac{iV_ba}{2\hbar}\int_{0}^{t}\left[e^{i(\omega_0+\omega)t'}+e^{i(\omega_0-\omega)t'}\right]dt'
\end{equation}
\begin{equation}
	=-\frac{V_{ba}}{2\hbar}\left[\frac{e^{i(\omega_0+\omega)t}-1}{\omega_0+\omega}	+ \frac{e^{i(\omega_0-\omega)t}-1}{\omega_0-\omega}	\right]
\end{equation}
Simplifying this equation by restricting our attention to driving frequency $\omega$ close to transition frequence $\omega_0$, the second term dominates. To be more specific, we assume,
\begin{equation}
	\omega_0+\omega>>|\omega_0-\omega|
\end{equation}
Perturbation at other frequencies have negligible probabilty for causing a transistion, so this isnt a limitation. Now the equation simplifies to,
\begin{equation*}
	c_b(t)\approx-\frac{V_{ba}}{2\hbar} \frac{e^{i(\omega_0-\omega)t}-1}{\omega_0-\omega}[e^{i(\omega_0-\omega)t/2}-e^{-i(\omega_0-\omega)t/2}]
\end{equation*}
\begin{equation}
	=-i\frac{V_{ba}}{\hbar}\frac{sin[(\omega_0-\omega) t/2]}{\omega_0-\omega}e^{i(\omega_0-\omega)t/2}
\end{equation}

The transition probability, the probability that a particle which started out in the state $ \psi_a $ wil be found at a time $t$, in the state $\psi_b$ is,
\begin{equation}
	P_{a\rightarrow b}(t)=|c_b(t)^2\approx \frac{|V_{ba}|^2}{\hbar^2}\frac{sin^2[(\omega_0-\omega) t/2]}{(\omega_0-\omega)^2}
\end{equation}	

\section{Emission and Absorption of Radiation}
\subsection{Electromagnetic Waves}
An atom, in the presence of a passing light wave, responds to the electric component. It is sinusoidal in nature,
\begin{equation}
	E= E_0cos(\omega t)\hat{k}
\end{equation}
Where $q$ is the charge of the electron. Evidently,
\begin{equation}
	H'_{ba}=-\wp E_0cos(\omega t), \; \wp\equiv q\langle\psi_b|z|\psi_a\rangle
\end{equation}
When $\psi$ is an even or odd function of $z$, $z|\psi|^2$ is odd and integrates to zero, hence the diagonal matrix elements of $H'$ vanish, hence the perturbation is oscillatory and is of the form,
\begin{equation}
	V_{ba}=-\wp E_0
\end{equation}

\subsection{Absorption, stimulated emission and spontaneous emission}
If an atom starts in the lower state $\psi_a$ and you shine a polarized moochromatic beam of light on it, the probablilty of a transistion to the "upper" state $\psi_b$ is given by,
\begin{equation}
	P_{a\rightarrow b}(t)=\left(\frac{\wp E_0}{\hbar}\right)^2\frac{sin^2[(\omega_0-\omega) t/2]}{(\omega_0-\omega)^2}
\end{equation}
The atom absorbs an energy of $E_B-E_a=\hbar\omega_0$ from the electromagnetic field. We could say that it has absorbed a photon.\\
Now doing the same derivation for the transition from lower to upper state, we get
\begin{equation}
	P_{b\rightarrow a}(t)=\left(\frac{\wp E_0}{\hbar}\right)^2\frac{sin^2[(\omega_0-\omega) t/2]}{(\omega_0-\omega)^2}
\end{equation}
Here we note that the probability of transition from $a\rightarrow b$ is the same as $b\rightarrow a$, and this was called stimulated emission. The electromagnetic field gains energy $\hbar\omega_0$ from the atom.\\
Spontaneous emission is when an atom in the excited state makes a transition downward with a release of a photon without the application of any electromagnetic field.

\subsection{Incoherent perturbations}
The energy density of an electromagnetic wave is,
\begin{equation}
	u=\frac{\epsilon_0}{2}E^2_0
\end{equation}	
The transition probablity is proportional to the energy density of the fields,
\begin{equation}
	P_{b\rightarrow a}(t)=\frac{2u}{\epsilon_0\hbar^2}|\wp|^2\frac{sin^2[(\omega_0-\omega) t/2]}{(\omega_0-\omega)^2}
\end{equation}
But this is for a monochromatic perturbation for a single frequency $\omega$, now subjecting the system to a range of frequencies, $u\rightarrow\rho(\omega)d\omega$ where $\rho(\omega)d\omega$ is the enegy density in the frequency range $d\omega$ and the net transistion probability takes the form of an integral,
\begin{equation}
	P_{b\rightarrow a}(t)=\frac{2}{\epsilon_0\hbar^2}|\wp|^2 \int_{0}^{\infty}\rho(\omega)\left\{\frac{sin^2[(\omega_0-\omega) t/2]}{(\omega_0-\omega)^2}\right\}d\omega
\end{equation}	
The term in the curly brakets is sharply peaked about $\omega_0$. while $\rho(\omega)$ is relatively broad, so we replace $\rho(\omega)$ with $\rho(\omega_0)$ anf tak it outside the integral,
\begin{equation}
	P_{b\rightarrow a}(t)\approx\frac{2|\wp|^2}{\epsilon_0\hbar^2}\rho(\omega_0)\ \int_{0}^{\infty}\left\{\frac{sin^2[(\omega_0-\omega) t/2]}{(\omega_0-\omega)^2}\right\}d\omega
\end{equation}
Changing variables from $x\equiv(\omega_0-\omega)t/2$, extending the limits of integration to $x=+_-\infty$, looking up the definite integral,
\begin{equation*}
	\int_{-\infty}^{\infty}\frac{sin^2x}{x^2}dx=\pi
\end{equation*}
we find that,
\begin{equation}
	P_{b\rightarrow a}(t)\approx\frac{\pi|\wp|^2}{\epsilon_0\hbar^2}\rho(\omega_0)t
\end{equation}	
The transition probability is proportional to $t$. The flopping phenomenon characteristic of a monochromatic perturbation gets washed out when we hit the system with an incoherent spread of frequencies. The transition rate 
is now a constant,
\begin{equation}
	R_{b\rightarrow a}=\frac{\pi}{\epsilon_0\hbar}|\wp|^2\rho(\omega_0)
\end{equation}	
We take all possible polarisations and not only from x and z directions, hence instead of $|\wp|^2$, we have an average $|\hat{n}\cdot\bm{\wp}|^2$, where
\begin{equation}
	\bm{\wp}=q\langle\psi_b|r|\psi_a\rangle
\end{equation}	
And the average is over both polarisations and over all incident directions. This averaging is carried out as follows,\\
\\
\textbf{Polarisation:} For propogation in the z-direction, the two possible polarisations are $\hat{i}$ and $\hat{j}$, so the polarisation average ($p$) is,
\begin{equation}
	(\hat{n}\cdot\bm{\wp})^2_p=\frac{1}{2}[	(\hat{i}\cdot\bm{\wp})^2+ 	(\hat{j}\cdot\bm{\wp}^2)=\frac{1}{2}\wp^2 sin^2\theta 
\end{equation}
Where $\theta$ is the angle between $\bm{\wp}$ and the direction of propogation.\\
\\
\textbf{Propogation direction}:	Setting the polar axis along $\bm{\wp}$ and integrate over all propogation directions to get the polarisation-propogation average,

\begin{equation}
	(\hat{n}\cdot\bm{\wp})^2_pp=\frac{1}{4\pi}\int\left[\frac{1}{2}\wp^2 \sin^2\theta\right]sin\theta d\theta d\phi=\frac{\wp^2}{4}\int_{0}^{\pi}\sin^3\theta d\theta=\frac{\wp^3}{3} 
\end{equation}
So the transition rate for the stimulated emission from state $b$ to state $a$, under the influence of incoherent, unpolarized light incident from all directions, is,
\begin{equation}
	R_{b\rightarrow a}=\frac{\pi}{3\epsilon_0\hbar^2}|\bm{\wp}|^2\rho(\omega_0)
\end{equation}
Where \bm{$\wp$} is the matrix element of the electric dipole moment between the two tate and $\rho(\omega_0)$ is the energy density in the fields, per unit frequency, evaluated at $\omega_0=(E_b-E_a)/\hbar$.

\section{Spontaneous Emission}
\subsection{Einstein's A and B coefficients}
Picture a container of atoms, $N_a$ of them in a lower state $\psi_a$ and $N_b$ of them in the upper state $\psi_b$. Let A be the spontaneous emission rate, so that the number of particle leaving the upper state by this process per unit time is $N_bA$. The transition rate for stimulated emission is proportional to the energy density of the electromagnetic field. The number of upper state particles leaving is also similarly found. The absorption rate is proportional to $\rho(\omega_0)$. Now,
\begin{equation}
	\frac{dN_b}{dt}=-N_bA-N_bB_{ba}\rho(\omega_0)+N_aB_{ab}\rho(\omega_0)
\end{equation}
Supposing that the atoms are in thermal equlibrium with the ambient field so that the number of particles in each level is constant. $dN_b/dt=0$, hence,
\begin{equation}
	\rho(\omega_0)=\frac{A}{(N_a/N_b)B_{ab}-B_{ba}}
\end{equation}

From statistical mechanics, the number of particles with energy $E$. in thermal equilibrium at temperature $T$, is proportional to the Boltzmann factor, $e^{-E/k_bT}$, so
\begin{equation}
	\frac{N_a}{N_b}=\frac{e^{-E_a/k_bT}}{e^{E_b/k_bT}}=e^{\hbar\omega_0/k_bT}
\end{equation}
and hence,
\begin{equation}
	\rho(\omega_0)=\frac{A}{e^{\hbar\omega_0/k_bT}}B_{ab}-B_{ba}
\end{equation}
From Planck's blackbody formula,
\begin{equation}
	\rho(\omega)=\frac{\hbar}{\pi^2c^3}\frac{\omega^3}{e^{\hbar\omega/k_bT}-1}
\end{equation}
And from these two equations we can deduce that,
\begin{equation}
	B_{ab}=B_{ba}
\end{equation}
And so,
\begin{equation}
	A=\frac{\omega^3\hbar}{\pi^2c^3}B_{ba}
\end{equation}
We see that from equation (52), that the transition rate for stimulated emission is the same as for absorption. Now, we deduce from equation (46) that,
\begin{equation}
	B_{ba}= frac{\pi}{3\epsilon_0\hbar^2}|\bm{\wp}|^2
\end{equation}
And spontaneous emission rate is,
\begin{equation}
	A=\frac{\omega^3|\bm{\wp}|^2}{3\pi\epsilon_0\hbar c^3}
\end{equation}

\subsection{The lifetime of an excited state}
Suppose that you have a bottle full of atoms, with $B_b(t)$ of them in the excited stat. Due to spontaneous emission, this number decreases as time goes on, and to be more specifi, in a time interval $dt$ you will lose a fraction of $Adt$ of them,
\begin{equation}
	dN_b=-AN_bdt
\end{equation}	
Solving for $N_b(t)$, we get,
\begin{equation}
	N_B(t)=N_b(0)e^{-At}
\end{equation}
Here we can see that the number remaining in the excited state decreases exponentially, with a time constant,
\begin{equation}
	\tau=\frac{1}{A}
\end{equation}
We call this the lifetime of a state, or to be more technical, it is the time taken for $N_B(t)$ to reach $1/e\approx0.368$ of its initial value.\\
Suppose that there are many more states, and many more decays, the transition rates add up, and the net lifetime is,
\begin{equation}
	\tau=\frac{1}{A_1+A_2+A_3+...}
\end{equation}

\subsection{Selection rules}
The calculation of spontaneous emission rates are reduced to a matter of evaluating matrix elements of the form,
\begin{equation*}
	\langle\psi_b|\textbf{r}|\psi_a\rangle
\end{equation*}
Specifying the states with quantum numbers $n$, $l$ and $m$,	
\begin{equation}
	\langle b'l'm'|\textbf{r}|nlm\rangle
\end{equation}
Exploitatio of the angular momentum commutation relations and the hermiticity of the angular momentum operators yield these constraints on this quantity,\\
\\
\textbf{Selection rules involving $m$ and $m'$:} No transitions occur unless $\Delta m=+_-1$ or $0$\\
\\
\textbf{Selection rules involving $l$ and $l'$:} No transitions occur unless $\Delta l=+_-1$











\chapter{Scattering}
\section{Classical Scattering}
\subsection{Motivation}
We can also define the size/radius of the proton is through its rate of interacting with 	itself or other particles.
This is done by us determining the cross-sectional area. The larger this area is, the more likely it is that you will interact with it. The smaller the area, the less likely to interact.
This motivates a connection between proton size and scattering probability. In particle physics, a collision or interaction rate is expressed in effective cross-sectional area, typically just called cross section.
As an	“area,” we can measure scattering cross sections as the square of some relevant 	length scale.

\subsection{The problem}
Consider a particle incident on some scattering center. It comes in with an energy \textrm{\textit{\textbf{E}}} and an impact parameter \textrm{\textit{\textbf{b}}}, and it emerges at some scattering angle \textrm{\textit{$\theta$}}.
The essential problem of classical scattering theory is this: \textit{Given the impact parameter, calculate the scattering angle.} 
Ordinarily, of course, the smaller the impact parameter, the greater the scattering angle. 
\begin{center}
	\includegraphics[scale=1.2]{ClassicalScattering}
\end{center}
\newpage

\subsection{Solving for differential cross-section}
Particles incident within an infinitesimal patch of cross-sectional area d$\sigma$ will scatter into a corresponding infinitesimal solid angle dQ. 
The larger d$\sigma$ is, the bigger dQ will be; the proportionality factor, D($\theta$) = d$\sigma$/dQ, is called the differential (scattering) cross-section, and is given by,

\begin{equation}
	d\sigma=D(\theta)d\Omega
\end{equation}
In terms of the impact parameter and the azimuthal angle $\phi, d\sigma = b.db.d\phi$ and $d\Omega=\sin(\theta)d\theta d\phi$, and so,

\begin{equation}
	D(\theta)=\frac{b}{\sin\theta}\abs{\frac{db}{d\theta}}
\end{equation}
And then the total cross-section is the integral of D($\theta$) over all solid angles,

\begin{equation}
	\sigma=\int D(\theta)d\Omega
\end{equation}

The differential cross-section is the total area of incident beam that is scattered by the target. Beams incident within this area will hit the target, and those farther out will miss it completely.\\
If we have a beam of incident particles, with uniform intensity/luminosity ($\mathcal{L}$), the number of particles entering area d$\sigma$ (and hence scattering into solid angle d$\Omega$), per unit time, is dN = $\mathcal{L}$d$\sigma$= $\mathcal{L}$D($\theta$)d$\Omega$, then,

\begin{equation}
	D(\theta)=\frac{1}{\mathcal{L}}\frac{dN}{d\Omega}
\end{equation}

This is the definition of the differential cross-section.

\begin{center}
	\includegraphics[scale=0.8]{DiffScattering}
\end{center}
\newpage


\section{Quantum Scattering}
\subsection{Defining the problem}
In the quantum theory of scattering, we imagine an incident plane wave,$\psi(z)=Ae^{ikz}$ , traveling in the z
direction, which encounters a scattering potential, producing an outgoing spherical wave That is, we look for solutions to the Schrödinger equation of the generic form,

\begin{equation}
	\psi(r,\theta)\approx A\left\{e^{ikz}+f(\theta)\frac{e^{ikr}}{r}\right\} \textrm{,	for large $r$}
\end{equation}

The relation between wave number $k$ and energy of incident particles are,
\begin{equation}
	k\equiv \frac{\sqrt{2mE}}{\hbar}
\end{equation}
\subsection{Determining scattering amplitude}
Probability of the incident particle travelling with speed $v$ passing through infinitesimal area $d\sigma$ in time $dt$ is,
\begin{equation}
	dP=\vert \psi_{incident}\vert^2dV=\vert A \vert^2(v dt)d\sigma
\end{equation}
This is equal to the probability that the particle later emerges into the corresponding solid angle $d\Omega$,
\begin{equation}
	dP=\vert \psi_{scattered}\vert^2dV=\frac{\vert A \vert^2\vert f \vert^2}{r^2}(v dt)r^2d\sigma
\end{equation}
And $d\sigma$=$\vert f\vert^2d\Omega$, so,
\begin{equation}
	D(\theta)=\frac{d\sigma}{d\Omega}=\vert f(\theta)\vert^2
\end{equation}
The differential cross-section (which is the quantity of interest to the experimentalist) is equal to the	absolute square of the scattering amplitude. Now we look at different methods to determine this scattering amplitude.

\newpage

\section{Partial Wave Analysis}
\subsection{Formalism}
We know that the Schrodinger equation for a spherically symmetrical potential $V(r)$ admits the separable solutions,
\begin{equation}
	\psi(r,\theta,\phi)=R(r)^m_l(\theta,\phi)
\end{equation}
Where $Y^m_l$ is a spherical harmonic and $u(R)=rR(r)$ satisfies the radial equation,
\begin{equation}
	-\frac{\hbar^2}{2m}\frac{dl^2 u}{d r^2} + \left[V(r) + \frac{\hbar^2}{2m}\frac{l(l+1)}{r^2}\right]u = Eu
\end{equation}
At very large $r$, the potential goes to zero, and the centrifugal term is negligible, so,
\begin{equation}
	\frac{d^2u}{dr^2}\approx-k^2u
\end{equation}
Whose general solution is takes the form,
\begin{equation}
	u(r)=Ce^{ikr}+De^{-ikr}
\end{equation}
The first term represents an outgoing spherical wave, and the second an incoming one. For the scattered wave, we want $D=0$. At very large r, then.
\begin{equation}
	R(r)\approx\frac{e^{ikr}}{r}
\end{equation}
The radial equation then becomes,
\begin{equation}
	\frac{d^2}{dr^2}-\frac{l(l+1)}{r^2}u=-k^2u
\end{equation}
The general solution for the radial equation is a linear combinations of spherical Bessel functions,
\begin{equation}
	u(r)=Arj_l(kr)+Brn_l(kr)
\end{equation}		
We need solutions that are linear combinations analogous to $e^ikr$ and $e^-ikr$, these are called the spherical Hankel functions,
\begin{equation}
	h^{(1)}_l\equiv j_l(x)+in_l(x)
\end{equation}
\newpage
Below we see some examples of Hankel functions,

\begin{center}
	\includegraphics[scale=0.8]{Hankels}
\end{center}
The Hankel function of the first kind becomes $e^-ikr/r$ for large r, so we use these to get,
\begin{equation}
	R(r)=Cj^{(1)}_l(kr)
\end{equation}
\subsection{Exact wavefunction and the partial wave amplitude}
The exact wave function in the exterior region where $V(r)=0$ is,
\begin{equation}
	\psi(r,\theta,\phi)=A\left\{e^{ikz}+f(\theta,\phi)\frac{e^{ikr}}{r}\right\}
\end{equation}
Where,
\begin{equation}
	f(\theta,\phi)+\frac{1}{k}\sum_{l,m}^{}(-i)^{l+1}C_{l,m}Y^m_l(\theta,\phi)
\end{equation}
The $C_{l,m}$ are called the partial wave amplitudes. Now the cross section is,
\begin{equation}
	D(\theta,\phi)= \vert f(\theta,\phi)\vert^2= \frac{1}{k^2} \sum_{l,m}\sum_{l',m'}(i)^{l-l'}C^*_{l,m}C_{l'm'}(Y^m_l)^*Y^{m'}_{l'}
\end{equation}
And the total cross-section is,
\begin{equation}
	\sigma= \frac{1}{k^2} \sum_{l,m}\sum_{l',m'}(i)^{l-l'}C^*_{l,m}C_{l'm'}\int(Y^m_l)^*Y^{m'}_{l'}d\Omega=\frac{1}{k^2}\sum_{l,m}^{}\vert C_{l,m}\vert ^2
\end{equation}
We know from the Legendre functions that,
\begin{equation}
	Y^0_l(\theta,\phi)=\sqrt{\frac{2l+1}{4\pi}}P_l(cos\theta)
\end{equation}
where $P_l$ is the $l$th Legendre Polynomial. Now the exact wave function in the exterior region is,
\begin{equation}
	\psi(r,\theta)=A\left\{e^{ikz}+\sum_{l=0}^{\infty}\sqrt{\frac{2l+1}{4\pi}}C_lh_l^{(1)}(kr)P_l(cos\theta)\right\}
\end{equation}
The scattering amplitude is now given by,
\begin{equation}
	f(\theta)=\frac{1}{k}\sum_{l=0}^{\infty}(-i)^{l+1}\sqrt{\frac{2l+1}{4\pi}}C_lP_l(cos\theta)
\end{equation}
and the total cross-section is,

\begin{equation}
	\sigma=\frac{1}{k^2}\sum_{l=0}^{\infty}|C_l|^2
\end{equation}
To fix the hybrid notation of the cartesian incoming wave and the spherical outgoing wave, we write it in a more consistent form.\\ 
We know that the general solution to the Schrodinger equation with $V=0$ can be written in the form,
\begin{equation}
	\sum_{l,m}^{}[A_{l,m}j_l(kr)+B_{l,m}n_l(kr)]Y^m_l(\theta,\phi)
\end{equation}
Expanding the plane wave in terms of spherical waves using Rayleigh's formula,
\begin{equation}
	e^{ikz}=\sum_{l=0}^{\infty}i^l(2l+l)j_l(kr)P_l(cos\theta)
\end{equation}
Substituting this in Equation (24), the consistent exterior region wave function can be written as,
\begin{equation}
	\psi(r,\theta)=A\left[l(2l+l)j_l(kr)+\sum_{l=0}^{\infty}\sqrt{\frac{2l+1}{4\pi}}C_lh_l^{(1)}(kr)\right]P_l(cos\theta)
\end{equation}

\section{Phase Shift}
Let's begin by considering a one-dimensional scattering problem with a localized potential on the half-line $x < 0$ and a brick wall at $x = 0$. So a wave incident from the left,
\begin{equation}
\psi_{i}(x) = A e^{ikx}
\end{equation}
is entirely reflected,
\begin{equation}
\psi_{r}(x) = B e^{-ikx}
\end{equation}
\begin{figure}[h]
	\centering
	\includegraphics[scale=0.6]{1d_scatt.png}
	\caption{1D scatterubg frin a localized potential bounded on the right by an infinite wall}
\end{figure}
where $x < -a$. No matter what happens in $-a < x < 0$ (the interaction region), the amplitude (amplitude in the context of waves not probability amplitude) of the reflected wave is the same as the incident wave simply due to conservation of probability. However, the two waves need not have the same phase. If there were no potential at all ($V(x) = 0$), but just at the wall ($x = 0$), then $B = -A$, since the total wave function, incident + reflected must vanish at the origin,
\begin{equation}
\psi_{0} = A (e^{ikx} - e^{-ikx})
\end{equation}
If the potential is not zero ($V(x) \neq 0$), then the wave function ($x < -a$) takes the form:
\begin{equation}
\psi = A \left(e^{ikx} - e^{i(2 \delta - kx)} \right)
\end{equation}
Thus, the whole scattering problem reduces to the problem of calculating the phase shift $\delta$ as a function of $k$ and hence of the Energy $E = \hbar^{2}k^{2}/2m$. Yes there's a factor of 2, before $\delta$, but that's only conventional. We think of the incident wave as being phase shifted once on the way in and again on the way out. Thus, by $\delta$ we mean the one-way phase shift and $2\delta$ the total phase shift. We go about this by solving the Schrodinger equation in $-a < x < 0$ along with relevant boundary conditions. Why are we working with $\delta$ rather than the complex amplitude $B$? It makes the physics and math simpler:
\begin{itemize}
\item \textbf{Physically:} We only need to think of the conservation of probability. The potential merely shifts the phase
\item \textbf{Mathematically:} We trade a complex number for a real one
\end{itemize}
Let's return to the 3D  case. The incident plane wave carries no angular momentum in the z direction. Thus Rayleigh's formula contains no terms with $m \neq 0$ but insteaed it cfontains all values of the total angular momentum ($l = 0, 1, 2$). Since angular momentum is conserved by a spherically symmetric potential each partial wave labelled by a particular $l$ scatters independently with no change in amplitude (amplitude in this context refer to the amplitude of the wave no the probability amplitude) but differing in phase. If there is no potential then $\psi_{0} = A e^{ikx}$ and the $l$th partial wave is
\begin{equation}
\psi^{l}_{0} = A i^{l} (2l +1) j_{l}(kr)P_{l}(\cos(\theta))
\end{equation}
But from our previous considerations,
\begin{equation}
j_{l}(x) = \frac{1}{2} \left[h^{(1)}(x) + h^{2}_{l}(x) \right] \approx \frac{1}{2x} \left[(-i)^{l+1} e^{ix} + i^{l+1}e^{-ix} \right]
\end{equation}
for $x >>1$. So for large $r$,
\begin{equation}
\psi^{(l)}_{0} \approx A \frac{2l + 1}{2ikr} \left[e^{ikr} - {(-1)}^{l}e^{-ikr}\right] P_{l}(\cos(\theta))
\end{equation}
The second term in square brackets corresponds to an incoming spherical wave. It is unchanged when we introduce the scattering potential. The first term is the outgoing wave. It picks up a phase shift $\delta_{l}$:
\begin{equation}
	\label{tobeass}
\psi^{(l)} \approx A \frac{2l + 1}{2ikr} \left[e^{i(kr + 2\delta_{l})} - {(-1)}^{l}e^{-ikr}\right] P_{l}(\cos(\theta))
\end{equation}
Think of it as a converging spherical wave due to the $h^{(2)}_{l}$ component in $e^{ikz}$, which is phase shifted by $2\delta_{l}$ and emerges as an outgoing spherical wave i.e. the $h^{l}_{l}$ part of $e^{ikz}$ as well  as the scattered wave itself. In the previous section the whole theoryu was expressed in terms of partial wave amplitudes $a_{l}$, now we have formulated it in terms of the phase shifts $\delta_{l}$. There must be a connectioni between the two. Well if we take the assymptotic i.e. large $r$ limit of eq. (\ref{tobeass}):
\begin{equation}
\psi^{(l)} \approx A\left( \frac{(2l + 1)}{2ikr} \left[e^{i(kr + 2\delta_{l})} - {(-1)}^{l}e^{-ikr}\right] + \frac{(2l + 1)}{r}a_{l}e^{ikr} \right) P_{l}(\cos(\theta))
\end{equation}
With the generic expression in terms of $e^{i \delta_{l}}$ we find
\begin{equation}
a_{} = \frac{1}{2ik}(e^{2i \delta_{l}} - 1) = \frac{1}{k} e^{i \delta_{l}} \sin(\delta_{l})
\end{equation}
Although we used the assymptotic form of the wave function to find the connection there's nothing approximate about the result. Both of them are constants independent of $r$ and $\delta_{l}$ means the phase shift in the asymptotic region i.e. where the Hankel functionis have settled down to $e^{\pm ikr}/kr$. It follows in particular that,
\begin{equation}
f(\theta) = \frac{1}{k}\sum_{l=0}^{\infty}(2l + 1)e^{i \delta_{l}} \sin(\delta_{l}) P_{l}(\cos(\theta)) 
\end{equation}
and,
\begin{equation}
\sigma = \frac{4 \pi}{k^{2}} \sum_{l=0}^{\infty}(2l + 1) \sin[2](\delta_{l})
\end{equation}
Voila!
\section{Born Approximation}
\subsection{Integral Form of the Schrodinger Equation}
Before we even head to deriving the "Integral Form of the Schrodinger Equation". Why you might ask? It will become evident in the upcoming sections. So let's begin by recalling the time-independent Schrodinger equation
\begin{equation}
\frac{num}{2m} + V \psi = E \psi
\end{equation}
We can rewrite this as,
\begin{equation}
(\nabla^{2} + k^{2}) \psi = Q
\end{equation}
where
$$k = \frac{\sqrt{2mE}}{\hbar}$$
$$Q = \frac{2m}{\hbar^{2}}V \psi$$
This looks pretty similar to the Helmholtz equation from electrodynamics. Here however the "inhomogeneous" term Q itself depends on $\psi$
Suppoase we could find a function that solves the Helmholtz equation with a delta function source:
\begin{equation}
(\nabla^{2} + k^{2}) G(\vec{r}) = \delta^{3}(\vec{r})
\end{equation}
We can then express as an integral:
\begin{equation}
	\psi(\vec{r}) = \int G(\vec{r}} - \vec{r_{0}}) Q(\vec{r_{0}}) d^{3}\vec{r_{0}}
\end{equation}
$G(\vec{r})$ is called the Green's function for the Helmholtz equation. Moreover, generally speaking the Green's function for a linear differential equation represents the response to a delta function. Our goal now is to solve this differential equation, we start by Fourier transforming it to turn it into an algebraic equation:
\begin{equation}
G(\vec{r}) = \frac{1}{{(2 \pi )}^{3/2}} \int e^{i \vec{s}. \vec{r}} g(\vec{s}) d^{3}\vec{s}
\end{equation}
Then,
$$$$
But,
$$$$
and
$$$$
thus,
$$$$
It follows from Plancherel's theorem that,
$$$$
Plugging this back, we see that
$$$$
Text
$$$$
Thus,
$$$$
We can rewrite this as,
$$$$
From Cauchy's integral formula it follows that,
$$$$
$$$$
Therfore,
\begin{equation}
	content...
\end{equation}
Note that $G + G_{0}$ still satisfies Equation (). This is simply due to the multivalued nature of the holomorphic function. Thus, the integral form of the Schrodinger equation can be written as,
\begin{tcolorbox}
\begin{equation}
\psi(\vec{r}) = \psi_{0}(\vec{r}) - \frac{m}{2 \pi \hbar^{2}}\int\frac{e^{ik\abs{\vec{r} - \vec{r}_{0}}}}{\abs{\vec{r} - \vec{r}_{0}}}  V(\vec{r}_{0}) \psi(\vec{r}_{0}) d^{3}\vec{r}_{0}
\end{equation}
\end{tcolorbox}
Let's see how this helps us.
\subsection{The First Born Approximation}
Suppose $V(\vec{r}_{0})$ is localized about r = 0, that is the potential drops to 0 after a finite region and we want to calculate $\psi(\vec{r})$ at points distant from the scattering center. Then for all points that contribute to the integral form of the Schrodinger equation. So,
\begin{equation}
\abs{\vec{r} - \vec{r}_{0}} = r^{2} + r_{0}^{2} - 2\vec{r}\vec{r_{0}}  \approxeq r^{2} \left(1 - 2\frac{\vec{r}.\vec{r}_{0}}{r^{2}} \right)
\end{equation}
and hence,
\begin{equation}
\abs{\vec{r} - \vec{r}_{0}} \approxeq r - \hat{r} . \vec{r}_{0}
\end{equation}
Let,
\begin{equation}
\vec{K} = k \hat{z}
\end{equation}
then
\begin{equation}
e^{-i \vec{K}\abs{\vec{r} - \vec{r}_{0}}} \approx e^{ikr}e^{-i \vec{K}.\vec{r}_{0}}
\end{equation}
and therefore,
\begin{equation}
\frac{e^{-i \vec{K}\abs{\vec{r} - \vec{r}_{0}}}}{\abs{\vec{r} - \vec{r}_{0}}} \approx \frac{e^{ikr}}{r}e^{-i \vec{K}.\vec{r}_{0}}
\end{equation}
In the case of scattering, we want:
\begin{equation}
\psi_{o}(\vec{r}) = Ae^{ikz}
\end{equation} 
to represent an incident plane wave. For large $r$,
\begin{equation}
\psi \approxeq Ae^{ikz} - \frac{m}{2 \pi \hbar^{2}A} \int e^{-i \vec{K}.\vec{r}_{0}}V(\vec{r}_{0})\psi(\vec{r}_{0})d^{3}\vec{r}_{0}
\end{equation}
This is in the standard form. We can read off the scattering amplitude:
\begin{equation}
f(\theta, \phi) = \frac{m}{2 \pi \hbar^{2}A} \int e^{-i \vec{K}.\vec{r}_{0}}V(\vec{r}_{0})\psi(\vec{r}_{0})d^{3}\vec{r}_{0}
\end{equation}
So far this is exact. Now we invoke the Born approximation: "Suppose the incoming plane wave is not substantially altered by the potential; then we can say that
\begin{equation}
\psi(\vec{r}_{0}) \approxeq \psi_{0}(\vec{r}_{0}) = A e^{ikz_{o}} = A e^{i\vec{K}^{'}\vec{r}_{o}}
\end{equation}
where
$$K^{'}= k \hat{z}$$
inside the integral. This would be just the wave function if $V$ were zero. It is essentially just a weak potential approximation. Generally partial wave analysis is useful when the incident particle has low energy the only the first few terms in the series contribute significantly. The Born approximation applies when the potential is weak when compared to the incident energy, thus the deflection is small. In the Born approximation then,
\begin{tcolorbox}
	\begin{equation}
		f(\theta, \phi) \approxeq -\frac{m}{2 \pi \hbar^{2}} \int e^{i(k^{'}-k). \vec{r}_{0}}V({r}_{0})d^{3}\vec{r}_{0}
	\end{equation}
\end{tcolorbox}
In particular, for low energy scattering, the exponential factor is essentially constant over the scattering region and the Born approximation simplifies to:
\begin{tcolorbox}
	\begin{equation}
		f(\theta, \phi) \approxeq -\frac{m}{2 \pi \hbar^{2}} \int V(\vec{r}) d^{3}r
	\end{equation}
\end{tcolorbox}
For a spherically symmetrical potential, $V(\vec{r}) = V(r)$ but not necessarily at low energy. The Born approximation reduces to a simpler form. First we define:
\begin{equation}
\mathcal{K} = k^{'} - k
\end{equation}
and let the polar axis for the $r_{0}$, the integral lies along so that;
\begin{equation}
(k^{'} - k).r_{0} = \mathcal{K}r_{0} \cos(\theta_{0})
\end{equation}
Then,
\begin{equation}
f(\theta) \approxeq -\frac{m}{2 \pi \hbar^{2}} \int e^{i \mathcal{K} r_{0} \cos(\theta_{0})} V(\theta_{0}) r^{2}_{0} \sin(\theta_{0}) dr_{0}d \theta_{0} d \phi_{0}
\end{equation}
The integral is trivial, $2\pi$, and the integral $\theta_{0}$ is on we have encountered before in equation (). Dropping the subscript on $r$, we are left with
\begin{tcolorbox}
\begin{equation}
f(\theta) \approxeq -\frac{2m}{\hbar^{2} \mathcal{K}} \int_{0}^{\infty} rV(r)\sin(\mathcal{K}r) dr
\end{equation}
\end{tcolorbox}
The angular dependence of $f$ is carried by $\mathcal{K}$. From our previous considerations we can see that:
\begin{equation}
\mathcal{K} = 2k \sin(\theta /2)
\end{equation}
\subsection{Examples}
\subsubsection{Low-energy soft-sphere scattering}
Note: We can't apply the Born approximationi to hard-sphere scattering as the integral blows up due to our assumption (i.e. potential does not affect the wave function) here. Suppose,
\begin{equation}
	V(\vec{r}) = \begin{cases}
		V_{0}, & \text{ if } r \leq a \\
		0, & \text{ if } r > a
	\end{cases}
\end{equation}
In this case the low-energy scattering amplitude is,
\begin{equation}
	f(\theta, \phi) \approxeq -\frac{m}{2 \pi \hbar^{2}}V_{0}\left(\frac{4}{3} \pi a^{3} \right)
\end{equation}
This is independent of $\theta$ and $\phi$ ! Thus, the differential cross-section is:
\begin{equation}
\frac{d \sigma}{d \Omega} = \abs{f}^{2}  \approxeq \left[\frac{2m V_{0} a^{3}}{3 \hbar^{2}} \right]^{2}
\end{equation}
and the total cross-section:
\begin{equation}
\sigma 	\approxeq 4 \pi \left(\frac{2m V_{0} a^{3}}{3 \hbar^{2}}\right)^{2}
\end{equation}
\subsubsection{Yukawa Scattering}
The Yukawa potential is a toy-model for the binding force in the nucleus of an atom. It has the form,
\begin{equation}
V(r) = \beta \frac{e^{-\mu r}}{r}
\end{equation}
where $\beta$ and $\mu$ are constants. The Born approximation gives,
\begin{equation}
f(\theta) \approxeq -\frac{2m \beta}{\hbar^{2}k} \int_{0}^{\infty} e^{- \mu r} \sin(kr) dr = - \frac{2m \beta}{\hbar^{2} (\mu^{2} + k^{2})}
\end{equation}
\subsubsection{Rutherford Scattering}
If we substitute $\beta = q_{1}q_{2}/ 4 \pi \epsilon_{0}$ and $\mu = 0$. The scattering amplitude is given by,
\begin{equation}
f(\theta) \approxeq - \frac{2m q_{1}q_{2}}{4 \pi \epsilon_{0} \hbar^{2} k^{2}}
\end{equation}
or,
\begin{equation}
f(\theta) \approxeq - \frac{q_{1}q_{2}}{16 \pi \epsilon_{0} E \sin[2](\theta / 2)}
\end{equation}
The differential cross-section is the square of this:
\begin{equation}
	\frac{d \sigma}{d \Omega} = \left[\frac{q_{1}q_{2}}{16 \pi \epsilon_{0} E \sin[2](\theta / 2)} \right]^{2}
\end{equation}
\subsection{The Born series}
The Born approximation is very similar to the impulse approximation in the contex of classical scattering. 
\begin{figure}[h]
	\centering
	\includegraphics[scale=0.5]{impulse_scatt.png}
	\caption{An example of the impulse approximation: the particle continues undeflected}
\end{figure}
In that sector we start by assuming that the particle keeps going in a straight line and compute the transverse impulse that would be delivered to it in that case:
\begin{equation}
I = \int F_{\perp} dt
\end{equation}
If the deflection is small in comparism to the motion, it would then be a good approximation to the transverse momentum supplied to th particle. Thus we express the scattering angle as:
\begin{equation}
\theta = \arctan(I/p)
\end{equation}
where $p$ is th incident momentum. This is the "first-order" impulse approximation. The zeroth-order is what we started wtih i.e. no deflection at all. Likewise, in the zeroth-order Born approximation the incident plane wave passes by with no modification and what we saw earlier was just the first order correction to this. But the same pattern of thought can lead us to a series which then leads us to higher-order corrections. Let's recall the integral form of the Schrodinger equation:
\begin{equation}
		\label{schrod_int_2}
	\psi (\vec{r}) = \psi_{0} (\vec{r}) + \int g(\vec{r} - \vec{r}_{0})V(\vec{r}_{0})\psi(\vec{r}_{0})d^{3}r_{0}
\end{equation}
where is the incident wave and,
$$g(\vec{r}) = -\frac{m}{2 \pi \hbar^{2}}\frac{e^{ikr}}{r}$$
is the Green's function with a factor $m/2 \pi \hbar^{2}$ for convenience and $V$ is the scattering potential.
Suppose we take the equation for $\psi$ and plug it back into (\ref{schrod_int_2}),
$$\psi = \psi_{0} + \int g V\psi_{0} + \int \int g V g V\psi$$
Iterating this we obtain the series expasion for $\psi$,
\begin{equation} 
	\label{dysonseries}
\psi = \psi_{0} + \int g V\psi_{0} + \int \int g V g V\psi_{0} + \int \int \int g V g V g V\psi_{0} ...
\end{equation}
\begin{figure}[h]
	\label{born_diag}
	\centering
	\includegraphics[scale=0.5]{born_series.png}
	\caption{A diagram representing the Born series}
\end{figure}
We notice the following from (\ref{dysonseries}):
\begin{itemize}
\item The first Born approximation truncates the series after the Next to Leading Order (NLO) term
\item In the Leading Order $\psi$ is untouched by $V$
\item In the first order (Next to Leading Order) it is kicked once
\item In the second order it is kicked, propagates to a new location and is kicked again and so on
\item In this context the Green's function is essentially just the propagator \footnote{In this context it tells us how the disturbance propagates between one interaction and the next}
\item This was in fact the inspiration for Feynman diagrams which is expressed in terms of vertex factors ($V$) and propagators ($g$)
\end{itemize}
Figure (\ref{born_diag}) might look familiar, because it closely represents Feynman diagrams.
\begin{figure}[h]
	\centering
	\includegraphics[scale=0.1]{2000px-Bhabha_S_channel.svg}
	\caption{Bhabha scattering: Annihilation}
\end{figure}
\chapter{Graphs \& Combinotorial Optimization}
\section{Combinotorial Optimization} 
Combinotorial Optimization concerns optimization problems of a discrete or combinatorial strcuture. It uses graphs and digraphs as basic tools.
\section{Graphs}

\section{Digraphs}
\section{Shortest Path Problems}
\subsection{Complexities}
\section{Bellman's Principle}
\section{Dijkstra's Algorithm}
\section{Shortest Spanning Trees}
\subsection{Greedy's Algorithm}
\subsection{Prim's Algorithm}
\section{Flows in Networks}
\subsection{Maximum Flow: Ford-Fulkerson Algorithm}
\section{Bipartite Graphs}

\chapter{Relativistic Quantum Mechanics}
\section{The Klein-Gordon Equation}
In order to develop quantum mechanics relativistically, they started from the relativistic
energy–momentum relation,
\begin{equation}
	E^2 - \big|\vec{p}\big|^2c^2 -m^2c^4 = 0
\end{equation}
Substituting $E$ and $p$ for their respective operators,

$$E \Rightarrow i\hbar\frac{\partial }{\partial t},\hspace{50pt}\vec{p} \Rightarrow -i\hbar\vec{\nabla},$$ 

and letting the equation act on a wavefunction $\phi(\vec{x} , t)$  the equation becomes,
\begin{equation}
	\Big( -\hbar^2\frac{\partial^2 }{\partial t^2} + \hbar^2c^2\nabla^2 - m^2c^4 \Big)\phi(\vec{x} , t) = 0
\end{equation}

This is the Klein-Gordon Equation. The wave function $\phi(\vec{x} , t)$ is also an object called a field because its arguments extend over all space and time. Another way to say this is that it exists
throughout spacetime and its fluctuations are described by the Klein–Gordon
equation. In natural units,
$$\Big( \frac{\partial^2}{\partial t^2} + \nabla^2 - m^2 \Big)\phi(\vec{x},t) = 0$$

This equation was the first attempt at a relativistic quantum mehcanical equation of a wave. it tells us how the field of a particle fluctuates.

The Klein-Gordon equation can also be written in the Lorentz-invariant form,
\begin{equation}
	(\partial^\mu\partial_\mu+m^2)\psi=0
\end{equation}

Where,
\begin{equation*}
	\partial^\mu\partial_mu\equiv\frac{\partial^2}{\partial t^2}- \frac{\partial^2}{\partial x^2}-\frac{\partial^2}{\partial y^2}-\frac{\partial^2}{\partial z^2}
\end{equation*}
The Klein-Gordon equation has plane wave solutions,
\begin{equation}
	\psi(x,t)=Ne^{i(p\cdot x-Et)}
\end{equation}

Substituting in Equation (3),
\begin{equation*}
	E^2\psi=p^2\psi+m^2\psi
\end{equation*}
\section{The Dirac Equation}
\section{Covariant Formalism}
\subsection{The Adjoint Spinor and the Covariant Current}
\section{Solutions to the Dirac Equation}
\subsection{Particles at Rest}
\subsection{General Free-Particle Solutions}
\section{Antiparticles}
\subsection{The Dirac Sea Interpretation}
\subsection{The Feynman-Stuckelberg Interpretation}
\subsection{Antiparticle Spinors}
%\chapter{Epilogue: What lies ahead}
Quantum foundations rant. Refer to:
\begin{itemize}
\item Lee Smolin
Carlo Rovelli

\end{itemize}
%\appendix
%\chapter{Analytical Mechanics}
\section{Lagrangian Mechanics}

\section{Hamiltonian Mechanics}


%\chapter{Classical Field Theory}
\section{Potentials and the Helmholtz Theorem}
\section{Electromagnetic Radiation}
\section{Electromagnetic Waves}
\section{Special Relativity}
\section{Electromagnetism in the Language of Tensors and Lagrangians}


\backmatter
\begin{thebibliography}{9}
	
	\bibitem{p1} Griffiths, D. J. (2005). Introduction to quantum mechanics. Upper Saddle River, NJ: Pearson/Prentice Hall 
	
	\bibitem{p2} Shankar, R (1994). Principles of quantum mechanics. New York: Springer
	
	\bibitem {p3} McIntyre, D. H., Manogue, C. A., \& Tate, J. (2016). Quantum mechanics. India: Pearson.
	
	\bibitem {p4} Townsend, J. S. (2000). Amodern approach to quantum mechancs. Saisalito, CA: University Science Books.
	
	 \bibitem {p5} Binney, J. J., \& Skinner, D. (2015). The physics of quantum mechanics. Oxford: Oxford University Press.
	
	\bibitem {p6} Wienberg, S. (2013). Lectures on quantum mechanics. Cambridge: Cambridge University Press.
	
	 \bibitem {p7} Bransden, B. H., \& Joachain, C. J. (2017). Quantum mechanics. Uttar Pradesh, India: Pearson.
	
	\bibitem {p8} Smolin, L. (2020). Einstein's unfinished revolution: The search for what lies beyond the quantum. Toronto: Vintage Canada.
	
	\bibitem {p9} Susskind, L., \& Friedman, A. (2015). Quantum mechanics: The theoretical minimum. S. l.: Penguin Books.
	
	\bibitem {p10} Schwichtenberg, J. (2020). No-nonsense quantum mechanics: A student-friendly introduction. Karlsruhe, Germany: No-Nonsense Books.
	
	\bibitem {p11} \emph{Introduction to elementary particles}
	\newblock Griffiths, D. J.
	\newblock Weinheim: Wiley-VCH Verlag, 2014
	
	\bibitem {p12} Larkoski, A. J. (2019). Elementary particle physics: An intuitive introduction. Cambridge University Press 
	
	\bibitem {p13} Peskin, M. (2018). Concepts of Elementary Particle Physics. Oxford Higher Education 
	
%	 \bibitem {p14} Arfken, G., Weber, H. J., \& Harris, F. E. (2013). Mathematical methods for physicists. Amsterdam: Elsevier Academic Press.
	
%	\bibitem {p15} Riley, K. F., \& Hobson, M. P. (2008). Mathematical methods for physics and engineering a comprehensive guide. Cambridge: Cambridge University Press.
	
%	\bibitem {p16} Artin, M (2018). Algebra.NY, NY: Pearson
	
%	\bibitem {p17} Fleisch, D. A. (2018). A student's guide to vectors and tensors. Cambridge: Cambridge University Press
%	
%	\bibitem {p18} Jeevanjee, N. (2015). An introduction to tensors and group theory for physicists: Nadir Jeevanjee. Cham: Birkhauser.
	
%	\bibitem {p19} Das, A. J. (2007). Tensors: The Mathematics of Relativity Theory and Continuum Mechanics. New York: Springer Science Business Media, LLC
	
%	 \bibitem {p20} Kees Dullemond \& Kasper Peeters (1991), Introduction to Tensor Calculus 
		
%	\bibitem {p21} Nakahara, M. (2017). Geometry, Topology and Physics. Boca Raton, FL: CRC Press.
	
%	\bibitem {p22} Neuenschwander, D. E. (2015). Tensor Calculus for Physics: A Concise Guide. Baltimore, MD: JOHNS HOPKINS University Press.
	
%	 \bibitem {p23} Oppenheim, A. V., Willsky, A. S., \&amp; Nawab, S. H. (2015). Signals \&amp; systems. Second edition.: Pearson.
	
%	\bibitem {p24} HOOFT, G. '. (2018). Cellular automaton interpretation of quantum mechanics. Place of publication not identified: SPRINGER INTERNATIONAL PU.
	 
%	 \bibitem {p25} Rovelli, C. (1996). Relational quantum mechanics. International Journal of Theoretical Physics, 35(8), 1637-1678. doi:10.1007/bf02302261
	 
%	 \bibitem{p26} Hossenfelder, S., \&amp; Palmer, T. (2020). Rethinking Superdeterminism. Frontiers in Physics, 8. doi:10.3389/fphy.2020.00139
	 
%	 \bibitem {p27} Laloë, F. (2019). Do we really understand quantum mechanics? Cambridge: Cambridge University Press.
	 
%	 \bibitem {p28} Maudlin, T. (2011). Quantum non-locality and relativity metaphysical intimations of modern physics. Malden, MA: Wiley-Blackwell.
	 
%	 \bibitem {p29} Gisin, N. (2019). Real numbers are the hidden variables of classical mechanics. Quantum Studies: Mathematics and Foundations, 7(2), 197-201. doi:10.1007/s40509-019-00211-8
	 
%	 \bibitem {p30} Bassi, A., Lochan, K., Satin, S., Singh, T. P., \& Ulbricht, H. (2013). Models of wave-function collapse, underlying theories, and experimental tests. Reviews of Modern Physics, 85(2), 471-527. doi:10.1103/revmodphys.85.471
	 
%	 \bibitem {p31} Schlosshauer, M., Kofler, J., \& Zeilinger, A. (2013). A snapshot of foundational attitudes toward quantum mechanics. Studies in History and Philosophy of Science Part B: Studies in History and Philosophy of Modern Physics, 44(3), 222-230. doi:10.1016/j.shpsb.2013.04.004
	 
%	 \bibitem {p32} Norsen, T. (2017). Foundations of quantum mechanics: An exploration of the physical meaning of quantum theory. Cham: Springer International Publishing AG.
     
 %    \bibitem {p33} Needham, T. (2012). Visual complex analysis. Oxford: Clarendon Press.
     
%     \bibitem {p34} Needham, T. (2012). Visual complex analysis. Oxford: Clarendon Press. 
\end{thebibliography}

\end{document}