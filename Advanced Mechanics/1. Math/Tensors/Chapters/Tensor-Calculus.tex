\chapter{Tensor-Calculus}
\section{Derivatives}
A derivative can be broken down as,
\begin{equation}
\frac{d \vec{R}}{d \lambda} = \frac{d c^{i}}{d \lambda} \frac{\partial \vec{R}}{\partial c^{i}}
\end{equation}
where $\frac{d c^{i}}{d \lambda}$ represents the coefficients and $\frac{\partial \vec{R}}{\partial c^{i}}$ represents the basis vectors.
We can rewrite the gradient for a scalar field we know from vector calculus as, 
\begin{equation}
{(grad \ f)}_{\mu} = \frac{\partial f}{\partial x^{\mu}}
\end{equation}
we can write the gradient of a vector field as,
\begin{equation}
{(grad \ \vec{v})}^{\mu}_{\  \nu} = \frac{\partial x^{\mu}}{\partial x^{\nu}}
\end{equation}
and of a Tensor field,
\begin{equation}
{(grad \ t)}_{\mu \nu \alpha} = \frac{\partial t_{\mu \nu}}{\partial x^{\alpha}}
\end{equation}
We can use the following notation for simplicity,
\begin{equation}
v^{\mu}_{ \ ,\nu} = \partial_{\nu} v^{\mu} := \frac{\partial v^{\mu}}{\partial x^{\nu}}
\end{equation}
and 
\begin{equation}
v^{\mu,\nu} = \partial^{\nu} v^{\mu} := g^{\nu \rho} \frac{\partial v^{\mu}}{\partial x^{\rho}}
\end{equation}
\section{Covector Field}
$df$ is called a differential or a differential form of $f$. We can think of the operators $d$ as something that take in a scalar field $f$ (function/0-form) and outputs a covector field $df$ (level sets/1-form). \\
A 1-form is defined as a \textbf{linear-map},
$$\vec{V} \rightarrow \mathbb{R}$$
That is,
$$df(\vec{V}) \in \mathbb{R}$$
Geometrically speaking, long arrows of the vector field match up with the densly packed areas of the covector field and vice versa. $df$ can be thought of being proportional to the steepness of $f$ in the direction of $\vec{V}$ and to the length of $\vec{V}$. $df$ tells us the rate of change of $f$ moving at a velocity $\vec{V}$. Simply put, $df(\vec{V})$ is the directional derivative.
\subsection{Covector Field Components}
$dx$ and $dy$ form the dual-basis of differential forms.\footnote{The proof for this can be found when you apply $df$ to $\frac{\partial}{\partial x}$ and find that it is $1$. Thus generally, }
$$df = A dx + B dy$$
Where $A = \frac{\partial f}{\partial x}$ and $B = \frac{\partial f}{\partial y}$. \footnote{This can be derived from trying to take the derivative of $f$ with respect to a tangent vector or by simply expanding the differential $df$}
Or more eloquently,
$$df = \frac{\partial f}{\partial c^{i}} dc^{i}$$
\subsection{Covector Field Transformation Rules}
\subsection{Covector Field Components}
\section{Integration with Differential Forms}

\section{Gradient}
\begin{equation}
df(\vec{v}) = \nabla f \cdot \vec{v} 
\end{equation}
So $df$ is the dual covector field of $\nabla f$. It follows this relation,
\begin{equation}
df = (\nabla f)^{i} g_{ij}dc^{j} 
\end{equation}
where $dc^{j}$ are the basis covectors. We know that,
$$df = \frac{df}{dc^{j}}dc^{j}$$
Thus,
$$\frac{df}{dc^{j}}dc^{j} = (\nabla f)^{i} g_{ij}dc^{j}$$
\begin{equation}
	\frac{df}{dc^{j}} = (\nabla f)^{i} g_{ij}
\end{equation}
Now if we multiply by the contravariant metric tensor on both the sides,
$$\frac{df}{dc^{j}}\mathfrak{g}^{ij} = (\nabla f)^{i} g_{ij}\mathfrak{g}^{ij}$$
$$\frac{df}{dc^{j}}\mathfrak{g}^{ij} = (\nabla f)^{i} \delta^{k}_{i}$$
we find that
\begin{equation}
	\frac{df}{dc^{j}} \mathfrak{g}^{ij} = (\nabla f)^{j}
\end{equation}
the inverese metric takes us from the Gradient to the differential. We can also expand the components out as alinear combinations of the basis or in vector notation. It is quite common in this context to use $\flat$ and $\sharp$ instead of $\mathfrak{g}^{ij}$ and $g_{ij}$ .
\section{Geodesics}
\section{Covariant Derivative}
The covariant derivative of a vector field $v^{\mu}$ is defined as,
\begin{equation}
\nabla_{\mu} v^{\alpha} = v^{\alpha}_{ \  ; \mu} = \partial_{\mu} v^{\alpha} + \Gamma^{\alpha}_{\mu \nu} v^{\nu}
\end{equation}
where, the object $\Gamma^{\alpha}_{\mu \nu}$ is called the Christoffel symbol. The Christoffel symbol is not a tensor because it contains all the information about the curvature of the coordinate system and can therefore be transformed entirely to zero if the coordinates are straightened. Nevertheless we treat it as any ordinary tensor in terms of the index notation. It can be defined in terms for derivatives as:
\begin{equation}
\Gamma^{k}_{ij}\vec{e}_{k} = \frac{\partial \vec{e}_{i}}{\partial x^{j}}
\end{equation}
Where, the index $i$ specifies the basis vector for which the derivative is being
taken, the index $j$ denotes the coordinate being varied to induce this change
in the $i$th basis vector, and the index $k$ identifies the direction in which this
component of the derivative points. We define the Christoffel symbol in terms of the metric $g_{\mu \nu}$ and it's inverse $g^{\alpha \beta}$ as,
\begin{equation}
\Gamma^{\alpha}_{\ \mu \nu} = \frac{1}{2} g^{\alpha \beta} \left(\frac{\partial g_{\beta \nu}}{\partial x^{\mu}} + \frac{\partial g_{\beta \mu}}{\partial x^{\nu}} - \frac{\partial g_{\mu \nu}}{\partial x^{\beta}}\right) = \frac{1}{2}g^{\alpha \beta} \left(g_{\beta \nu, \mu} + g_{\beta \mu, \nu} - g_{\mu \nu, \beta}\right)
\end{equation}
We can define the covariant derivative of a covector as,
\begin{equation}
\nabla_{\mu} w_{\alpha} = w^{\alpha}_{ \  ; \mu} = \partial_{\mu} w_{\alpha} + \Gamma^{\alpha}_{\mu \nu} w_{\nu}
\end{equation}
The covariant derivative of a tensor $t^{\alpha\beta}$ is then,
\begin{equation}
\nabla_{\mu}t^{\alpha \beta} = \partial_{\mu} t^{\alpha \beta} + \Gamma^{\alpha}_{\mu \sigma}t^{\sigma \beta} + \Gamma^{\beta}_{\mu \sigma}t^{\alpha \sigma }
\end{equation}
and of a tensor $t^{\alpha}_{\beta}$,
\begin{equation}
\nabla_{\mu}t^{\alpha}_{\beta} = \partial_{\mu} t^{\alpha}_{\beta} + \Gamma^{\alpha}_{\mu \sigma}t^{\sigma}_{\beta} + \Gamma^{\beta}_{\mu \sigma}t^{\alpha}_{\sigma}
\end{equation}
\subsection{Properties}
\begin{itemize}
\item The covariant derivative produces, as its name says, covariant expressions.
\item $\nabla_{\mu} g_{\alpha \beta} = 0$
\item $g^{\alpha \gamma} \nabla_{\alpha}t^{\mu \nu}_{\gamma} = \nabla_{\alpha}(t^{\mu \nu}_{\gamma}g^{\alpha \gamma}) = \nabla_{\alpha}t^{\mu \nu \alpha}$
\item $\nabla^{\alpha} = g^{\alpha \beta}\nabla_{\beta}$
\end{itemize}
\section{Lie Brackets and Flow}
\section{Interesting Tensors}
\subsection{Kronecker delta}
It simply has the ‘function’ of ‘renaming’ an index:
$$\delta^{\mu}_{\nu} x^{\nu} = x^{\mu}$$
it is in a sense simply the identity matrix.
\subsection{Levi-Civita Pseudotensor}
\label{Levi}
The Levi-Civita Pseudotensor i.e. Tensor density is a completely anti-symmetric i.e. $\epsilon_{ijk} = -\epsilon_{jik} = -\epsilon_{ikj} = -\epsilon_{kji}$, we define it as:
\begin{equation}
\epsilon_{ijk} = \begin{cases}
1 \ \text{if } ijk \text{ is an even permuation of } 123\\
-1 \ \text{if } ijk \text{ is an odd permuation of } 123\\
0  \text{ if two indices are equal}\\
\end{cases}
\end{equation}
\subsubsection{Identities}
\begin{equation}
\epsilon_{\alpha \beta \nu}\epsilon_{\alpha \beta \sigma} = \delta_{\mu \rho} \delta_{\nu \sigma} - \delta_{\mu \sigma}\delta_{\nu \rho}
\end{equation}
From this it follows that,
\begin{equation}
\epsilon_{\alpha \beta \nu}\epsilon_{\alpha \beta \sigma} = 2\delta_{\nu \sigma}
\end{equation}
and
\begin{equation}
\epsilon_{\alpha \beta \gamma}\epsilon_{\alpha \beta \gamma} = 6
\end{equation}
\subsubsection{Cross-Product}
Using these identities and the definition we can rewrite the cross-product of two vectors as,
\begin{equation}
\vec{a} = \vec{a} \cross \vec{b}  = \epsilon_{ijk}a_{j}b_{k}
\end{equation}
Thus the expressions in vector product notation can be changed to index notation for example,
$$
c = \nabla.(\nabla \cross \vec{a}) = \nabla_{i}(\epsilon_{ijk}\nabla_{j}a_{k}) = \epsilon_{ijk}\partial_{i}\partial_{j}a_{k}
$$
because,
$$\nabla_{i} = \frac{\partial}{\partial x_{i}} := \partial_{i}$$
\subsubsection{Rot}
Rot is defined as the generalized rotation of a covector,
\begin{equation}
{(rot \tilde{w})}_{\alpha \beta} = \partial_{\alpha} w_{\beta} - \partial_{\beta}w_{\alpha}
\end{equation}
\subsection{Electromagnetic Field Tensor}
\begin{equation}
F^{\mu \nu} = \begin{pmatrix}
0 & -E_{x}/c & -E_{y}/c & -E_{z}/c\\
E_{x}/c & 0 & -B_{z} & B_{y}\\
E_{y}/c & B_{z} & 0 & -B_{x}\\
E_{z}/c & -B_{y} & B_{x} & 0
\end{pmatrix}
\end{equation}
and it's covariant version (by lowering the indices),
\begin{equation}
F_{\mu \nu} = \begin{pmatrix}
0 & E_{x}/c & E_{y}/c & E_{z}/c\\
-E_{x}/c & 0 & -B_{z} & B_{y}\\
-E_{y}/c & B_{z} & 0 & -B_{x}\\
-E_{z}/c & -B_{y} & B_{x} & 0
\end{pmatrix}
\end{equation}
and it's dual as,
\begin{equation}
G^{\mu \nu} = \begin{pmatrix}
0 & -B_{x} & -B_{y} & -B_{z}\\
B_{x}/c & 0 & E_{z}/c & -E_{y}/c\\
B_{y}/c & -E_{z} & 0 & E_{x}/c\\
B_{z}/c & E_{y} & -E_{x} & 0
\end{pmatrix}
\end{equation}
Along with the 4-current (the four-dimensional analogue of the electric current density), $$J^{\mu} = (c\rho , j^{1}, j^{2}, j^{3})$$
we can now rewrite Maxwell's equations as:
\begin{equation}
\frac{\partial F^{\alpha \beta}}{\partial x^{\alpha}} = \mu_{0} J^{\beta}
\end{equation}
\begin{equation}
\frac{\partial G^{\alpha \beta}}{\partial x^{\alpha}} = 0
\end{equation}
\subsection{Torsion Tensor}
\subsection{The Riemann Curvature Tensor}
\subsubsection{Derivation}
To begin this process we first take the covariant derivative of a vector $V_{\alpha}$, with respect to $x^{\beta}$,
\begin{equation}
	V_{\alpha ; \beta} = \frac{\partial V_{\alpha}}{\partial x^{\beta}} - \Gamma^{\sigma}_{\alpha \beta} V_{\sigma}
\end{equation}
Now call this result $V_{\alpha \beta}$ and take another covariant derivative, this time with
respect to $x^{\gamma}$,
\begin{equation}
	V_{\alpha \beta; \gamma} = \frac{\partial V_{\alpha \beta }}{\partial x^{\gamma}} - \Gamma^{\tau}_{\alpha \gamma} V_{\tau \beta} - \Gamma^{\eta}_{\beta \gamma} V_{\alpha \eta	} 
\end{equation}
Substituting the expression from Eq. 6.21 into this equation gives,
\begin{equation} \label{eq1}
\begin{split}
V_{\alpha \beta; \gamma} & = \frac{\partial^{2} V_{\alpha}}{\partial x^{\gamma} \partial x^{\beta}} - \frac{\partial \Gamma^{\sigma}_{\alpha \beta}}{\partial x^{\gamma}}V_{\sigma}- \Gamma^{\sigma}_{\alpha \beta} \frac{\partial V_{\sigma}}{\partial x^{\gamma}} \\
& - \Gamma^{\tau}_{\alpha \gamma} \left( \frac{\partial V_{\tau}}{\partial x^{\beta}} -\Gamma^{\sigma}_{\tau \beta} V_{\sigma} \right)\\
& - \Gamma^{\eta}_{\beta \gamma} \left( \frac{\partial V_{\alpha}}{\partial x^{\eta}} -\Gamma^{\sigma}_{\alpha \eta} V_{\sigma} \right)
\end{split}
\end{equation}
What  we've done is ask what's the incremental change in V when I head in the xmu direction and then ask what happens to this quantity when I take a step in the $x^{\gamma}$ direction. Now we try to do the opposite of that and try to find the difference between the two by using the commutator.
\begin{equation}
V_{\alpha ; \gamma} = \frac{\partial V_{\alpha}}{\partial x^{\gamma}} - \Gamma^{\sigma}_{\alpha \gamma} V_{\sigma}
\end{equation}
Call this result $V_{\alpha \gamma}$ and take another covariant derivative, this time with respect to $x_{\beta}$:
\begin{equation}
V_{\alpha ; \gamma} = \frac{\partial V_{\alpha}}{\partial x^{\gamma}} - \Gamma^{\sigma}_{\alpha \gamma} V_{\sigma}
\end{equation}
As before, you can substitute the expression from Eq. 6.24 into this equation to get,
\begin{equation} \label{eq2}
\begin{split}
	V_{\alpha \gamma;\beta } & = \frac{\partial^{2} V_{\alpha}}{\partial x^{\beta} \partial x^{ \gamma}} - \frac{\partial \Gamma^{\sigma}_{\alpha \gamma}}{\partial x^{\beta	}}V_{\sigma}- \Gamma^{\sigma}_{\alpha \gamma} \frac{\partial V_{\sigma}}{\partial x^{\beta}} \\
	& - \Gamma^{\tau}_{\alpha \beta} \left( \frac{\partial V_{\tau}}{\partial x^{\gamma}} -\Gamma^{\sigma}_{\tau \gamma} V_{\sigma} \right)\\
	& - \Gamma^{\eta}_{\gamma \beta} \left( \frac{\partial V_{\alpha}}{\partial x^{\eta}} -\Gamma^{\sigma}_{\alpha \eta} V_{\sigma} \right)
\end{split}	
\end{equation}
In flat space, the order of covariant differentiation should make no difference, so \ref{eq2} should be identical to \ref{eq1}. Any differences between these equations can therefore be attributed to the curvature of the space. If we examine these equations term by term, we can see that the first two terms are equal,
$$\frac{\partial^{2} V_{\alpha}}{\partial x^{\gamma} \partial x^{\beta}} = \frac{\partial^{2} V_{\alpha}}{\partial x^{\gamma} \partial x^{\beta}}$$
These two terms are equal because the order of applying partial derivatives doesn't matter when the field we consider is continuous. Hence these terms cancel out in the commutator. Now if we compare the second terms, we see that these terms don't cancel out,
$$-\frac{\partial \Gamma^{\sigma}_{\alpha \beta}}{\partial x^{\gamma}}V_{\sigma} \neq -\frac{\partial \Gamma^{\sigma}_{\alpha \gamma}}{\partial x^{\beta}}V_{\sigma}$$
Comparing the third and fourth terms, we can see that they are equal
$$-\Gamma^{\sigma}_{\alpha \beta} \frac{\partial V_{\tau}}{\partial x^{\gamma}} = -\Gamma^{\tau}_{\alpha \beta}\frac{\partial V_{\sigma}}{\partial x^{\gamma}}$$
because the symbols used for dummy indices $\sigma$ and $\tau$ are irrelevant. The fourth term of \ref{eq1} equals the third term of \ref{eq2}:
$$-\Gamma^{\tau}_{\alpha \gamma} \frac{\partial V_{\tau}}{\partial x^{\beta}} = -\Gamma^{\sigma}_{\alpha \gamma}\frac{\partial V_{\sigma}}{\partial x^{\beta}}$$
for the same reason. The fifth terms are not equal:
$$\Gamma^{\tau}_{\alpha \gamma}\Gamma^{\sigma}_{\tau \beta} V_{\sigma} \neq \Gamma^{\tau}_{\alpha \beta}\Gamma^{\sigma}_{\tau \gamma} V_{\sigma}$$
But the sixth terms are equal:
$$-\Gamma^{\eta}_{\beta \gamma} \frac{\partial V_{\alpha}}{\partial x^{\eta}} = -\Gamma^{\eta}_{\gamma \beta}\frac{ \partial V_{\alpha}}{\partial x^{\eta}}$$
because Christoffel symbols are symmetric in their lower indices. The seventh terms are equal for the same reason:
$$\Gamma^{\eta}_{\beta \gamma}\Gamma^{\sigma}_{\alpha \eta}V_{\sigma} = \Gamma^{\eta}_{\gamma \beta}\Gamma^{\sigma}_{\alpha \eta}V_{\sigma}$$
So when the commutator is formed, most of the terms cancel out, but the second and fifth terms remain after subtraction. Those terms are
\begin{equation}
\begin{split}
V_{\alpha \beta ; \gamma} - V_{\alpha \gamma ; \beta} & = \frac{\partial \Gamma^{\sigma}_{\alpha \gamma}}{\partial x^{\beta}}V_{\sigma} - \frac{\partial \Gamma^{\sigma}_{\alpha \beta}}{\partial x^{\gamma}}V_{\sigma} + \Gamma^{\tau}_{\alpha \gamma} \Gamma^{\sigma}_{\tau \beta}V_{\sigma} - \Gamma^{\tau}_{\alpha \beta} \Gamma^{\sigma}_{\tau \gamma}V_{\sigma} \\
& = \left( \frac{\partial \Gamma^{\sigma}_{\alpha \gamma}}{\partial x^{\beta}} - \frac{\partial \Gamma^{\sigma}_{\alpha \beta}}{\partial x^{\gamma}} + \Gamma^{\tau}_{\alpha \gamma} \Gamma^{\sigma}_{\tau \beta} - \Gamma^{\tau}_{\alpha \beta} \Gamma^{\sigma}_{\tau \gamma} \right) V_{\sigma}
\end{split}
\end{equation}
The terms within the parentheses define the Riemann curvature tensor:
\begin{equation}
R^{\sigma}_{\alpha \beta \gamma} = \frac{\partial \Gamma^{\sigma}_{\alpha \gamma}}{\partial x^{\beta}} - \frac{\partial \Gamma^{\sigma}_{\alpha \beta}}{\partial x^{\gamma}} + \Gamma^{\tau}_{\alpha \gamma} \Gamma^{\sigma}_{\tau \beta} - \Gamma^{\tau}_{\alpha \beta} \Gamma^{\sigma}_{\tau \gamma}
\end{equation}
The reason we have derivatives . Thus, the necessary and sufficient condition for flat space is,
\begin{equation}
R^{\sigma}_{\alpha \beta \gamma} = 0
\end{equation}
\subsection{The Ricci Tensor}
A tensor related to the Riemann curvature tensor is the Ricci tensor, which you can find by contracting the Riemann tensor along the $\sigma$ and $\beta$ indices, written in four dimensions as,
\begin{equation}
R_{\alpha \gamma} = R^{\sigma}_{\alpha \sigma \gamma} = R^{1}_{\alpha 1 \gamma}	+ R^{2}_{\alpha 2 \gamma} + R^{3}_{\alpha 3 \gamma} + R^{4}_{\alpha 4 \gamma}
\end{equation}
\subsection{Einstein Tensor}
The Einstein Tensor is constructed using a combination of the Ricci tensor, Ricci scalar and the metric
\begin{equation}
	G_{\alpha \gamma} = R_{\alpha \gamma} - \frac{1}{2}R g_{\alpha \gamma}
\end{equation}
This appears in Einstein's field equation for General Relativity, which is commonly written as
\begin{equation}
G_{\mu \nu} + \Lambda g_{\mu \nu} = \frac{8 \pi G}{c^{4}} T_{\mu \nu}
\end{equation}
where $\Lambda$, is the "Cosmological Constant" (which in a very loose sense denote the expansion of the universe), $c$ the speed of light and $G$ the gravitational constant.
\section{Interesting Scalars}
\subsection{$g$}
$g$ is the determinant of the metric tensor, given by
\begin{equation}
g = det(g_{\mu \nu})
\end{equation}
A straightforward calculation shows that under a coordinate transformation $x^{\mu} \rightarrow x^{\mu^{'}}$, this
doesn’t transform by the tensor transformation law (under which it would have to be invariant, since it has no indices), but instead as
\begin{equation}
	g \rightarrow  \left[det \left(\frac{\partial x^{\mu^{\prime}}}{x^{\mu}} \right) \right]^{-2} g
\end{equation}
Here, the factor $det(\partial x^{\mu^{'}}/\partial x^{\mu})$ is the Jacobian of the transformation. Objects with this kind of transformation law which involve powers of the Jacobian are known as \textbf{tensor densities}; the determinant $g$ is sometimes called a “scalar density.”
\subsection{Volume Element}
The Volume element is a tensor density which can be exemplified as $dx^{4} = dx^{0}dx^{1}dx^{2}dx^{3}$. It transforms like,
\begin{equation}
dx^{4} \rightarrow det \left(\frac{x^{\mu^{\prime}}}{x^{\mu}} \right) dx^{4}
\end{equation}
\subsection{Ricci Scalar}
This is the Trace (sum of the elements in the leading diagonal) of the Ricci Tensor
\begin{equation}
R = R^{\lambda}_{\lambda} = g^{\mu \nu}R_{\mu \nu} = Tr(R_{\mu \nu})
\end{equation}
This plays an important role in General Relativity
