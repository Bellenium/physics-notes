\chapter{Addition of Angular Momentum}
\section{The General Problem}
\subsection{Addition of $\hat{L}$ and $\hat{S}$}
\subsection{Modified Spectroscopic Notation}
In absence of spin using s,p,d.. is sufficient to describe angular momentum. However, in the presence of spin we change out notation to:
\begin{itemize}
\item Use capital letters S,P,D or L typically to indicate the value of angular momentum
\item Append a subscript $J$ to the right of $L$ i.e. $L_{J}$ to indicate the $j$ value
\item Append a superscript 2S+1 to the left of $L$ i.e $\prescript{14}{2}L$. to indicate the degeneracy due to spin projections
\end{itemize}
\section{Irreducible Tensor Operators}
\subsection{Tensor Operators}
We know that a vector can be linearly expanded in terms of it's basis,
\begin{equation}
\ket{V} = \sum^{3}_{i} v_{i}\ket{i}
\end{equation}
A second rank tensor similarly can be expressed as,
\begin{equation}
	\ket{T^{(2)}} = \sum^{3}_{i}\sum^{3}_{j} t_{ij}\ket{i}\otimes\ket{j}
\end{equation}
\begin{itemize}
\item 
\end{itemize}