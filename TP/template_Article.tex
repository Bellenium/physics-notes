\documentclass[]{article}

%opening
\title{Notes on Thermal Physics}
\author{Pugazharasu A D}

\begin{document}

\maketitle

%\begin{abstract}

%	\end{abstract}

\section{What is Thermodynamics?}
Thermodynamics is a branch of physics that deals with heat, work, and temperature, and their relation to energy, radiation, and physical properties of matter. The behavior of these quantities is governed by the four laws of thermodynamics which convey a quantitative description using measurable macroscopic physical quantities.
\section{Thermodynamic Processes}
\begin{itemize}
\item Defined by change in a system, a thermodynamic process is a passage of a thermodynamic system from an initial to a final state of thermodynamic equilibrium. 
\item The initial and final states are the defining elements of the process. The actual course of the process is not the primary concern, and thus often is ignored. This is the customary default meaning of the term 'thermodynamic process'. 
\item In general, during the actual course of a thermodynamic process, the system passes through physical states which are not describable as thermodynamic states, because they are far from internal thermodynamic equilibrium.
\end{itemize}
\subsection{Isothermal Processes}
\begin{itemize}
\item An isothermal process is a change of a system, in which the temperature remains constant: $\Delta T =0$.
\item  This typically occurs when a system is in contact with an outside thermal reservoir (heat bath), and the change in the system will occur slowly enough to allow the system to continue to adjust to the temperature of the reservoir through heat exchange.
\item In contrast, an adiabatic process is where a system exchanges no heat with its surroundings ($Q = 0$).
\item In other words, in an isothermal process, the value $\Delta T = 0$ and therefore the change in internal energy $\Delta U = 0$ (only for an ideal gas) but $Q \neq 0$, while in an adiabatic process, $\Delta T \neq 0$ but $Q = 0$.
\item Simply, we can say that in an isothermal process:
\begin{itemize}
\item $T = $ constant
\item $dt = \Delta T = 0$
\end{itemize}
\end{itemize}
\subsubsection{Examples}
\begin{itemize}
\item Isothermal processes can occur in any kind of system that has some means of regulating the temperature, including highly structured machines, and even living cells. Some parts of the cycles of some heat engines are carried out isothermally (for example, in the Carnot cycle). 
\item In the thermodynamic analysis of chemical reactions, it is usual to first analyze what happens under isothermal conditions and then consider the effect of temperature. Phase changes, such as melting or evaporation, are also isothermal processes when, as is usually the case, they occur at constant pressure.
\item Isothermal processes are often used and a starting point in analyzing more complex, non-isothermal processes.
\item Isothermal processes are of special interest for ideal gases. 
\item This is a consequence of Joule's second law which states that the internal energy of a fixed amount of an ideal gas depends only on its temperature.	 
\item Thus, in an isothermal process the internal energy of an ideal gas is constant. 
This is a result of the fact that in an ideal gas there are no intermolecular forces. Note that this is true only for ideal gases; the internal energy depends on pressure as well as on temperature for liquids, solids, and real gases.
\item In the isothermal compression of a gas there is work done on the system to decrease the volume and increase the pressure.
\item Doing work on the gas increases the internal energy and will tend to increase the temperature. To maintain the constant temperature energy must leave the system as heat and enter the environment. 
\item If the gas is ideal, the amount of energy entering the environment is equal to the work done on the gas, because internal energy does not change. For isothermal expansion, the energy supplied to the system does work on the surroundings. 
\item In either case, with the aid of a suitable linkage the change in gas volume can perform useful mechanical work. For details of the calculations, see calculation of work.
\item For an adiabatic process, in which no heat flows into or out of the gas because its container is well insulated, $Q = 0. $
\item If there is also no work done, i.e. a free expansion, there is no change in internal energy. For an ideal gas, this means that the process is also isothermal. 
\item Thus, specifying that a process is isothermal is not sufficient to specify a unique process. 
\end{itemize}


\subsection{Adiabatic Process}
\begin{itemize}
\item A process without transfer of heat or matter to or from a system, so that $Q = 0$, is called adiabatic, and such a system is said to be adiabatically isolated. 
\item The assumption that a process is adiabatic is a frequently made simplifying assumption. 
\item For example, the compression of a gas within a cylinder of an engine is assumed to occur so rapidly that on the time scale of the compression process, little of the system's energy can be transferred out as heat to the surroundings. 
\item Even though the cylinders are not insulated and are quite conductive, that process is idealized to be adiabatic. The same can be said to be true for the expansion process of such a system. 

\end{itemize}

\subsubsection{Examples}
For a closed system, one may write the first law of thermodynamics as : $\Delta U = Q – W$, where $\Delta U$ denotes the change of the system's internal energy, $Q$ the quantity of energy added to it as heat, and $W$ the work done by the system on its surroundings.
\begin{itemize}
\item If the system has such rigid walls that work cannot be transferred in or out ($W = 0$), and the walls are not adiabatic and energy is added in the form of heat ($Q > 0$), and there is no phase change, then the temperature of the system will rise.
\item If the system has such rigid walls that pressure–volume work cannot be done, but the walls are adiabatic ($Q = 0$), and energy is added as isochoric work in the form of friction or the stirring of a viscous fluid within the system ($W < 0$), and there is no phase change, then the temperature of the system will rise.
\item If the system walls are adiabatic ($Q = 0$) but not rigid ($W \neq 0$), and, in a fictive idealized process, energy is added to the system in the form of frictionless, non-viscous pressure–volume work ($W < 0$), and there is no phase change, then the temperature of the system will rise. 
\item Such a process is called an isentropic process and is said to be "reversible". Fictively, if the process were reversed the energy could be recovered entirely as work done by the system. 
\item If the system contains a compressible gas and is reduced in volume, the uncertainty of the position of the gas is reduced, and seemingly would reduce the entropy of the system, but the temperature of the system will rise as the process is isentropic ($\Delta S = 0$). 
\item Should the work be added in such a way that friction or viscous forces are operating within the system, then the process is not isentropic, and if there is no phase change, then the temperature of the system will rise, the process is said to be "irreversible", and the work added to the system is not entirely recoverable in the form of work.
\item If the walls of a system are not adiabatic, and energy is transferred in as heat, entropy is transferred into the system with the heat. Such a process is neither adiabatic nor isentropic, having $Q > 0$, and $\Delta S > 0$ according to the second law of thermodynamics.
\item Naturally occurring adiabatic processes are irreversible (entropy is produced).
\item The transfer of energy as work into an adiabatically isolated system can be imagined as being of two idealized extreme kinds.
\item In one such kind, no entropy is produced within the system (no friction, viscous dissipation, etc.), and the work is only pressure-volume work (denoted by $P dV$). 
\item In nature, this ideal kind occurs only approximately because it demands an infinitely slow process and no sources of dissipation.
\item The other extreme kind of work is isochoric work ($dV = 0$), for which energy is added as work solely through friction or viscous dissipation within the system. 
\item A stirrer that transfers energy to a viscous fluid of an adiabatically isolated system with rigid walls, without phase change, will cause a rise in temperature of the fluid, but that work is not recoverable. Isochoric work is irreversible. 
\item The second law of thermodynamics observes that a natural process, of transfer of energy as work, always consists at least of isochoric work and often both of these extreme kinds of work.
\item  Every natural process, adiabatic or not, is irreversible, with $\Delta S > 0$, as friction or viscosity are always present to some extent. 
\end{itemize}
\section{Thermodynamic Free Energy}
\begin{itemize}
\item The change in the free energy is the maximum amount of work that a thermodynamic system can perform in a process at constant temperature
\item Its sign indicates whether a process is thermodynamically favorable or forbidden. 
\item Since free energy usually contains potential energy, it is not absolute but depends on the choice of a zero point. 
\item Therefore, only relative free energy values, or changes in free energy, are physically meaningful.
\item Free energy is that portion of any first-law energy that is available to perform thermodynamic work at constant temperature, i.e., work mediated by thermal energy. 
\item Free energy is subject to irreversible loss in the course of such work.[1] Since first-law energy is always conserved, it is evident that free energy is an expendable, second-law kind of energy.
\item Several free energy functions may be formulated based on system criteria. Free energy functions are Legendre transforms of the internal energy. 
\item The basic definition of "energy" is a measure of a body's (in thermodynamics, the system's) ability to cause change.
\item The difference between the change in internal energy, which is $\Delta U$, and the energy lost in the form of heat is what is called the "useful energy" of the body, or the work of the body performed on an object. In thermodynamics, this is what is known as "free energy". In other words, free energy is a measure of work (useful energy) a system can perform at constant temperature. 
\end{itemize}
\subsection{Gibbs free energy}
\begin{itemize}
\item In thermodynamics, the Gibbs free energy is a thermodynamic potential that can be used to calculate the maximum of reversible work that may be performed by a thermodynamic system at a constant temperature and pressure. 
\item The Gibbs free energy  $\Delta G=\Delta H-T \Delta S$, measured in joules in SI) is the maximum amount of non-expansion work that can be extracted from a thermodynamically closed system (can exchange heat and work with its surroundings, but not matter). 
\item This maximum can be attained only in a completely reversible process. When a system transforms reversibly from an initial state to a final state, the decrease in Gibbs free energy equals the work done by the system to its surroundings, minus the work of the pressure forces
\end{itemize}

\subsection{Grand potential}
\begin{itemize}
\item The grand potential is a quantity used in statistical mechanics, especially for irreversible processes in open systems. The grand potential is the characteristic state function for the grand canonical ensemble. 
\item A grand canonical ensemble is the statistical ensemble that is used to represent the possible states of a mechanical system of particles that are in thermodynamic equilibrium with a reservoir
\item It is defined as: $$\phi_{G} = U - TS - \mu N$$
\item where $U$ is the internal energy, $T$ is the temperature of the system, $S$ is the entropy, $\mu$ is the chemical potential, and N is the number of particles in the system. 
\end{itemize}

\subsection{Helmholtz free energy}
\begin{itemize}
\item In thermodynamics, the Helmholtz free energy is a thermodynamic potential that measures the useful work obtainable from a closed thermodynamic system at a constant temperature and volume (isothermal, isochoric). 
\item The negative of the change in the Helmholtz energy during a process is equal to the maximum amount of work that the system can perform in a thermodynamic process in which volume is held constant. If the volume were not held constant, part of this work would be performed as boundary work. 
\item This makes the Helmholtz energy useful for systems held at constant volume. Furthermore, at constant temperature, the Helmholtz free energy is minimized at equilibrium. 
\item The Helmholtz energy is defined as: $$F = U - TS$$
\item where:
\begin{itemize}
\item $F$ is the Helmholtz free energy (sometimes also called "A") (SI: joules, CGS: ergs),
\item $U$ is the internal energy of the system (SI: joules, CGS: ergs),
\item $T$ is the absolute temperature (kelvins) of the surroundings, modelled as a heat bath,
\item $S$ is the entropy of the system (SI: joules per kelvin, CGS: ergs per kelvin).
\end{itemize}
\item The Helmholtz energy is the Legendre transformation of the internal energy U, in which temperature replaces entropy as the independent variable
\end{itemize}

\section{Thermodynamic Equillibrium}
Thermodynamic equilibrium is an axiomatic concept of thermodynamics. It is an internal state of a single thermodynamic system, or a relation between several thermodynamic systems connected by more or less permeable or impermeable walls. In thermodynamic equilibrium there are no net macroscopic flows of matter or of energy, either within a system or between systems. 

\subsection{Conditions for Thermodynamic Equillibrium}
\begin{itemize}
\item For a completely isolated system, S is maximum at thermodynamic equilibrium.
\item For a system with controlled constant temperature and volume, A is minimum at thermodynamic equilibrium.
\item For a system with controlled constant temperature and pressure, G is minimum at thermodynamic equilibrium.
\end{itemize}
The various types of equilibriums are achieved as follows:
\begin{itemize}
\item Two systems are in thermal equilibrium when their temperatures are the same.
\item Two systems are in mechanical equilibrium when their pressures are the same.
\item Two systems are in diffusive equilibrium when their chemical potentials are the same.
\item All forces are balanced and there is no significant external driving force.
\end{itemize}
\section*{Footnote}
In thermodynamics, \textbf{chemical potential} of a species is energy that can be absorbed or released due to a change of the particle number of the given species, e.g. in a chemical reaction or phase transition.
\end{document}

