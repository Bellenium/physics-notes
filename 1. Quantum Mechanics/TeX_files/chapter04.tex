\chapter{Systems with N degrees of freedom}
%\section{Schrodinger Equation in 3 Dimensions}
%\section{The Hydrogen Atom}
\section{N Particles in One Dimension}
\subsection{The Two-Particle Hilbert Space}
Consider two particles described classically by the coordinate system $\{(x_{1},p_{1}),(x_{2},p_{2})\}$. The rule for quantizing this system now is to promote them to be QM operators $()$ and $()$ that obey the relations:
\begin{equation}
\begin{cases}
[\hat{X}_{i}, \hat{P}_{j}] & = i \hbar \delta_{ij}\\
[\hat{X}_{i}, \hat{X}_{j}] & = 0 \\
[\hat{P}_{i}, \hat{P}_{j}] & = 0
\end{cases}
\end{equation}
In a few cases one might be able to extract all the information about the system simply from these equations. Usually they are represented in a basis. In this case we represent them in a basis of simulatneous eigenkets of the position operators

they are normalized,
\begin{equation}
	content...
\end{equation}
In this basis
We may interpret
We obviously could have chosen a differeny basis from any two arbitrary commuting operators. We generally denote this Hilbert space as $\mathbb{V}_{1 \otimes 2}$.
\subsection{$\mathbb{V}_{1 \otimes 2}$ as a Direct Product Space}
Another way to arrive at this space is to construct it out of two one-particle spaces. 
\subsection{$N$ particles in $d=1$}
\begin{itemize}
\item All the results in the previous sections generalize to an arbitrary N
\item The only exception being the problem for arbitrary N cannot be reduced to N independent one-particle problems by means of any set of coordinates
\item There are however exceptions to this including quadratic Hamiltonians which may be reduced to a sum over oscillator hamiltonians by the use of normal coordinates
\item In such cases the oscillators become independent and their energies add up in the classical and QM cases
\end{itemize}
W.k.t that can be decoupled if we use the normal coordinates
\begin{equation}
	x_{I,II} = \frac{x_{1} \pm x_{2}}{\sqrt{2}}
\end{equation}
and the corresponding momenta,
\begin{equation}
	p_{I,II} = \frac{p_{1} \pm p_{2}}{\sqrt{2}}
\end{equation}
So here's the algorithm
\begin{enumerate}
\item Rewrite $\hat{H}$ in terms of normal coordinates
\item Verify that the coordinates are canonical i.e.
\newcounter{enumTemp}
\setcounter{enumTemp}{\theenumi}
\end{enumerate}
\begin{equation}
\{x_{i}, p_{j}\} = \delta_{ij}
\end{equation}
\begin{enumerate}
\setcounter{enumi}{\theenumTemp}
\item Now, quantize the system by promoting these variables to operators obeying the equation
\newcounter{counterr}
\setcounter{counterr}{\theenumi}
\end{enumerate}
\begin{equation}
[\hat{X}_{i}, \hat{P}_{j}] = i \hbar \delta_{ij}
\end{equation}
\begin{enumerate}
	\setcounter{enumi}{\thecounterr}
\item Write the eigenvalue equation for $\hat{H}$ in the simultaneous basis of $X_{I}$ and $X_{II}$
\end{enumerate}
Analogously,
\begin{enumerate}
\item Qauntize the system directly by promoting the coordinate set $\{x_{1},x_{2},p_{1},p_{2} \}$ to a set of operators
\item Write the eigenvalue equation for $H$ in the simultaneous eigenbasis of $X_{1}$ and $X_{2}$
\item Now relabel the coordinates from $\{x_{1},x_{2}\}$ to $\{x_{I},x_{II}\}$
\end{enumerate}
\section{More Particles in More Dimensions}
\subsection{For $d=2$}
\begin{itemize}
\item Mathematically, a single particle in $d=2$ is equivalent to that of 2 particles in $d=1$
\item For the sake of convenience we use different notation in the two cases
\item The Cartesian coordinates of the two 
\end{itemize}
\subsection{For $d=3$}
\section{Identical Particles}
In this section we will apply the formalism that we have developed to identical particles (i.e. particles which are exact replicas of each other in terms of their intrinsic properties such as mass, charge and so on via experiment).
\subsection{The Classical Case}
\subsection{Two-Particle Systems: Symmetric and Antisymmetric States}
\subsection{Bosons and Fermions}
\subsection{Bosonic and Fermionic Hilbert Spaces}
